\documentclass[10pt, a4paper, oneside]{book}

\usepackage[
  left=1.5cm, 
  right=1.5cm, 
  top=2cm, 
  bottom=2cm, 
]{geometry}

% math mode additions
\usepackage{amsmath}
\usepackage{amsfonts}
\usepackage{amssymb}
\usepackage{mathtools} 

% nice derivatives
\usepackage[ISO]{diffcoeff}[=v4]

% in-document references
\usepackage{hyperref}
\usepackage{cleveref}

% Colors, f.ex. in mdframed boxes
\usepackage{xcolor}

% make todo-notes
\usepackage[
  textsize=scriptsize,
  textwidth=1.5cm,
  linecolor=blue,
  backgroundcolor=cyan,
  colorinlistoftodos
  ]{todonotes}

% Text boxes
\usepackage{framed}


% Strike through (sout), without normalem, it overwrites \emph :(
\usepackage[normalem]{ulem}

% to write 'goes to zero/infinity' in equatoations
\usepackage{cancel}


% Tabel of contents
\usepackage{tocloft}

% Placing graphics
\usepackage{graphicx}
\usepackage{caption}
\usepackage{subcaption}

% Better titlepage
\usepackage{titlesec}

% Bold math
\usepackage{bm}

\usepackage{slashed}

% Feynman diagrams
% \usepackage{feynmf}fy
\usepackage{feynmp-auto}
\setlength{\unitlength}{1mm}



% only subsection in toc
\setcounter{tocdepth}{0}

% Nice part page
\titleclass{\part}{top} % make part like a chapter
\titleformat{\part}
[display]
{\raggedleft\normalfont\large\bfseries}
{\vspace{3pt}\MakeUppercase{\partname} \thepart}
{0pt}
{\vspace{1pc}\huge\MakeUppercase}
\titlespacing*{\part}{0pt}{0pt}{20pt}



\usepackage{graphicx}
% \usepackage{ur l}


% \usepackage[%
%     style=numeric-comp, % Combine consecutive citations, e.g., [15]-[19]
%     sorting=none,       % Sorts citations after their appearance in the document
%     sortcites=true,     % Sorts within one "autocite", e.g., [15][17][44]
%     doi=true,
%     giveninits=true,    % use initials
%     hyperref,
%     % pages=false
%     ]{biblatex}

% \DeclareFieldFormat{pages}{#1}

% \AtEveryBibitem{\clearfield{issue}} % Zotero messes up number and issue field
% \AtEveryBibitem{
%     \clearfield{urlyear}
%     \clearfield{urlmonth}
% }

% New names for derivatives
\newcommand{\odv}[3][1]{\diff[#1]{#2}{#3}}
\newcommand{\pdv}[3][1]{\diffp[#1]{#2}{#3}}
\newcommand{\fdv}[3][1]{\diff.delta.[#1]{#2}{#3}}

% Shorthands
\newcommand{\dd}{\mathrm{d}}

% Expectation value
\newcommand{\E}[1]{\left \langle #1 \right \rangle}

% Fancy letters     
\newcommand{\Eff}{\mathcal F}
\newcommand{\D}{\mathcal D}
\newcommand{\Ce}{\mathcal C}
\newcommand{\A}{\mathcal A}
\newcommand{\Oh}{\mathcal O}
\newcommand{\V}{\mathcal V}
\newcommand{\Ci}{\mathcal C}
\newcommand{\N}{\mathcal N}
\newcommand{\He}{\mathcal H}
\newcommand{\R}{\mathbb R}
\newcommand{\Ell}{\mathcal L}
\newcommand{\Em}{\mathcal M}
\newcommand{\J}{\mathcal J}
\newcommand{\Es}{\mathcal S}
\newcommand{\Ge}{\mathcal G}
\newcommand{\Te}{\mathcal T}
\newcommand{\Pe}{\mathcal P}
\newcommand{\Ess}{\mathcal S}
\newcommand{\Essdot}{\mathcal{\dot S}}
\newcommand{\Eh}{\mathcal E}
\newcommand{\En}{\mathcal N}
\newcommand{\Je}{\mathcal J}

\newcommand{\C}{\mathbb C}

\newcommand{\br}{\mathbf}

\newcommand{\eps}{\varepsilon}

\newcommand{\const}{\mathrm{const.}}
\DeclareMathOperator{\sgn}{sgn}

\newcommand{\one}{\text{\usefont{U}{bbold}{m}{n}1}}
\MakeRobust{\one}

\newcommand{\stkout}[1]{\ifmmode\text{\sout{\ensuremath{#1}}}\else\sout{#1}\fi}







\title{Active Matter Notes}

% Subfolders containing chapter
\graphicspath{
    {chapters/}
}
\makeatletter
\def\input@path{
    {chapters/}
}
\makeatother


\begin{document}

    \maketitle
    \clearpage


    \tableofcontents

\vspace{5cm}
This document contains lecture notes from the Active Matter course held in the summer semester of 2024 at the University of Göttingen.
The lectures are organized by Benoît Mahault and Ramin Golestanian of the Max Planck Insitute of Dynamics and Self-Organization, featuring a range of guest lecturers: Luca Cocconi, Yuto Hosaka, Giulia Pisegna, Navdeep Rana, Suropriya Saha, and Gennaro Tucci. 
These notes are typeset by Martin Kjøllesdal Johnsrud, and
are meant to give a short introduction to the field of Active Matter Physics, while presenting some background material in the form of concepts and mathematical tools used to study what is known as \textit{active matter}.
    
    \clearpage

    % space instead of indentation for paragraphs 
    \setlength{\parindent}{0em}
    \setlength{\parskip}{0.8em}

    \chapter{Introduction - Lectures 1\&2}
    \label{chap_intro}
    \chapter{Introduction}

These are the notes from the Active Matter course held at Göttine University by Benôit Mahault and Ramin Golestanian of the Max Planck Insitute of Dynamics and Self-Organization in the spring of 2024.
They feature a range of guest lecturers and are typeset by Martin Kjøllesdal Johnsrud.

These notes are meant to give a short introduction to the Active Matter course, and present some background material in the form of concepts and mathematical tools that are often used to apprehend active systems.

\subsection{What is active matter?}

As widely accepted definition of active matter consists to say that it is made of elementary units that locally dissipate energy in a continuous and sustained manner. Hence, the dynamics of active particles break time-reversal symmetry, \textit{it is inherently far from thermodynamic equilibrium}. In many cases, activity manifests as persistent motion, but can also take the form of local production of forces, sustaining of chemical reactions, growth (reproduction), or several of them at the same time. This definition is very broad and encompasses a multitude of biological examples spanning many scales:
\begin{itemize}
    \item Enzymes, molecular motors (nm): Enzymes catalyse chemical reactions, and thus locally generate chemical gradients. Molecular motors couple chemical cycles to displacements or rotations, so that these molecular machines can exhibit self-propulsion.
    \item Bacteria, cells ($\mu$m): Bacteria and algae usually move thanks to filamentous appendices called flagella or cilia, which are powered by molecular motors. Another example are cells which self-propel on substrate by polymerizing actin filaments at their front.
    \item Insect swarms (mm)
    \item Animal herds, human crowds (m)
\end{itemize}

\subsection{What makes active matter special?}
Active systems, because they constantly dissipate energy in their bulk, and in contrast to passive systems globally driven (turbulence, shaken granulars, \ldots), are able to spontaneously self-organize and create macroscopic structures. 
Again, these ideas are relevant to many examples in the living world including the architecture of the cytoskeleton, the morphogenesis of multicellular organisms, or the collective motion arising from social interactions in groups of several hundred or thousands of individuals. 

Active matter is also not restricted to the study of biological systems, as the design and study of synthetic active particles now represents a large part of the field. Those allow for better controlled experiments, and often exhibit similar phenomenology as their biological counterpart, thus opening the way to biomimetic materials engineering. Notorious examples of artificial active matter include assemblies of self-propelled colloids (Phoretic Janus particles, Quincke rollers), artificial microswimmers (self-propelled drops, magnetic swimmers), and robots.

\subsection{On the nonequilibriumness of active assemblies}
As written above, active matter is inherently `far' from equilibrium, since it is assembled from particles whose dynamics constantly breaks time-reversal symmetry. This means that active systems lack many of the properties familiar to equilibrium systems. These include:
\begin{enumerate}
    \item Absence of minimization principle: active dynamics are not constrained by the second law, so that they do not converge over long times to a state with maximal entropy (or minimum free energy). In fact, whether an active system converges to a steady state or not is a priori unknown.
    \item No time-reversal symmetry: as a consequence of 1, active systems can (and do) generally produce macroscopic currents, leading to phases where TRS is collectively broken.
    \item Lack of thermodynamic framework: Basic equilibrium concepts such as temperature or thermodynamic pressure are usually not defined for active systems. Additionally, active systems generally do not exhibit an equation of state, meaning that their bulk properties are not only determined by their material properties but can also be substantially affected by the nature and shape of the container they live in.
    \item Memory: again, due to 1 and contrary to systems at equilibrium, the state of an active system may be dependent of its past history.
    \item In active matter, momentum is not conserved so that forces do not necessarily need to satisfy the action-reaction principle. Familiar examples of such `nonreciprocity' are encountered in predator-prey systems, while it also generally arises when particles interact via self-generated fields such as bacteria or phoretic colloids.
\end{enumerate}	
In summary, the collective behaviours found in active systems thus generally break our intuition built on equilibrium physics, and their theoretical understanding requires to use and develop specifically tailored tools and concepts.

\subsection{The idea of this course}
Active matter is very large and quickly developing field borrowing tools and concepts from various other areas of physics including soft matter, biological physics, hydrodynamics, and statistical physics. So, this course is only meant as an introduction to the basics of the very diverse phenomenology of active systems, and how it is usually apprehended. In particular, our journey will be driven by two general aspects which are still the topic of active research:
\begin{itemize}
    \item How is active motion generated at the micro scale, and what are its consequences on the dynamics of active particles?
    \item How do active systems convert energy dissipated at the molecular scale into the formation of structure and order macroscopic scales?
\end{itemize}
These two lectures is meant to give an overview of the topic that will be addressed in the following weeks, while we will also introduce useful concepts and mathematical tools.

\section{Hydrodynamics of microswimmers}

Many active particles evolve in a fluid, which they need to displace in order to move. Understanding the locomotion of such swimmers thus requires to study the dynamics of the flow around them. The fluid surrounding the particles is generally incompressible so that it is defined by its local velocity $\bm v$ which obeys the Navier-Stokes equations:
\begin{subequations}
\label{eq_NS}
\begin{align}
    \label{eq_NS_v}
    \rho \left[ \partial_t \bm v + (\bm v \cdot \nabla)\bm v \right] & = \eta \nabla^2\bm v - \nabla P + \bm f, \\
    \label{eq_NS_inc}
    \nabla \cdot \bm v & = 0.
\end{align}
\end{subequations}
The terms on the lhs of~\eqref{eq_NS_v} correspond to the material derivative and simply model the effect of inertia, while on the rhs the first term originates from viscous dissipation, the second is the pressure which ensures incompressibility~\eqref{eq_NS_inc}, while the active force $\bm f$ arises as a source term.
In cases dominated by inertia or dissipation, the NSEq.~\eqref{eq_NS} simplifies. To understand this, let us define the problem typical length and velocity scales $L$ and $V$, and apply the following rescaling:
\begin{equation*}
    \bm v \to V \bm v', \quad
    \bm x \to L \bm x', \quad
    t \to \frac{L}{V}t', \quad
    P \to \frac{\eta V}{L} P', \quad
    \bm f \to \frac{\eta V}{L^2} \bm f'.
\end{equation*}
We obtain the following equation in terms of the dimensionless primed variables
\begin{equation}
    {\rm Re}\left[ \partial_{t'} \bm v' + (\bm v' \cdot \nabla')\bm v' \right] = \nabla'^2\bm v' - \nabla' P' + \bm f',
\end{equation}
where ${\rm Re} = \rho V L / \eta$ is known as the Reynolds number. 
Alternatively, note that the Reynolds number can be obtained by the ratio of the typical scales associated with the convection ($\bm v\cdot \nabla \bm v$) and dissipation ($\eta\nabla^2\bm v$) in the Navier-Stokes equation. 
Hence, the value of the Reynolds number controls the relative importance of inertial effects and viscous dissipation.
For a micron-sized swimmer in water such as bacteria, we can evaluate $\rho \approx 10^3 \; {\rm kg / m^3}$, $\eta \approx 10^{-3} \; {\rm Pa \cdot s}$,
$V \approx 10^{-5}\; {\rm m/s}$, and $L \approx 10^{-6} \; {\rm m}$, such that ${\rm Re} \approx 10^{-5}$.
\textit{Microswimmer therefore evolve at vanishing Reynolds number} where inertial effects are negligible. Their dynamics then evolves in the so-called Stokes regime, 
such that the flow they generate obeys the Stokes equation:
\begin{subequations}
\label{eq_Stokes}
\begin{align}
    \label{eq_Stokes_v}
    \eta \nabla^2\bm v - \nabla P + \bm f & = \bm 0, \\
    \label{eq_Stokes_inc}
    \nabla \cdot \bm v & = 0.
\end{align}
\end{subequations}
Contrary to the Navier-Stokes equation, Eqs.~\eqref{eq_Stokes} do not have time derivatives and are linear.
The flow velocity $\bm v$ only therefore only depends on the instantaneous value of the force $\bm f$ (there is no inertia), and it varies linearly with it\footnote{By taking the divergence of~\eqref{eq_Stokes_v}, you can convince yourself that the pressure is also a linear function of the force.}. 
An important consequence of these features is that is that Stokes flow obey kinematic reversibility, meaning that they obey the symmetry
\begin{equation*}
    \bm f \rightarrow -\bm f, \qquad \bm v \rightarrow -\bm v.
\end{equation*}

[Further consequences: scallop theorem, no building of momentum (force and torque free swimmers...]

\section{A minimal model for active motion}

In the previous section, discussed by which mechanism active motion can arise (in the nontrivial context of microswimmers).
Here, we will be rather ask the question of what the consequences of the presence of a self-propulsion speed on the dynamics of an active particle are.
For this, assume that a particle is able to self-propel but we won't particularly focus on the details of its self-propulsion mechanism.

\subsection{Some basics on Brownian motion}

Before discussing active motion, let us give a brief recap on the theory of Brownian motion.
Consider a micron size particle in a fluid at temperature $T$. 
The particle constantly undergoes collisions with the fluid molecules, and its mass is low enough that the results of such collisions result in a visible erratic dynamics known as Brownian motion.
The dynamics of the particle is described by the Langevin equation
\begin{equation} \label{eq_LangevinBM}
    m\frac{\rmd^2 \bm r}{\rmd t^2} + \zeta \frac{\rmd \bm r}{\rmd t} = \bm f(t).
\end{equation}
Equation~\eqref{eq_LangevinBM} is nothing but Newton's second law, so that the first term on the lhs represents the acceleration of the particle multiplied by its mass, and the second term accounts for the Stokes friction exerted by the fluid in response to the motion of the particle.
In fact, the ration of the particle mass and friction coefficient defines a timescale $\tau = m / \zeta$ beyond which inertial effects can be neglected.
For a micron-size sphere with a density comparable to that of water, we have
\begin{equation*}
    \tau = \frac{\rho \tfrac{4\pi}{3} R^3}{6 \pi \eta R} 
    = \frac{2\rho R^2}{9\eta} \approx 0.2 \; \mu{\rm s}.
\end{equation*}
Hence, for observation times beyond the micro-second inertial effects can generally be safely neglected so that we can neglect the interaction term in~\eqref{eq_LangevinBM}, which can be expressed in this overdamped limit as
\begin{equation} \label{eq_LangevinBM_ovd}
    \frac{\rmd \bm r}{\rmd t} = \frac{1}{\zeta}\bm f(t).
\end{equation}

The force $\bm f$ arises from the multiple collisions that the particle undergoes with the solvent molecules. 
Under the approximation that these collisions are isotropic, identically distributed and uncorrelated random events, we can use the central limit theorem to approximate the distribution of the stochastic process $\bm f$ by a normal distribution with zero mean and variance $\sigma$.
Namely, denoting averages with independent realizations of the noise with the notation $\langle \cdot \rangle$, the first two moments of the components of $\bm f$ read
\begin{equation*}
    \langle f_i(t) \rangle = 0, \qquad
    \langle f_i(t) f_j(t') \rangle = \sigma^2 \delta_{ij}\delta(t - t').
\end{equation*}

Equation~\eqref{eq_LangevinBM_ovd} can be easily solved as
\begin{equation*}
    \Delta \bm r(t) = \bm r(t) - \bm r(t=0) = \frac{1}{\zeta} \int_0^t\rmd\tau \bm f(\tau).
\end{equation*}
Solution implies that $\langle \Delta \bm r(t) \rangle = \bm 0$ and $\langle |\Delta \bm r(t)|^2 \rangle = 2 d \sigma^2 / \zeta^2 t$.
Two main features of Brownian motion (diffusion): no net motion and MSD grows linearly in time (diffusion coefficient is $D = \sigma^2 / \zeta^2$).
Additionally, the fluctuation-dissipation relation relates $D$ to the solvent temperature as $D = k_{\rm B}T / \zeta$. The signification of this relation is that the damping friction and fluctuations come from the same medium.

Problem is sufficiently simple that one can actually obtain the full distribution of the particle position. For this, we define $\calP(\bm x,t) = \langle \delta(\bm x - \bm r(t)\rangle$, which obeys the Fokker-Planck equation:
\begin{equation} \label{eq_FP_BM}
    \partial_t \calP(\bm x,t) - D \nabla^2 \calP(\bm x,t) = 0
\end{equation}
Assuming that the particle sits at $\bm x = \bm 0$ at time $t=0$, the solution of Equation~\eqref{eq_FP_BM} is given by
\begin{equation*}
    \calP(\bm x,t) = \frac{1}{(4\pi D t)^{d/2}}e^{-\tfrac{|\bm x|^2}{4 D t}}.
\end{equation*}
Comments: no drift of the distribution, width grows as $\sqrt{t}$... 

\textit{Homework: Consider a Brownian particle in a potential $U$, write the associated Fokker-Planck equation. Show that the stationary solution corresponds to the Boltzmann distribution.
For a harmonic potential $U = \tfrac{1}{2} k |\bm x|^2$, the solution of the FPE is given by 
\begin{equation*}
    \calP(\bm x,t) = \left(\frac{\lambda}{2\pi D (1 - e^{-2\lambda t})}\right)^{d/2}e^{-\tfrac{\lambda}{2 D} \tfrac{|\bm x|^2}{1 - e^{-2\lambda t}}},
\end{equation*}
with $\lambda = k / \zeta$.
Calculate mean displacement and MSD, comment.
}

\subsection{Active Brownian motion}

Although we described the dynamics of a passive particle in the previous section, the Langevin description approach does not a priori assume that the particle is in equilibrium. 
Hence, we can straightforwardly generalize it by assuming that the particle is able to self-generate a force leading to an active velocity $\va$. 
Therefore, Equation~\eqref{eq_LangevinBM_ovd} now becomes
\begin{equation}\label{eq_LangevinABM}
    \frac{\rmd \bm r}{\rmd t} = \va + \frac{1}{\zeta}\bm f(t).
\end{equation}
To fully characterize the dynamics, we now need to specify how $\va$ evolves in time. 
Self-propelled motion can in fact arise in various ways, but what we argue below is that its most important feature is that $\va$ introduces a finite persistence length in the particle motion.

For example, the dynamics of bacteria like \textit{E. coli} consists of alternating sequences of `run' and `tumble', respectively corresponding to persistent motion along a fixed direction and random reorientations. 
Other active swimmers like self-propelled colloids, on the other hand, follow smooth trajectories as their direction of motion slowly diffuses in time due to thermal fluctuations.
In both cases, defining the typical particle velocity as $v_0$ and reorientation time $\tau_r$, the persistence length can be built from dimensional analysis as $\lp = v_0 \tau_r$.
Physically, this means that on length scales below $\lp$ the active particle exhibits ballistic motion with a speed $\approx v_0$, while on scales $\gg \lp$ its direction of motion has randomized and its overall dynamics is diffusive. 

As a minimal model of active velocity, we can thus simply assume that the particle self-propels at constant speed $v_0$, while the unit vector $\he$ parametrizing its orientation is a colored noise with zero mean and exponential correlations:
\begin{equation} \label{eq_cf_e}
    \langle \he(t) \cdot \he(t + \tau) \rangle = e^{-\tau / \tau_r}.
\end{equation}
% \textit{Homework problem:
% In two dimensions, the vector $\he(\theta) = (\cos\theta, \sin\theta)$ is parametrized by the angle $\theta_t$. Assuming that $\theta_t$ is a Markov process, justify why we can write the joint distribution as $P(\theta_2,t_2;\theta_1,t_1) = P(\theta_2-\theta_1,t_2-t_1|0,0)P(\theta_1,t_1)$ for $t_2 > t_1$.
% Deduce that
% \begin{equation*}
%     \langle \he(\theta_t) \cdot \he(\theta_{t+\tau}) \rangle = \int \rmd\phi \cos\phi \, P(\phi,\tau|0,0) \equiv \langle \cos\phi \rangle_0, 
% \end{equation*}
% %where $P(\theta,t)$ is the distribution of $\theta$ satisfying $P(\theta,0) = \delta(\theta)$.
% %(Hint: Use the properties of $\theta$ to express the joint distribution $P(\theta_2,t_2;\theta_1,t_1)$
% }
We then immediately deduce that
\begin{align}
    \langle |\Delta \bm r(t)|^2 \rangle & = 2 d D t 
    + v_0^2 \int_0^t\rmd\tau_1\int_0^t\rmd\tau_2 \, e^{-|\tau_1-\tau_2|/\tau_r} \nonumber \\
    & = 2 d D t 
    + 2 \lp^2 \left( \frac{t}{\tau_r} - 1 + e^{-t/\tau_r} \right).
\end{align}
Limiting regimes:
\begin{align*}
    \langle |\Delta \bm r(t)|^2 \rangle & 
    \underset{t \ll \tau_r}{\simeq} 2 d D t + v_0^2 t^2 , \\
    \langle |\Delta \bm r(t)|^2 \rangle & 
    \underset{t \gg \tau_r}{\simeq} 2 d D_{\rm eff} t , \qquad  D_{\rm eff} = D + \frac{v_0^2\tau_r}{d}. 
\end{align*}
Three dynamical regimes: thermal diffusion at short times ($t < 2 d D / v_0^2)$, ballistic dynamics at intermediate times ($2 d D / v_0^2 < t < \tau_r)$ and diffusive motion at long times $t > \tau_r$. 
Confirms the qualitative picture drawn earlier. 

\subsection{Description of the long-time regime via the moment expansion}

\section{Phase transitions and critical phenomena}

\subsection{A simple example: the Ising model}

\subsection{Continuous symmetry breaking: Goldstone modes and the Mermin-Wagner theorem}


    \chapter{Hydrodynamics - Lecture 3}
    \label{chap_hydro}
    %Notes from Yuto Lecture 3

    \chapter{Thermodynamics - Lecture 4}
    \label{chap_thermo}
    %%%%%%%%%%%%%
% Lecture 4 %
%%%%%%%%%%%%%

In this lecture, we give an overview of the thermodynamic tools that we have at our disposal for the characterization of active matter. 
As mentioned in Lecture 1, active matter defies thermodynamic equilibrium at the local scale. Some of the consequences of this property are: (i) the absence of a minimization principle for the steady-state statistics; (ii) the breaking of time-reversal symmetry of the steady-state dynamics; (iii) the absence of a generic thermodynamic framework. These are all related, as we will demonstrate shortly, via the concept of entropy production, which provides a measure of ``distance'' from equilibrium.


\section{Guessing the arrow of time}

Imagine being given the equation of motion of a particular process, after which a video of a sample realization of said process starts playing. How can you guess whether the video is currently being played in its original form, or whether it has been reversed?

Optimal decision theory, in particular Wald's ratio test \cite{PhysRevLett.115.250602}, tells us that the best time to make a guess between two competing hypotheses (here F=forward movie, B=backward movie) is the first occurrence of the quantity 
%
\begin{equation}\label{eq:Q_def}
    Q = \ln \frac{P(F|O)}{P(B|O)}
\end{equation}
%
reaching either of the pre-defined values $\pm Q^*$ (i.e. it is a first passage problem),
where $P(F,B|O)$ denotes the posterior probability of either F or B being true given an observed realization $O$ of the process. We can equivalently write,
%
\begin{equation}
    P(F|O) = \frac{1}{2} + \frac{\tanh(Q)}{2},\quad P(B|O) = \frac{1}{2} - \frac{\tanh(Q)}{2}~,
\end{equation}
%
whereby $Q^*$ can be translated into a given degree of confidence about our guess. Using Bayes' theorem,
%
\begin{equation}
    P(F|O) = \frac{\mathbb{P}(O|F)P(F)}{\mathbb{P}(O)}
\end{equation}
%
and similarly for $P(B|O)$,
where we introduced the notation $\mathbb{P}$ to highlight the fact that these are ``path probabilities'' associated with a particular trajectory $O = \{o(t)\}_0^\tau$ of duration $\tau$. We set our prior probabilities $P(F) = P(B) = 1/2$ and substitute into \eqref{eq:Q_def} to obtain
%
\begin{equation}\label{eq:Q_def_paths}
    Q[O = \{o(t)\}_0^\tau] = \ln \frac{\mathbb{P}(O|F)}{\mathbb{P}(O|B)}~.
\end{equation}
%
Now, since all possible sample trajectories, whether played backward or not, originate from the original dynamics, we conclude that $\mathbb{P}(O|B) = \mathbb{P}(\hat{\Omega} O|F)$ with $\hat{\Omega}$ the time-reversal operator, i.e. the probability of being shown a particular trajectory O \emph{given that it was time-reversed} is simply the probability of the corresponding time-reversed trajectory $\hat{\Omega}O$ in the forward ensemble. It should be noted that the operator $\hat{\Omega}$ acts differently on different types of observables, depending in particular on their \emph{signature} under time-reversal: even quantities, such as densities and position, are invariant under time-reversal, while odd quantities, such as velocities and magnetic fields, change sign.

The quantity $Q[O]$ appearing in \eqref{eq:Q_def_paths} is a random variable, being a functional of a random path. To determine the mean rate with which confidence about our guess of the arrow of time is built, we average with respect to the ensemble of possible observed trajectories and divide by the trajectory duration
%
\begin{equation}
    \dot{\bar{Q}} \equiv \frac{\bar{Q}}{\tau} = \frac{1}{\tau} \int \mathcal{D}O \ \mathbb{P}(O|F) \ln \frac{\mathbb{P}(O|F)}{\mathbb{P}(O|B)}~.
\end{equation}
%
The structure in the right-hand side can be recognized as the Kullbeck-Leibler divergence \cite{kullback1951information} per unit time of the forward and time-reversed ensembles. This is a standard measure of statistical distinguishibility, vanishing when the two path probability distributions are identical. \\

Let us now compute this quantity for a simple case, specifically one-dimensional Brownian motion in the presence of an external position-dependent force. The governing Langevin equation reads
%
\begin{equation}
    \dot{x}(t) = \mu F(x(t)) + \sqrt{2k_B T \mu} \eta(t)
\end{equation}
%
where $\eta$ is a zero-mean Gaussian white noise of unit covariance, $\langle \eta(t)\eta(t')\rangle = \delta(t-t')$. In discrete time,
%
\begin{equation}
    dx_t = \mu F(x_t) dt + \sqrt{2k_B T \mu dt} \eta_t
\end{equation}
%
with $\eta_t \sim \mathcal{N}(0,1)$ now a set of iid zero-mean, unit variance Gaussian random variables. Thus, the probability of a particular realization of the noise has a simple factorized form,
%
\begin{equation}
    \mathbb{P}[\{\eta_t\}] \propto \prod_{i=0}^{\tau/dt} {\rm exp}\left( - \frac{\eta_i^2}{2} \right) 
    = {\rm exp}\left( - {\sum}_i \frac{\eta_i^2}{2} \right)~.
\end{equation}
%
Given a particular initial condition $x(0)$, any trajectory $x(t \in [0,\tau])$ is uniquely defined by the corresponding noise realisation $\eta(t \in [0,\tau])$. Thus, for an observable $A[x(t)]$ depending on $x(t)$ we can write expectations as
%
\begin{equation}
    \langle A \rangle = \int \mathcal{D}\eta \ \mathbb{P}[\eta] A[x[\eta]] = \int \mathcal{D}x \ \underbrace{\mathbb{J}[x] \tilde{\mathbb{P}}[x]}_{\mathbb{P}[x]} A[x]
\end{equation}
%
where we have performed a change of variable by including the relevant Jacobian $\mathbb{J}[x] \equiv \mathcal{D}\eta/\mathcal{D}x$. Whether the latter is a trivial quantity or not, depends on the implicit convention about the interpretation of the relevant stochastic integrals (Ito vs Stratonovich, or generalizations thereof \cite{tauber2014critical}). This discussion falls beyond the scope of this course so we sweep this difficulty under the carpet and simply take for granted that $\mathbb{J}[x] = \mathbb{J}[\hat{\Omega}x]$ is invariant under time-reversal and that the Stratonovich mid-point convention can be assumed when looking at products of random variables (the only discretization that allows for straightforward use of the standard chair rule of calculus). With all these caveats out of the way, we may write
%
\begin{equation}
    \mathbb{P}[x|F] \propto {\rm exp}\left( - \frac{1}{2} \sum_i \frac{(dx_t - \mu F(x_t)dt)^2}{2k_B T \mu dt} \right) \propto {\rm exp}\left( - \frac{1}{4k_B T \mu} \int dt\  [\dot{x}(t) - \mu F(x(t))]^2 \right)
\end{equation}
%
having taken the continuum limit $dt \to 0$. Recalling that positions and velocities are even and odd under time reversal, respectively, we then have
%
\begin{align}
    \mathbb{P}(x|B) \propto {\rm exp}\left( - \frac{1}{4k_B T \mu} \int dt\  [-\dot{x}(t) - \mu F(x(t))]^2 \right)
\end{align}
%
whence
%
\begin{equation}\label{eq:q_vs_eqr}
    \dot{\bar{Q}} = \frac{\langle \dot{x}(t) F(x(t))\rangle}{k_B T}~. 
\end{equation}
%
Note how the numerator in the right-hand side of this expression is a product of the particle velocity times the force applied to it, and can thus be interpreted as the stochastic work done per unit time on the particle by the external force \cite{sekimoto1998langevin}. At steady-state, due to the conservation of energy, that same power needs to be dissipated as heat into the bath providing the thermal fluctuations. This quantity is thus nothing but the rate of entropy generation in the heat reservoir, $ \dot{\bar{Q}} = \dot{\sigma}$. This is a sign of the profound connection between entropy production as an informatic measure of time-reversal symmetry breaking, and its more traditional thermodynamic interpretation \cite{gaspard2004time}. In the following sections, we will follow a more traditional, i.e.\ thermodynamic, path for the derivation of this quantity.

Before moving on, let us note one important property of \eqref{eq:q_vs_eqr}. For conservative forces originating from an external potential that is bounded below and constant in time, $F(x) = -\partial_x U(x)$, we have by the chain rule that $\langle \dot{x}(t) F(x(t))\rangle = -\langle \dot{U} \rangle = 0$. In other words, the steady state of Brownian dynamics in a stable potential is equilibrium/time-reversal symmetric. 



% Handwritten notes section (2)

\section{Gibbs-Shannon entropy (axiomatically)}

We briefly motivate the form of the Gibbs-Shannon entropy, which we will use in the following section. This can be derived ``axiomatically'' by imposing the following requirements on the thermodynamic entropy $S(\{p_i\}_{i=1,...,M})$ of a statistical ensemble, assuming for simplicity a discrete state space:

\begin{enumerate}
    \item For separate systems A and B treated as a single system, $S(A \cup B) = S(A) + S(B)$. Statistically speaking, the systems being ``separate'' or ``non-interacting'' means that the joint probability $p(a \in A, b \in B) = p(a)p(b)$ factorises ;
    \item Continuity/smoothness with $p_i$;
    \item $S(\{p_i\})$ should have a minimum $S=0$ when $p_i = 1$, $p_{j\neq i} = 0$ for some $i$;
    \item $S(\{p_i\})$ should have a positive maximum when $p_i = 1/M$ for all $i$.
\end{enumerate}
The Gibbs-Shannon entropy satisfies all these constraints,
%
\begin{equation}
    S_{GS} = -k_B \sum_i p_i \ln p_i~.
\end{equation}
%
The ``modern'' view of thermodynamics ({\`a} la Callen) takes the entropy as the fundamental quantity. For example, the ideal gas equations of state can be derived from the Sackur-Tetrode entropy
%
\begin{equation}
    S = \frac{3}{2}N k_B \ln \epsilon + N k_B \ln V + N k_B \left[ \ln\left( N^{-1} \left( \frac{4\pi m}{2Nh^2}\right)^{\frac{3}{2}} + \frac{5}{2} \right) \right]~,
\end{equation}
%
with intensive quantities following from suitable differentiation, e.g.
%
\begin{equation}
    P \equiv T \frac{\partial S}{\partial V} = \frac{Nk_B T}{V}~.
\end{equation}
%
% Handwritten notes section (2)
Similarly, the Boltzmann distribution of equilibrium thermodynamics, which applies to open systems coupled to a heat reservoir, can be obtained by maximizing the Gibbs-Shannon entropy while keeping the mean energy fixed (max-ent principle). In particular, one can check that $p_i \propto e^{-\beta E_i}$ with $\beta^{-1} = k_B T$ maximises the functional
%
\begin{equation}
    \mathcal{L}[\{p_i\}] = -k_B \sum_{i} p_i \ln p_i + \lambda_1 \left(\sum_i E_i p_i - \epsilon \right) + \lambda_2 \left(\sum_i p_i - 1 \right) 
\end{equation}
%
where $\epsilon = \langle E_i \rangle$ is the mean energy, while $\lambda_1$ and $\lambda_2$ are Lagrange multipliers enforcing the constraint on the mean energy and normalization, respectively.



\section{Entropy production rate from Gibbs-Boltzmann entropy}

% Handwritten notes section (4)

We now demonstrate how a formula analogous to \eqref{eq:q_vs_eqr} can be derived from thermodynamic, rather than informatic, considerations, taking the Gibbs-Boltzmann entropy as a starting point. 
%The entropy production rate (EPR) is exactly what it sounds like, the amount of entropy produced per unit of time.
Given a system that evolves stochastically with time, which we assume to be discrete for simplicity, we can write the Markov master equation
%
\begin{align}
    \odv{  }{ t } P_i(t) = \sum_j K_{ij} P_j(t),
\end{align}
%
where $K_{ij}$ denotes the (Poisson) rate of transitioning from state $j$ to state $i$. The transition rates must obey $\sum_i K_{ij} = 0$ to conserve probability.
The corresponding Gibbs-Shannon entropy is
%
\begin{align}
    S(t) = - \sum_i P_i(t) \ln P_i(t),
\end{align}
%
which allows us to directly compute the rate of change of the entropy
%
\begin{align}
    \odv{  }{ t } S(t)
    \equiv \dot S(t)
    & = - \underbrace{{\sum}_i \dot P_i(t)}_{= 0}
    - \sum_i \dot P_i(t) \ln P_i(t)\\
    & = 
    -\frac{1}{2}\sum_{ij}
    \left(
        K_{ij}P_j \ln P_i
        + K_{ji} P_i \ln P_j
    \right) 
\end{align}
%
where we have inserted the master equation, and rewritten the sum as two copies where the dummy indices are renamed. Assuming that a statistical steady-state exists, we have that $\lim_{t \to \infty} \dot{S}(t)=0$ and in this limit $P_i$ converges to the eigenvector of $K$ with eigenvalue zero.
With some more massaging, we can write this as
%
\begin{align}
    \dot S(t) &=
    -\frac{1}{2}\sum_{ij}
    \left(
        K_{ij}P_j \ln P_i
        + K_{ji} P_i \ln P_j
    \right) \\
    & =
    -\frac{1}{2}\sum_{ij}
    \left(
        K_{ij}P_j \ln P_i
        - K_{ij} P_{j} \ln P_j
        + K_{ji} P_{i} \ln P_i
        + K_{ji} P_i \ln P_j
    \right) \\
    & =
    -\frac{1}{2}\sum_{ij}
    \left(
        K_{ij}P_j \ln \frac{P_i}{P_j}
        - K_{ji} P_i \ln \frac{P_i}{P_j}
    \right) 
    %\\
    %& 
    -\frac{1}{2}\sum_{ij}
    \left(
        K_{ij}P_j 
        - K_{ji} P_i 
    \right) \ln \frac{P_i}{P_j}\\
    & =
    -\frac{1}{2}\sum_{ij}
    \underbrace{
        ( K_{ij}P_j - K_{ji} P_i) 
        }_{{{\mathcal J}_{ij}}}
    \ln
    \left(
        \frac{K_{ji }P_i}{K_{ij}  P_j}
        \frac{K_{ij}}{K_{ji}}
    \right)
    \equiv \dot S_e + \dot \sigma~, \label{eq:epr_decomp}
\end{align}
%
where $\mathcal J_{ij}$ is the net probability current from state $j$ to state $i$.
In the last step, we have defined the external entropy production rate or entropy flow,
%
\begin{align}
    \dot S_e = - \frac{1}{2}\sum_{ij}
        ( K_{ij}P_j - K_{ji} P_i) 
    \ln \left( \frac{K_{ij}}{K_{ji}}\right) = - \sum_{i<j} \frac{\mathcal J_{ij} \Delta Q_{ij}}{k_b T}
\end{align}
%
which quantifies the rate of increase in total entropy due to heat dissipation from the system to the isothermal environment. To see this, we have invoked an important physical principle known as \emph{local detailed balance}, which formally amounts to demanding 
\begin{align}
    \frac{K_{ij}}{K_{ji}} = e^{\Delta Q_{ij} / k_b T}~,
\end{align}
i.e.\ that the ratio of forward and backward rates associated with a microscopically reversible transition between two states is equal to the exponential of the heat release in the transition, $\Delta Q_{ij}$, in units of the thermal energy scale. It should be noted that this relation is introduced on physical grounds and might not apply in all cases. In cases where it does apply, it can often be justified using Kramer's reaction rate theory.

Eq.~\eqref{eq:epr_decomp} additionally defines the total entropy production rate
%
\begin{align}
    \dot \sigma = 
    \frac{1}{2}\sum_{ij}
    \left(K_{ij}P_j - K_{ji} P_i\right) 
    \ln \left( \frac{K_{ij}P_j}{K_{ji} P_i}\right), \label{eq:epr_int_discr}
\end{align}
%
which obeys $\dot \sigma \geq 0$. To see that $\dot \sigma$ is non-negative, consider that $(a-b)\ln(a/b) > 0$ for all $a,b>0$.
Thus the total entropy production rate can be written as 
%
\begin{align}
    \dot \sigma = \dot S + \frac{\dot Q}{k_b T} \geq 0,
\end{align}
%
which is the second law of thermodynamics!
At a steady state, the probability distribution $P$ is independent of time, so $\dot S = 0$, which gives
%
\begin{align}
    \dot \sigma = \frac{\dot Q}{k_b T}~.
\end{align}
Thus, finite entropy production in a steady state (a signature of activity) must be accompanied by a finite rate of heat dissipation into the surrounding environment.
%
Indeed, it is also possible to show that \eqref{eq:epr_int_discr} corresponds to the Kullbeck-Leibler divergence per unit time of the ensemble of forwards vs backward paths on the discrete state space \cite{gaspard2004time}, completing the parallel with the calculation presented in the first part of this lecture. 




\section{(Non)conserved active field theories: fluctuation-dissipation}

% Handwritten notes section (5, 5b)


We now consider the thermodynamics of systems described at a more coarse grain level, using \emph{hydrodynamic theory}. 
In this case the relevant degrees of freedom are fields, which follow Langevin-field equations.
These are to Langevin equations as PDEs are to ODEs.
If we assume the density $\psi(\bm x)$ is not conserved, as is the case if we consider magnetization, driven quantum systems such as polariton Bose-Einstein condensates, or a density of cells that are born and die, we may generically write the equation of motion as
%
\begin{align}
    \partial_t \psi(\bm x, t) 
    = M[\psi(\bm x, t)] G_\psi[\psi(\bm x, t), \bm \nabla \psi(\bm x, t)] 
    + \sqrt{ 2 k_B T \tilde M[\psi(\bm x, t)] } \eta(\bm x, t).
\end{align}
%
In the simplest case, the mobilities $M$ and $\tilde{M}$ are constant, but in principle, they can have a more complicated dependence on the field.
For model A in the Halperin-Hohenberg (HH) classification~\cite{HohenbergRMP}, the potential takes the form $G = -(a \psi + b \psi^3 + \kappa \nabla^2 \psi)$ with $M=\tilde{M}=\gamma^{-1}$. 
%

In the case of, for example, a non-degrading chemical density, the order parameter $\rho$ is conserved, and its equation of motion is the conservation law (aka continuity equation)
%
\begin{align}
    \partial_t \rho(\bm x, t)
    = 
    \bm \nabla\cdot
    \left[
        M[\rho(\bm x, t)]
        \bm \nabla G_\rho[\rho(\bm x, t), \bm \nabla\rho(\bm x, t)]
        + \sqrt{ 2 k_B T \tilde M[\rho(\bm x, t)] }
        \bm \eta(\bm x, t)
    \right]~.
\end{align}
%
Here again, the mobilities are constant in the simplest case.
For model B in the HH classification, the forces are given by $\bm \nabla G_\rho = \bm \nabla (a \rho + b \rho^3 - \kappa \nabla^2 \rho)$ with $M = \tilde{M} = \gamma^{-1}$.

Model A is the canonical stochastic field theory for describing the spontaneous symmetry-breaking of the Ising model, while model B plays a similar role for the diffusive phase separation of a conserved scalar.
We capture both models at once by writing
%
\begin{align}
    \partial_t \phi(\bm x, t)
    = 
    \bm \nabla^{(k)}
    \left[
        M[\rho(\bm x, t)]
        \bm \nabla^{(k)} G_\rho[\rho(\bm x, t), \bm \nabla\rho(\bm x, t)]
        + \sqrt{ 2 k_B T  Q[\rho(\bm x, t)] }
        \bm \eta(\bm x, t)
    \right]
\end{align}
%
Here, $k = 0$ in the case of model A, and $k = 1$ in the case of model B.
Solving for the noise, we get
%
\begin{align}
    \eta
    = 
    \frac{[\nabla^{(k)}]^{-1} \partial_t \phi + M[\phi] \nabla^{(k)} G}{\sqrt{ 2 k_B T Q(\phi) }}.
\end{align}
%
We now draw on the Onsager-Machlup function for the path probabilities, which was introduced in the first part of the lecture, to calculate the average rate of entropy production associated with the field dynamics. Once again, the idea is to relate the probability of a certain history of the field to the probability of the corresponding noise realisation, keeping in mind that an additional integral over space needs to be introduced to account for the spatially-extended nature of the fluctuations,
%
\begin{align}
    \mathbb{P}_F[\phi] \propto
    \exp 
    \left\{ 
        - \frac{1}{4 k_B T}
        \int_0^\tau \dd t \int \dd^d r \, \frac{[(\nabla^{(k)})^{-1} \partial_t \phi - M(\phi) \nabla^{(k)}G]^2}{Q[\phi]}
     \right\}.
\end{align}
%
The probability for the reverse path is $\mathbb{P}_R[\phi, \dot \phi] = \mathbb{P}_F[\phi, -\dot \phi]$,  so the log ratio of the path probabilities becomes
%
\begin{align}
    \ln \frac{\mathbb P_F}{\mathbb P_R}
    = 
    \frac{1}{k_B T}
    \int_0^\tau \dd t\ \int \dd^d r \, \partial_t \phi [\nabla^{(k)}]^{-1}
    \left[
        \frac{M[\phi]}{Q[\phi]}
        \nabla^{(k)}G(\phi)
    \right].
\end{align}
%
In the case of equilibrium, the fluctuation-dissipation theorem (FDT) means that $M = Q$, and the forces are given by minimizing free energy, $G =\fdv{F}{\phi}$.
The integral may thus be easily solved,
%
\begin{align}
    \int_0^\tau \dd t\ \int \dd^d r \, \partial_t \phi [\nabla^{k}]^{-1}
    \left[
        \nabla^{(k)}\fdv{F}{\phi}
    \right]
    = 
    \int_0^\tau \dd t\, \odv{}{t} F(t)
    = 
    F(\tau) - F(0),
\end{align}
%
which means the path probabilities follow detail balance
%
\begin{align}
    \frac{\mathbb P_F}{\mathbb P_R} = e^{\Delta F / k_B T}.
\end{align}
%
Furthermore, as long as $F$ is bounded from below, the Kullbeck-Leibler divergence per unit time, $\langle F(\tau)-F(0)\rangle/\tau$, will vanish at steady state as $\tau \to \infty$ and necessarily so will the total entropy production $\dot\sigma$.
When either the FDT is not satisfied by our (a priori generic) choice of $M$ and $Q$, or the generalized force $G$ cannot be expressed as the functional derivative of a free-energy-like functional, all the above simplifications fail to apply and we should expect a non-vanishing value of $\dot\sigma$.

As an example of how this undestanding can help us build models of active matter, take a ``dry'' active field theory for a conserved density.
This can typically be written as a conservation law,
%
\begin{align}
    \partial_t \rho = - \bm \nabla \cdot \bm J,
\end{align}
%
where the current can be split into an active and a passive term, $\bm J = \bm J^a + \bm J^p$.
The passive terms alone are equilibrium effects and therefore have the form
%
\begin{align}
    \bm J^p = - M \bm \nabla \fdv{F}{\rho} + \sqrt{2 k_B T M} \bm \eta
\end{align}
%
satisfying both FDT and the derivative structure of the force.
Note that this might still be an ``effective'' passive term, i.e. the dynamics could still originate from activity at the microscopic level, but such activity might lead to effective equilibrium dynamics of the slow modes on a coarse-grained level.
The active current, on the other hand, can generically be written as
%
\begin{align}
    \bm J^a = \sum_i \lambda_i \times (\text{active terms}).
\end{align}
%
Here, $\lambda_i$ are sometimes referred to as activity parameters \cite{cates2022active}.

\subsubsection{Example: dry polar active matter}

For the case of dry polar active matter, we may write $\bm J^{a} = \lambda_1 \rho \bm p$ with $\bm p$ the local polar order parameter to indicate, e.g., that particles like to self-propel in the direction of their polarity, leading to additional currents in regions where particles are oriented in a similar direction. To close our system of equations, we then need to include a second Langevin-field equation
%
\begin{align}
    \partial_t \bm p + \lambda_2 \bm p \cdot \bm \nabla \bm p
    = - \Gamma \fdv{F}{\bm p} + \sqrt{ 2 k_B T\Gamma } \bm \eta.
\end{align}
%
This reduces to the ``Toner-Tu model'' of dry flocking, if we set $M\rightarrow 0$, with a particular choice of free energy
%
\begin{align}
    F = \int \dd x \, 
    \left(
        \frac{1}{2}a p^2 + \frac{1}{4}b p^4 + \frac{1}{2} \bm \nabla \bm p : \bm \nabla \bm p
        - \bar \omega \bm \nabla \cdot \bm p \rho + \frac{\lambda}{2 \Gamma} p^2 \bm \nabla \cdot \bm p
    \right)
\end{align}
%
``Wet'' active matter is more complicated, as we need to include the Navier-Stokes (NS) equation for the surrounding fluid.
We will say a bit about this later.

What we have done until now is a \emph{top-down approach}, where we add minimal active terms based on physical intuition and the symmetries of the microscopic description.
A \emph{bottom-up approach} typically starts with a description of the microscopic dynamics and proceeds via a formal coarse-graining procedure.
This is typically much harder, and you often end up with the same macroscopic theories, but with the additional advantage of having a microscopic interpretation of the terms in the hydrodynamic equations. In particular, it might be possible to establish a clear relationship between the effective parameters of the coarse-grained theory and those of the microscopic dynamics, essential for connection with experiments.

For dynamics near equilibrium, linear irreversible thermodynamics gives us a more systematic (although still top-down) recipe to construct hydrodynamic theories. We briefly review this approach in the following.


\section{A primer on linear irreversible thermodynamics}

The following is based on the more extended discussions in \cite{pottier2009nonequilibrium} and \cite{de2013non}. Consider the dynamics of a set of $N+M$ interacting fields $\{\phi_i\}_{i=1}^N\cup \{\rho_j\}_{j=1}^M$, where $\phi$ and $\rho$ denote non-conserved and conserved densities, respectively.
The rate of change of the energy $\mathcal{F} = U - TS$ under isothermal conditions is
%
\begin{align}
    \dot{\mathcal{F}} = \dot U - T \dot S
    = 
    -\int \dd r \, 
    \left[
        \sum_{i} \dot \phi_i \left(-\fdv{\mathcal{F}}{\phi_i}\right)
        + \sum_{j} \dot \rho_j \left(- \fdv{\mathcal{F}}{\rho_j}\right)
    \right].
\end{align}
%
Again, we assume that $\rho_j$ follows the conservation law
%
\begin{align}
    \partial_t \rho_j = \bm \nabla \cdot \bm J^{(c)}_j,
\end{align}
%
while the current of $\phi$, $\dot \phi_i = J^{(nc)}_i$, is non-conserved.
We define the effective/thermodynamic forces as
%
\begin{align}
    F_i^{(nc)} &= - \fdv{\mathcal{F}}{\phi_i} &
    \bm F_j^{(c)} &= - \bm \nabla \fdv{\mathcal{F}}{\rho_j}.
\end{align}
%
With this, and after integration by parts, the change in free energy takes the form of fluxes times force,
%
\begin{align}
    \dot{\mathcal{F}} = 
    - \int \dd r \,
    \left(
        \sum_i J_i^{(nc)} F_i^{(nc)} 
        + \sum_j \bm J_j^{(c)} \cdot \bm F_j^{(nc)}
    \right).
\end{align}
%
Given that the free energy $\mathcal{F}$ is invariant under time reversal and assuming well-defined parity under time reversal of the fields $\phi$ and $\rho$, it follows
that the effective forces also have well-defined parity under time reversal. 
On the other hand, the fluxes $J^{(nc)}$ and $J^{(c)}$ are as of yet unconstrained and can in general have both contributions
%
\begin{align}
    J_i^{(nc)} &=  J_{R,i}^{(nc)} + J_{D,i}^{(nc)},  &
    \bm J_j^{(c)} &=  \bm J_{R,j}^{(c)} + \bm J_{D,j}^{(c)}.
\end{align}
%
Here we have split the currents into ``reactive'' fluxes, which have the opposite time-reversal signature as the associated force, thus contributing to $\dot U$, and ``dissipative'' fluxes, which have the same time-reversal signature of the associated force and contribute to $T \dot S$.


At equilibrium, all forces and fluxes vanish.
In a ``small neighborhood'' of the equilibrium state, we will assume the existence of linear phenomenological laws.
These take the form
%
\begin{align}
    J_{\bullet,i}
    = L_{ij}(\{\phi_i\}, \{\rho_j\}) F_j, 
\end{align}
%
with $\bullet \in \{R,D\}$ and are called \emph{constitutive equations}. Note that we have dropped the superscript distinguishing currents of (non)conserved fields for the sake of readability.
Here $F_j$ enumerates generalized forces acting on both conserved and non-conserved fields: it is \emph{not} the case that currents of conserved fields can only depend on effective forces acting on conserved fields, and similarly for non-conserved fields. However, it \emph{is} the case that current contributions with a given signature under time reversal can only depend on generalized forces with the same signature. 

The matrix $L_{ij}$ is made up of \emph{transport coefficients}, which can in general be functions of the fields themselves.
Again, in these laws, all couplings with equal time-reversal signatures are allowed.
The fact that $\mathcal{F}$ is minimized at equilibrium implies that $L$ should be positive definite, since we can write
\begin{equation}
    \dot{\mathcal{F}} = - F_i L_{ij} F_j \leq 0~.
\end{equation}
The possible values of the transport coefficients $L_{ij}$ can be further constrained based on the following physical principles:
\begin{itemize}
    \item {\bf Curie's principle} states that the tensorial structure of currents and forces must be matched in the constitutive equation, e.g. vector currents need to be a sum of vector-like objects that transform in the same way under rotation, etc.
    \item Furthermore, the {\bf  Onsager-Casimir relations}, rooted in the microscopic reversibility of Hamiltonian dynamics, state that dissipative fluxes must have symmetric (under transposition) couplings, while reactive fluxes must have antisymmetric couplings.
    In other words,
    %
    \begin{equation}
        L_{ij}  = \epsilon_i \epsilon_j L_{ji},
    \end{equation}
    %
    where $\epsilon_i = \pm 1$ are the parity of the associated force.
\end{itemize}

\subsubsection{Example: active gel theory}

As an example, take the case of a ``wet'' system consisting of a fluid (with velocity $v$) coupled to a non-conserved polar field (q) and to the non-conserved scalar density of chemical species, e.g. ATP (r). This is sometimes referred to as an active gel theory \cite{kruse2005generic}.
For the purpose of this example, let us just assume that the time derivative of the free energy can be written as 
%
\begin{align}
    \dot{\mathcal{F}} = - \int \dd r 
    \left(\sigma_{\alpha \beta} u_{\alpha \beta} + h_\alpha D_t q_\alpha + \Delta_\mu \dot r\right)
\end{align}
%
where $\sigma$ is the stress tensor, $u$ is the symmetrized strain rate tensor, $h$ is the crystal field, $D_t q$ is the comoving \& corotational time derivative of the local polarity, $\Delta\mu$ is the chemical free energy associated with the removal of a particle of chemical from the system and $\dot{r}$ is the instantaneous rate of particle removal.
Categorizing this as above, we have

\begin{table}[h]
\centering
\begin{tabular}{c|c|c}
    Fluxes & Forces & Parity of Force \\
    \hline
    $\sigma_{\alpha \beta}$ 
    & $u_{\alpha \beta} = \frac{1}{2}(\partial_\alpha v_\beta + \partial_\beta v_\alpha)$ 
    & $\epsilon = -1$\\
    $D_t q_\alpha$ &
    $\fdv{F}{q_\alpha} = h_\alpha$ & $\epsilon = +1$ \\
    $\dot r$ & 
    $\fdv{F}{r} = \Delta_\mu$ & 
    $\epsilon = +1$ \\
\end{tabular}
\end{table}

We split the fluxes into the reactive and diffusive parts, according to the parity under time-reversal of the corresponding force
%
\begin{align}
    \sigma_{\alpha \beta} 
    & = \underbrace{\sigma_{\alpha \beta}^{(R)}}_{\epsilon=+1}
    + \underbrace{\sigma_{\alpha \beta}^{(D)}}_{\epsilon=-1}\\
    D_t q_\alpha
    & = \underbrace{D_t p_\alpha^{(R)}}_{\epsilon=-1}
    + \underbrace{D_t p_\alpha^{(D)}}_{\epsilon=+1}\\
    \dot r
    & = \underbrace{\dot r^{(R)}}_{\epsilon=-1}
    + \underbrace{\dot r^{(D)}}_{\epsilon=+1}.
\end{align}
%
We now proceed to introduce linear phenomenological relations by writing fluxes as linear superpositions of forces with suitable time-reversal signatures. We see immediately that, for example, $\sigma^{(D)} \propto u$, since this is the only force that is odd under time reversal, as required.
Following similar arguments, we must have
%
\begin{align}
    D_t q_\alpha^{(D)} &= \lambda_{qq} h_\alpha + \lambda_{qr} \Delta_\mu \\
    \dot r^{(D)} &= \lambda_{rr} \Delta_\mu + \lambda_{rq} h_\alpha.
\end{align}
%
By Onsager reciprocity, we further need to impose $\lambda_{pr}(r, q) = \lambda_{rp}(r, q)$, and by Curie's principle, we also demand that $\lambda_{q r} \propto q$, as each term in the right-hand side of the first (second) equation needs to be a vector (scalar).
On the other hand, $D_t q_\alpha^{(R)}$ and $\dot r^{(R)}$ cannot depend on $h$ and $\Delta_\mu$ due to time-parity mismatch. They will thus only depend on $u$ via coefficients involving the polarity as required to combine with $u$ into a vector and a scalar respectively.
Using the Onsager relations, we get, for the terms coupling fluid velocity and chemical reaction rate
%
\begin{align}
    \sigma_{\alpha \beta}^{(R)} 
    &= \dots
    \left(- \xi p_\alpha p_\beta - \bar \xi \delta_{\alpha \beta} - \xi' p_\gamma qp_\gamma \delta_{\alpha \beta}\right)
    \Delta_\mu, \\
    \dot r^{(R)} 
    &= -
    \left(- \xi p_\alpha p_\beta - \bar \xi \delta_{\alpha \beta} - \xi' p_\gamma p_\gamma \delta_{\alpha \beta}\right)
    u_{\alpha \beta}~.
\end{align}
As one might suspect, this approach to constructing a field theory typically leads to models that are somewhat too generic to be amenable to analytical treatment. Approaches based on linear irreversible thermodynamics often proceed by reducing the number of parameters by physically arguing for the irrelevance of some of the ``off-diagonal'' couplings introduced along the way. 
 

    \chapter{Scalar active matter - Lecture 5}
    \label{chap_scalar}
    \section{Introduction}

Scalar active systems are those whose large-scale dynamics is well captured by a real (scalar) field, hereafter denoted $\rho$. 
As we have seen in chapter~\ref{chap_intro}, the field $\rho$ is usually referred to as the \emph{order parameter} of the theory.
While it can in general represent various quantities related to the dynamics, here it will be most of time given by the particle density.

% At equilibrium, the minimal continuous description of scalar systems with conserved order parameter is achieved via \emph{model B} (see~\cite{HohenbergRMP} for a comprehensive classification).
% Model B indeed describes a class of equilibrium systems whose dissipative dynamics is captured by a conserved field, such that $\intd{r} \, \phi(\bm r,t) = {\rm const}$, for all times $t$,  where hereafter the variables $\bm r$ and $t$ account for space and time, respectively, while $d$ denotes the number of spatial dimensions.
% As model B describes the universal large-scale features of many systems, it can be formulated based on relevant symmetries and conservation laws.
% Such a phenomenological approach can moreover be supplemented by direct coarse-graining from microscopic (particle-based) theories, which provide additional physical insights. 

% In these notes, we will apply both approaches---phenomenological and coarse-graining---to study the dynamics of scalar active systems.
% We begin with a bottom-up approach
% In \autoref{chapter: introduction}, we introduced the active Brownian particle (ABP). 
% By extending this model to a collection of interacting ABPs, we will employ coarse-graining techniques to obtain one of the paradigmatic examples of collective active matter, motility-induced phase separation (MIPS).
% Here we will review some of the essentials of equilibrium phase separation physics, what changes for methods that do not admit an equilibrium description, and discuss what is needed for this to be the case.
% We will then take a more phenomenological approach to derive the Active Model B (AMB) and the Non-Reciprocal Cahn-Hilliard (NRCH) model, and explore their rich out-of-equilibrium features.



\section{Interacting Active Brownian particles}

In Chapter~\ref{chap_intro}, we have introduced a simple model of noninteracting active Brownian particles. 
Here, we consider the simplest extension of this model by assuming that we have isotropic particles that interact via volume exclusion. 
For simplicity, we restrict our analysis to two spatial dimensions, most of what we present below will also hold in $d=3$.

We thus have $N$ active particles defined by their positions $\bm r_i$ and self-propulsion orientations $\theta_i$, $i \in \{1, \dots N\}$.
As we have seen before, we can write their dynamics in terms of overdamped Langevin equations
% The dynamics of each of these particles are described by an underdamped Langevin equation with an active velocity, as in \autoref{chapter: introduction}.
% Without interaction, it 
% %
% \begin{align}
%     \odv{\bm r_i}{t} = \bm v_{i,a}(t) + \sqrt{ 2 D_t } \bm \xi_i(t),
% \end{align}
% %
% where $\bm \xi_i(t)$ is white Gaussian noise, which obey
% %
% \begin{align}
%     \E{\xi_{k,i}(t)} &= 0, &
%     \E{\xi_{k,i}(t)\xi_{k',j}(t')} = \delta_{ij} \delta_{kk'} \delta(t - t').
% \end{align}
% %
% In two dimensions, the active velocity may be parametrized in terms of a single angle, $\bm v_a(t) = v_0 \hat {\bm e}(\theta(t))$, where 
% %
% \begin{align}
%     \hat {\bm e}(\theta) 
%     =
%     \begin{pmatrix}
%         \cos \theta \\ \sin \theta
%     \end{pmatrix}.
% \end{align}
%
%We focus on times much larger than the time scale of the rotational noise, $t\gg \tau_r = 1 / D_r$, so we may consider the rotational noise $\xi_i(t)$ to be white Gaussian noise as well.
%Including the interaction potential $U$, the full equations of motion for each particle becomes
%
\begin{subequations}
\label{eq_Langevin_int_ABPs}
\begin{align}
    \odv{\bm r_i}{t} & = v_0 \hat {\bm e}(\theta_i) - \frac{1}{\zeta} \nabla_{\bm r_i} U(\{\bm r_j\}) + \sqrt{ 2 D } \bm \xi_i, \\
    \odv{\theta_i}{t} & = \sqrt{ 2 D_r } \chi_i,
\end{align}
\end{subequations}
%
where $\zeta$ and $D$ correspond respectively to the friction and diffusivity and are related by the relation $D = k_B T / \zeta$,
while $D_r$ denotes the rotational diffusivity of the self-propulsion direction.
As before, the noises in~\eqref{eq_Langevin_int_ABPs} are Gaussian with zero mean and satisfy
\begin{equation*}
    \E{\xi_{k,i}(t)\xi_{k',j}(t')} = \delta_{kk'}\delta_{ij} \delta(t - t'), \qquad
    \E{\chi_{k}(t)\chi_{k'}(t')} = \delta_{kk'} \delta(t - t'). 
    %\qquad k,k' \in \{1,\ldots,N\}, \quad i,j \in \{1,2\}.
\end{equation*}
The interactions between the particles are modeled via a potential $U$, which we assume of the form
%
\begin{align*}
    U(\bm r_1, ..., \bm r_n) = \sum_{i \neq j} u(|\bm r_i - \bm r_j|).
\end{align*}
%
In the following, we also assume that $u$ is short-ranged, isotropic and repulsive: two particles only interact when they get in contact, while the amplitude and direction of the resulting force only depend on the vector joining their centers of mass.
The simplest choice satisfying these criteria is the hard-core potential 
\begin{equation*}
    u_{\rm HC}(r) = \begin{cases} +\infty & {\rm if}\; r < d_0\\
        0 & {\rm otherwise} \end{cases} ,
 \end{equation*}
with $d_0$ the particle diameter, 
that prevents any overlap between the particles.  
In practice --i.e. for simulations-- the hard core potential is often approximated via the Weeks-Chandler-Andersen potential (or truncated Lehnnard-Jones potential):
\begin{equation*}
    u_{\rm WCA}(r) = \begin{cases} 4\epsilon\left[ \left(\frac{d_0}{r}\right)^{12} - \left(\frac{d_0}{r}\right)^{6} \right] + \epsilon & {\rm if}\; r < 2^{1/6} d_0\\
        0 & {\rm otherwise} \end{cases} .
 \end{equation*}
Alternatively, one can also `allow' for finite overlap between particles by using a softer repulsion potential:
\begin{equation*}
    u_{\rm soft}(r) = \begin{cases} \tfrac{k}{2}(d_0 - r)^2 & {\rm if}\; r < d_0\\
        0 & {\rm otherwise} \end{cases} .
 \end{equation*}
 All the aspects that we describe below are qualitatively independent of the specific choice of the interaction potential.

\begin{figure}[!t]
    \centering
    \includegraphics[width=.4\textwidth]{chapters/Figures/scalar/scalar_01.jpg}
    \caption{Illustration of the possible choices for the potential $u$. A hard repulsion corresponds to $\lim_{r\to0} u(r) = +\infty$, such that the particles can never fully overlap. On the other hand, for the soft repulsion the maximum value of the potential is finite, such that overlaps are always possible.}
    \label{fig: hard soft}
\end{figure}


In \autoref{chapter: introduction}, we wrote down the Fokker-Planck equation for the single particle, \autoref{eq_FP_BM}, which describes the time evolution of the probability distribution of the system, $\calP(\bm x, t) = \E{\delta(\bm r(t) - x)}$.
We can likewise write the Fokker-Planck for a single active Brownian particle, whose probability-density also depends on the angle of said particle
%
\begin{align}
    \calP(\bm x, \phi, t)
    =
    \E{\delta(\bm x - \bm r(t))\delta(\phi - \theta(t))}.
\end{align}
%
This obeys the Fokker-Planck equation
%
\begin{align}
    \partial_t \calP(\bm x, \phi, t)
    + \bm \nabla \cdot [
        v_0 \hat {\bm e}(\phi ) \calP(\bm r, \phi, t)
        - D \bm \nabla \calP(\bm x, \phi, t)
    ]
        - D_r \partial_\phi^2 \calP(\bm x, \phi, t)
        = 0,
\end{align}
%
which describes how the probability distribution of a single ABP, or equivalently density of a large number of non-interacting ABPs, evolves with time.

When we consider $N$ \emph{interacting} particles, however, the picture becomes significantly more complicated.
We now have to consider the $N$-body distribution,
%
\begin{align}
    \calP_N(\{\bm x_i, \phi_i\}, t) = \E{ \prod_i \delta(\bm r_i(t) - \bm x_i)\delta(\theta_i(t) - \phi_i) },
\end{align}
%
and its corresponding Fokker-Planck.
If the particles are non-interacting, the probability distribution simply factors into $N$ independent one-body distributions, while interactions introduce non-trivial correlations.
One way of tackling this is the Bogoliubov-Born-Green-Kirkwood-Yvon (BBGKY) hierarchy, which is described in \autoref{appendxi: BBKGY}.
We will instead make a simplifying mean-field assumption about the effects of the interaction.

Consider a single particle moving freely, far from any other particles.
This will move at a velocity close to the self-propulsion velocity.
If, instead, the particle is in an area with a high density, the other particles will act as barriers due to their repulsive interactions, slowing our particle down.
As an analogy, imagine the difference between running in an open feel versus trying to move in a crowd at a concert.
With this picture in mind, we assume that the effective self-propulsion speed will be lowered as the local density of particles increases, and replace the constant self-propulsion speed with one that depends on the density $\rho$,
%
\begin{align}
    v_0 &\rightarrow v(\rho), &
    \odv{v}{\rho} & < 0,
\end{align}
%
and neglect any other effects of the interaction between the particles.
The Fokker-Planck equation for the one-particle density is then
%
\begin{align} \label{eq: mips fokker planck}
    \partial_t \calP(\bm r, \phi, t)
    + \bm \nabla \cdot [
        v(\rho) \hat {\bm e}(\phi ) \calP(\bm r, \phi, t)
        - D \bm \nabla \calP(\bm r, \phi, t)
    ]
        - D_r \partial_\phi^2 \calP(\bm r, \phi, t)
        = 0,
\end{align}
%
This equation has no simple analytical solution.
However, in the regime where $t\gg 1 / D_R \equiv \tau_R$ it may be described well using \emph{moment expansion}.
This is a powerful technique is is also very useful for more complex situations.


\subsection{Spatial variation}

Before we start with the momentum expansion, we consider the case where the local velocity depends on space directly, not through the density of particles.
This can be achieved experimentally by engineering the environment of active particles, which leads to rich behavior.\todo{add sources, discuss more?}
The Fokker-Planck is then
%
\begin{align} \label{eq: mips fokker planck}
    \partial_t \calP(\bm x, \phi, t)
    + \bm \nabla \cdot [
        v(\bm x) \hat {\bm e}(\phi ) \calP(\bm x, \phi, t)
        - D \bm \nabla \calP(\bm x, \phi, t)
    ]
        - D_r \partial_\phi^2 \calP(\bm x, \phi, t)
        = 0.
\end{align}
%
A steady state configuration $P_{SS}$ obeys by assumption $\partial_t P_{SS} = 0$.
Assuming it is isotropic, so $\partial_\phi P_{SS} = 0$, and that diffusion $D$ is small compared to the active velocity, then the steady-state distribution obeys \todo[noinline]{Are these the right assumptions?}
%
\begin{align}
    \bm \nabla \cdot [v(\bm x) \hat {\bm e}(\phi ) \calP(\bm x, \phi, t)] = 0
    \implies 
    \calP(\bm x, \phi, t) \propto \frac{1}{v(\bm x)}.
\end{align}
%
The particles thus clump together where the active velocity is high, leading to higher density, while in the regions with high active velocity, the density is low.

\begin{figure}[!htb]
    \centering
    \includegraphics[width=.4\textwidth]{chapters/Figures/scalar/scalar_02.jpg}
    \caption{Perturbation in density gives perturbation in velocity}
    \label{fig: perturb}
\end{figure}

We now consider what happens if we pertrub a homogenous state, $\rho= \bar \rho + \delta \rho$.
Then, the velocity changes as $v(\rho) = v(\bar \rho) + \delta \rho v'(\bar \rho)$.
This will again perturb the density further, $\delta \rho'$.
Inserting this into the steady state, with $\calP = \rho$, we get
%
\begin{align}
    \bar \rho + \delta \rho'
    \propto 
    \frac{1}{v(\bar \rho) + \delta \rho v'(\bar \rho)}
    \sim \frac{1}{v(\bar \rho)} \left( 1 - \frac{\delta \rho v'(\bar \rho)}{v(\bar \rho)} \right)
    =
    \rho\left( 1 - \frac{\delta \rho v'(\bar \rho)}{v(\bar \rho)} \right)
    .
\end{align}
%
This gives a criterion for whether the perturbations are growing in strength or diminishing in strength.
If $\delta \rho' > \delta \rho$, the system is unstable.
This happens if
%
\begin{align}
    \frac{\bar \rho v'(\bar \rho)}{v(\rho)} < -1.
\end{align}
%




\subsection{Moment expansion}

We begin by defining the particle density,
%
\begin{align}
    \rho(\bm x, t) = \int_0^{2\pi} \dd \phi\, \calP(\bm x, \phi, t).
\end{align}
%
We now seek to find the equation of motion of this field.
By integrating the Fokker-Planck for the one-particle distribution over $\phi$, i.e. applying $\int \dd \phi$ to  \autoref{eq: mips fokker planck}, we obtain
%
\begin{align}\label{eq: density FP}
    \partial_t \rho 
    = 
    - \bm \nabla \cdot [ v(\rho) \rho \bm p - D \nabla \rho ].
\end{align}
%
The radial diffusion term $\propto D_r$ vanishes as it is a
where we have defined the orientational order parameter
%
\begin{align}
    \bm p(\bm x, t)
    =
    \frac{1}{\rho}
    \int_0^{2\pi} \dd \phi \, \hat {\bm e}(\phi) \calP(\bm x, \phi, t).
\end{align}
%
This measures how much the particles align.
If there is no alignment, then $\cal P$ is constant as a function of $\phi$, and the integral vanishes.
If all particles are oriented in some direction $\phi = \phi_0$, then $\calP \propto \delta(\phi - \phi_0)$  and $\rho = \hat {\bm e}(\phi_0)$ is a unit-vector pointing in the direction of alignment.
% Handwritten page XI
In the same manner, we find the equation of motion for $\bm p$ by applying $\int \dd\phi \, \hat{\bm e}(\phi)$ to \autoref{eq: mips fokker planck}, which yields
%
\begin{align} \label{eq: polarity FP}
    \partial_t (\rho  p_i)
    =
    -
    \nabla_{j}
    \left[
        v(\rho) \rho \left( Q_{ij} + \frac{ 1 }{ 2 }  \delta_{ij}\right)
        - D \nabla_j (\rho p_i)
    \right]
    - D_r \rho p_i.
\end{align}\todo[noinline]{check indices}
%
The last term is obtained by integration by parts.
In this equation, we are using the Einstein summation convention, which means that repeated indices are summed over.
We have defined the tension, also called the nematic order parameter,
%
\begin{align}
    \rho Q_{ij}
    = \int_0^{2 \pi} \dd \phi
    \left[
        \hat e_i(\phi)\hat e_j(\phi) - \frac{1}{2} \delta_{ij}
    \right] \calP(\bm r, \phi, t).
\end{align}
%
The Krönecker delt term ensures that the nematic tensor is traceless, $\text{Tr}(Q) = 0$.
This quantity measures the \emph{nematic} alignment of particles, which means it neglects whether the particles are aligned or anti-aligned.
\todo[inline]{Draw picture of nematic alignment}
\noindent{\it Exercise: Explain why the matrix $\bm Q$ is defined as traceless (${\sum}_i Q_{ii} = 0$). Hint: Try to calculate its expression assuming a uniform distribution of orientations.}\\

As we can see, this expansion is a runaway process, and we need a criterion for closing this hierarchy of equations.
When we do coarse-graining, we are interested in the large-scale, long-term behavior of the system.
We therefore remove terms that are fourth order or more in gradients.
For example, the lowest order contribution to $\partial_t \rho $ is the diffusion term, so $\rho = \Oh(\nabla^2)$.
If we were to write out the equation for the nematic tensor, we would see that $Q = \Oh(\nabla)$, so $\nabla_j [v(\rho) \rho Q] = \Oh(\nabla^4)$ and is therefore neglected.
This closes the hierarchy of equations, and we are left with
%
\begin{align}
    \label{eq: closed density}
    \partial_t \rho &= - \nabla_i [v(\rho) \rho p_i - D \nabla_i \rho], \\
    \label{eq: closed polarity}
    \partial_t (\rho p_i)
    & = 
    - \nabla_j \left[\frac{1}{2} v(\rho) \rho \delta_{ij} - D\nabla_j (\rho p_i)\right] - D_r \rho p_i.
\end{align}
%
As all terms in the density equation are proportional to at least one gradient term, the only time scale in this equation in this equation is the system size, $\partial_t^{-1} \sim \nabla^{-1} \sim L$, which diverges in the thermodynamic limit $L\rightarrow\infty$.
We therefore expect this dynamics to be slow.
The equation for the polar order parameter $p_i$, on the other hand, has a time scale inherited from the rotational diffusion, $\tau_r = 1 / D_r$.
In this case, we expect any non-zero polar order to relax on a time-scale $\tau_r$, which given a large enough system will be much smaller than the time scale for the density field, leading to a \emph{separation of time scales}.
\todo[inline]{This can probably be written better\dots}

To see this, consider an equation of the form
%
\begin{align} \label{eq: enslave}
    \partial_t \rho \bm p = - D_r \rho \bm p + A(\bm r, t),
\end{align}
%
we may solve it exactly as
%
\begin{align}
    \rho(\bm x, t) \bm p(\bm x,t)
    = e^{- t D_r} \left[ \rho(\bm x, 0) \bm p(\bm x,0) + \int_0^t \dd t' \, e^{- t D_r} A(\bm x,t') \right].
\end{align}
%
We will assume now that $t\gg \tau_R$, so that the first term is suppressed, and change variables to $\tau = t - t'$, which yields
%
\begin{align}
    \rho(\bm r, t) \bm p(\bm x, t)
    & = \int_0^t \dd \tau \, e^{- \tau  D_r} A(\bm x,t - \tau)\\
    & \approx
    \int_0^\infty \dd \tau \, e^{- \tau  D_r} [A(\bm x,t ) + \tau \partial_\tau A(\bm x, t) + ...]
    \approx \frac{1}{D_r} A(\bm x, t),
\end{align}
%
where we in the last step we assume $A$ does not vary much over the time-scale $\tau_R = 1 / D_R$.
This shows that, given our assumptions, we may neglect the time-derivative term in the original equation \autoref{eq: enslave}, and dynamics of $\bm p$ is now complete determined by $A(\bm x, t)$.
We say $\bm p$ is \emph{enslaved} to the slow dynamics of $A$.

Finally, we see that the diffusion term, proportional to $D$ in \autoref{eq: closed polarity}, is fourth order in gradients, and should therefore be neglected.\todo[noinline]{Is this counting right?}
We can now explicitly solve for the polar order,
%
\begin{align}
    \rho p_i = 
    - \frac{1}{2 D_r} \nabla_i  [v(\rho) \rho].
\end{align}
%

A sanity check is to now substitute back in the special case $v(\rho) = v_0$, corresponding to non-interacting ABPs.
The polar order-parameter is then given by
%
\begin{align}
    \rho \bm p = - \frac{v_0}{2 D_R} \bm \nabla \rho,
\end{align}
%
which when substituted back into Fokker-Planck for the density, \autoref{eq: density FP}, gives
%
\begin{align}
    \partial_t \rho &= D_{\mathrm{eff}} \nabla^2 \rho, &
    D_{\mathrm{eff}} = D + \frac{v_0^2}{2 D_R},
\end{align}
%
which is excatly what we found in \autoref{chapter: introduction}.

% Handwritten XII

\textit{{\bf Homework}:
In two dimensions, the vector $\he(\theta) = (\cos\theta, \sin\theta)$ is parametrized by the angle $\theta_t$. Assuming that $\theta_t$ is a Markov process, justify why we can write the joint distribution as $P(\theta_2,t_2;\theta_1,t_1) = P(\theta_2-\theta_1,t_2-t_1|0,0)P(\theta_1,t_1)$ for $t_2 > t_1$.
Deduce that
\begin{equation*}
    \langle \he(\theta_t) \cdot \he(\theta_{t+\tau}) \rangle = \int \rmd\phi \cos\phi \, P(\phi,\tau|0,0) \equiv \langle \cos\phi \rangle_0, 
\end{equation*}
where $P(\theta,t)$ is the distribution of $\theta$ satisfying $P(\theta,0) = \delta(\theta)$.
(Hint: Use the properties of $\theta$ to express the joint distribution $P(\theta_2,t_2;\theta_1,t_1)$)
}



\subsection{MIPS}

If we take the more general case, where the active velocity depends on the density, $|v_a| = v(\rho)$, we may write the Fokker-Planck for the density as a conservation law
%
\begin{align}
    \partial_t \rho(\bm x, t)  = - \bm \nabla \cdot \bm J(\bm x, t),
\end{align}
%
where the current has the form \todo[noinline]{Is the factor $\rho$ right?}
% 
\begin{align}
    J[\rho] = - \rho M[\rho] \bm \mu[\rho].
\end{align}
%
Here, we have introduced the mobility,
%
\begin{align}
    M[\rho] = \frac{v[\rho]^2}{2 D_r},
\end{align}
%
and the chemical potential
%
\begin{align}
    \mu[\rho] = \ln(\rho v[\rho]).
\end{align}
%
This chemical may be written as the derivative of a free energy,
%
\begin{align}
    \mu(\rho) = f'(\rho) \implies
    f(\rho) = \int^\rho \dd x \, \ln (v[x] x).
\end{align}
%
This means that we can apply the methods of equilibrium theory of phase separation.
This is detailed in \autoref{chapter: phase sep}.
If we assume there is a homogenous solution $\rho = \bar \rho$, this will be unstable if $f''(\bar \rho) < 0$.
Inserting this into the free energy we found, we get the criterion
%
\begin{align}
    \frac{v'(\bar \rho)\bar \rho}{v(\bar \rho)} < - 1.
\end{align}
%
This gives the \emph{spinodals}, where the system is unstable for any perturbation.
However, as described in \autoref{chapter: phase sep}, the new phases are described by the \emph{binodals}.
Following the procedure laid out there, we write the equations for chemical and mechanical equilibrium for the two phases with density $\bar \rho_i$ and volume $\V_i$,
%
\begin{align}
    \V_i[f'(\bar \rho_i) - \mu] &= 0, \\
    f(\bar \rho_i) - \mu \rho_i + P &= 0,
\end{align}
%
whose solution gives the volumes and density of each phase given an overall density $\bar \rho$.

This is called motility-induced phase-separation, as the mechanism of phase is not attraction, as in passive systems, but a lowering of the motility in high densities.
The mechanics of phase separation are thus non-equilibrium, however, the resulting large-scale dynamics are still described by an effective equilibrium theory.
This comes as we can always find free energy, no matter the shape of $v(\rho)$.
To break equilibrium also at the large scale, we must have a more general current.
One way to obtain this is to extend the momentum expansion to higher orders.


\section{Top-down approach}

\subsection{Active model B}

Another way to achieve a model with out-of-equilibrium effects is to take the Ginzbur-Landau approach and write down all possible terms allowed by symmetry and conservation.
We begin with the conservation law,
%
\begin{align}
    \partial_t \rho = - \bm \nabla \cdot \bm J,
\end{align}
%
and write the current as
%
\begin{align}
    \bm J = 
    - \rho M(\rho) \bm \nabla 
    \left[
        f'(\rho) - K \nabla^2 \rho + \lambda |\nabla \rho|^2
    \right]
    + \zeta \nabla^2 \rho \bm \nabla \rho.
\end{align}
%
The two first terms may be written in terms of a free energy functional.
The two next cannot, due to the gradient structure.
With only the $\lambda$ term, this is called the active model B.
It was later discovered that another term was possible to include, namely the $\zeta$ term.
With this addition, the model is called active model B+.
More details on this model are given in \autoref{section: active model B top down}.






\section{Scalar active matter: a bottom-up approach}



\subsection{Motility Induced Phase Separation}

\todo[inline]{Should this be included?}

We now use the fact that the spatial dependencies of $v$ and $D$ come from their dependencies in the local density field $\phi$. We thus consider
\begin{equation} \label{eq_local_coeffs}
    v(\bm r) = v(\phi(\bm r)), \qquad D(\bm r) = D(\phi(\bm r)).
\end{equation}
Now Equation~\eqref{eq_closed_phi} for the particle density takes the form of simple diffusion equation $\partial_t\phi = \nabla \cdot [D_{\rm eff}(\phi)\nabla\phi]$ with effective nonlinear diffusivity
\begin{equation} \label{eq_Deff_QS}
    D_{\rm eff}(\phi) = D(\phi) + \frac{v^2(\phi) + \phi v'(\phi) v(\phi)}{d (d-1) D_r} ,
\end{equation}
where as previously primes denote derivatives wrt $\phi$.
As we immediately note, if the effective self propulsion speed $v(\phi)$ of the active particles decreases fast enough when their local density increases, $D_{\rm eff}(\phi)$ may become negative leading to, as we discussed in Sec.~\ref{sec_top_down}, spinodal decomposition of a state with uniform density $\bphi$. Namely, the corresponding condition reads
\begin{equation} \label{eq_cond_MIPS_sp}
    1 + \frac{\bphi v'(\bphi)}{v(\bphi)} < -\frac{d (d-1) D_r D(\bphi)}{v^2(\bphi)} .
\end{equation}
As we discussed previously, this instability will lead to a phase separated state between a dense, slow liquid and a dilute, fast gas. Remarkably, this phase separation phenomenon occurs despite the absence of explicit attractive interactions between the particles, but is induced by a nonequilibrium characteristic which couples the particle local density to their self propulsion force.
Namely, if active particles accumulate in regions of low self propulsion speeds while the latter decreases with the local density, a positive feedback loop sets in leading small density perturbations to be naturally amplified.
This phenomenon is known as \emph{Motility Induced Phase Separation (MIPS)}~\cite{CatesMIPS}.
Neglecting positional diffusion, the condition~\eqref{eq_cond_MIPS_sp} simplifies as $v'(\bphi)/v(\bphi) < -1/\bphi$.

Eq.~\eqref{eq_cond_MIPS_sp} defines the spinodals of MIPS.
As we saw in Sec.~\ref{sec_top_down}, calculating the corresponding binodals then imposes to know which nonlocal (higher order in gradients) terms will appear in the dynamics of $\phi$. 
At the level of Eq.~\eqref{eq_local_coeffs} this can be done by relaxing the constraint that the effective self-propulsion speed and diffusivity depend on locally on the density field. Considering, on the contrary, a short ranged interaction kernel $K(|\bm r|)$ such that
\begin{equation}
    v[\phi] = \tilde{v}\left(\intd{r'} K(|\bm r - \bm r'|)\phi(\bm r') \right),
    \qquad D[\phi] = \tilde D\left(\intd{r'} K(|\bm r - \bm r'|)\phi(\bm r')\right),
\end{equation}
we can expand $\tilde v$ and $\tilde D$ in the gradients of $\phi$ which leads to
\begin{equation} \label{eq_nonloc_coeffs}
    v[\phi] \simeq \tilde v(\phi) + \frac{\ell^2}{2}\tilde v'(\phi) \nabla^2\phi ,
    \qquad
    D[\phi] \simeq \tilde D(\phi) + \frac{\ell^2}{2}\tilde D'(\phi) \nabla^2\phi ,
\end{equation}
where we have used the fact that $K$ is isotropic and normalized to unity, while 
$\ell^2 \equiv \intd{r} K(|\bm r|) |\bm r|^2$.

\paragraph{Absence of positional diffusion}
In the case where $D[\phi]$ can be neglected, Eq.~\eqref{eq_closed_phi} takes the form
\begin{equation} \label{eq_phi_AMB_mapping}
    \partial_t\phi = -\nabla\cdot \bm J, \qquad \bm J = -M_{\rm eff}[\phi]\phi \nabla \ln(v[\phi]\phi),
\end{equation}
where the effective mobility is defined as $M_{\rm eff}[\phi] = v^2[\phi]/(d(d-1)D_r)$. Using the nonlocal expression of the self propulsion speed~\eqref{eq_nonloc_coeffs}, we find that the model maps onto AMB with the effective chemical potential given by
\begin{equation}
    \mu = \ln(\tilde v(\phi) \phi) - \kappa(\phi)\nabla^2 \phi, \qquad \kappa(\phi) = -\frac{\ell^2}{2}\frac{\tilde v'(\phi)}{\tilde v(\phi)}.
\end{equation}
In this case the binodals can thus be calculated from the free energy mapping that we studied in Sec.~\ref{sec_top_down}~\cite{Solon2018}.

\paragraph{The case with positional diffusion}
If $D[\phi]$ cannot be neglected, on the contrary, then one can easily check that the current in Eq.~\eqref{eq_phi_AMB_mapping} includes terms of the form $\sim (\nabla^2\phi)\nabla\phi$ which cannot be written as deriving from a generalized chemical potential. This case corresponds to the AMB+ (active model B+) class which, as shown in Ref.~\cite{Tjhung2018PRX}, shows a qualitatively different physics than that of AMB or equilibrium phase separation.
Indeed, such non-curl free current is responsible for the emergence reversed Ostwald ripening selecting a preferred (finite) phase separated domain length scale.
Although a comprehensive derivation of such term from coarse-graining approaches at outlined in this section is still missing, such microphase separation scenario has recently been shown to be revelant in numerical simulations of microscopic models of ABPs with pairwise repulsion~\cite{Caporusso2020MIPS,Shi2020bubbles}. 




    \chapter{The Non-Reciprocal Cahn-Hilliard model - Lecture 6}
    \label{chap_nrch}
    
\section{Multicomponent phase separation}
Consider for simplicity two conserved density fields $\phi_1$ and $\phi_2$ corresponding to concentrations of two chemical species $1$ and $2$ respectively. Assume that the system at constant volume $V$ fractionates into two phases with volumes $V_1$, $V_2$ and compositions $\phi_1^{(1)}, \phi_2^{(1)}$ in phase 1 and $\phi_1^{(2)},\phi_2^{(2)}$ in the two phases. Our objective is obtain conditions for phase equilibrium. Volume and number conservation constrains these quantities as 
\begin{eqnarray}{cc}
    V_1 +V_2 &=& V  \\
     \phi_1^{(1)} V_1 + \phi_1^{(2)} V_2 &=&  \bar{\phi_1} V \\
     \phi_2^{(1)} V_1 + \phi_2^{(2)} V_2 &=&  \bar{\phi_2} V.
     \label{eq:constraints}
\end{eqnarray}
The free energy of the homogeneous system is $\mathcal{F}(N_1,N_2,V,T)$. For constant temperature $T$ or any other external control parameter like Ph and noting that the free energy is extensive with respect to volume, the total free energy receives contribution from the two phases 
\begin{eqnarray}
\mathcal{F} = V_1 f(\phi_1^{(1)},\phi_2^{(1)}) + V_2 f(\phi_1^{(2)},\phi_2^{(2)}).
\label{eq:TotalFreeEnergy}
\end{eqnarray}
The unconstrained free energy $\bar{\mathcal{F}}$ is obtained by introducing Lagrange multipliers $P$ conjugate to the total volume and $\mu_1,\mu_2$ conjugate to the two conserved fields - 
\begin{eqnarray}
\bar{\mathcal{F}} = \mathcal{F} +P(V_1 + V_2) - \mu_1 (V_1 \phi_1^1 + V_2 \phi_2^1) - \mu_2 (V_1 \phi_2^1 + V_2 \phi_2^2).
\end{eqnarray}
Minimizing $\bar{\mathcal{F}}$ with respect to the six unknown quantities, we find three more conditions for phase equilibrium
\begin{eqnarray}
\mu_1 &=& \frac{\partial f(\phi_1^1,\phi_2^1)}{\partial \phi_1^1} = \frac{\partial f(\phi_1^2,\phi_2^2)}{\partial \phi_1^2}. \\
\mu_2 &=& \frac{\partial f(\phi_1^1,\phi_2^1)}{\partial \phi_1^1} = \frac{\partial f(\phi_1^2,\phi_2^2)}{\partial \phi_1^2}. \\
P &=& -f + \mu_1 \phi_1^1 + \mu_2 \phi_2^1  = -f + \mu_1 \phi_1^2 + \mu_2 \phi_2^2. 
\label{eq:Equilibrium}
\end{eqnarray}
Eqs. \eqref{eq:constraints} and \eqref{eq:Equilibrium} can be solved to obtain the Binodals. These equations represent a generalisation of the Maxwell construction for N=1. Similar equations can be written and solved for N components and $n$ phases. Gibbs phase rule restricts the number $n$. Going back to $N=2$, Figure \ref{fig:BinaryPD} shows the variety of phase diagrams and critical points that are possible. The relaxational dynamics of the fields $\phi_{1,2}$ are given by a generalisation of the Cahn-Hilliard dynamics. 
\begin{eqnarray}
\partial_t \phi_i(\bm{r},t) + \bm{\nabla} \cdot \bm{J}_i &=& 0 \\
 \bm{j}_i &=& - \bm{\nabla} \mu_{i}^\mathrm{eq}   +  \bm{\zeta}_i,
\end{eqnarray}
We choose a general form for $f  = f_1(\phi_1) + f_2(\phi_2) + f_{I}(\phi_1,\phi_2)$. A simple choice, often made for the Flory-Huggins system is $f_I = \chi \phi_1 \phi_2$, contributes , linear terms to $\mu_{1,2}$. For $\chi <0$, the densities modulations co-locate simulating attractive microscopic interactions while $\chi>0$ the densities anti-co-locate simulating repulsive microscopic interactions.

\section{Non-reciprocal Cahn-Hilliard Dynamics}
We will now introduce a new kind of activity in the system - non-reciprocity. We add terms to the chemical potential which cannot be derived from a free energy.  

\subsection{Non-reciprocal interactions in a binary mixture}
For a binary mixture, the free energy can be written as
\begin{eqnarray}
F &=&  \int \mathrm{d}\bm{r} \bigg\{ \sum_{i=1}^{2} (\phi_i-c_{i,1})^2(\phi_i-c_{i,2})^2  \nonumber \\ 
&& + \chi \phi_1 \phi_2 + \chi' \phi_1^2 \phi_2^2  
+ 
% \revise{
    \frac{\kappa}{2}
    % }  
    \sum_{i=1}^{2} |\bm{\nabla} \phi_i|^2  \bigg\},
\label{FreeEnergy2}
\end{eqnarray}
which results in the equilibrium chemical potentials
\begin{eqnarray}
\mu_1^\mathrm{eq} &=& 
% \revise{
    2
    % } 
(\phi_1 - c_{1,1})(\phi_1 - c_{1,2})(2\phi_1 - c_{1,1}- c_{1,2}) + \chi \phi_2 \nonumber \\
&& + 2 \chi' \phi_1 \phi_2^2, \nonumber \\
\mu_2^\mathrm{eq} &=& 
% \revise{
    2
    % } 
(\phi_1 - c_{2,1})(\phi_1 - c_{2,2})(2\phi_1 - c_{2,1}- c_{2,2})  + \chi \phi_1 \nonumber \\
&& + 2 \chi' \phi_2 \phi_1^2.
\label{ChemPot}
\end{eqnarray}
We note that, at equilibrium, the sign and strength of the interaction between the two components is governed by $\chi$. If $\chi>0$, the interaction between the two species is repulsive (their overlap increases the free energy of the system), whereas if $\chi<0$, the interaction is attractive (overlap decreases the free energy of the system). For two components, the activity matrix $\alpha_{ij}$ is simply given by $\alpha_{11}=\alpha_{22}=0$, and $\alpha_{12}=-\alpha_{21}=\alpha$, and there is a single scalar parameter $\alpha$ representing the non-reciprocal activity. The non-equilibrium chemical potentials become 
\begin{eqnarray}
 \mu^\mathrm{neq}_1 = \mu_1^\mathrm{eq} + \alpha \phi_2, \nonumber \\
 \mu^\mathrm{neq}_2 = \mu_2^\mathrm{eq} - \alpha \phi_1.
 \label{ChemPot2}
\end{eqnarray}

Considering the form of the equilibrium chemical potentials (\ref{ChemPot}), it becomes clear that the activity $\alpha$ acts to modify the equilibrium interaction parameter $\chi$ within the non-equilibrium chemical potential, so that we find a term $(\chi + \alpha) \phi_2$ in $\mu^\mathrm{neq}_1$, and a term $(\chi - \alpha) \phi_1$ in $\mu^\mathrm{neq}_2$. A direct numerical simulation of the NRCH equations in \eqref{eq:nrch} starting from random initial conditions shows that the time translation invariance of the bulk-phase separated state is broken when $|\alpha|>|\chi|$. 

\subsection{Stability analysis of the mixed binary system}
In order to obtain more analytical insight into the nature of the instabilities in the system, we linearize the dynamics of the binary NRCH model around a homogeneous state $(\avOne, \avTwo)$ to obtain
\begin{eqnarray}
\begin{pmatrix}
	\dot{\phi_1}(\bm{q}) \\
	\dot{\phi_2}(\bm{q})
\end{pmatrix}
&=& \begin{pmatrix} \mathcal{D}_{11} & \mathcal{D}_{12} \\ \mathcal{D}_{21} & \mathcal{D}_{22} \end{pmatrix}  \begin{pmatrix}
	{\phi_1}(\bm{q}) \\
	{\phi_2}(\bm{q})
\end{pmatrix}, \label{eq:matrixeq}
\end{eqnarray}
where the components of the matrix $\mathcal{D}$ are given by
\begin{eqnarray}
\mathcal{D}_{11} &=& -q^2 [ 2 (\avOne - c_{1,1})^2 + 8 (\avOne - c_{1,1}) (\avOne - c_{1,2})  \nonumber \\ && + 2 (\avOne - c_{1,2})^2 + 2 \avTwo^2 \chi'], \nonumber \\
\mathcal{D}_{12} &=& -q^2 [ (\chi+\alpha) +  4 \avOne \avTwo \chi' ],  \nonumber \\
\mathcal{D}_{21} &=& -q^2 [ (\chi-\alpha) +  4 \avOne \avTwo \chi' ],  \nonumber \\
\mathcal{D}_{22} &=& -q^2 [ 2 (\avTwo - c_{2,1})^2 + 8 (\avTwo - c_{2,1}) (\avTwo - c_{2,2})  \nonumber \\ && + 2 (\avTwo - c_{2,2})^2 + 2 \avTwo^2 \chi'], \nonumber
\end{eqnarray}
Recall that the stability of the homogeneous state defined by uniform $\bar{\phi}_i$ is determined by the signs of the eigenvalues of the matrix \eqref{eq:matrixeq}. In the absence of activity $\alpha=0$, the matrix $\mathcal{D}_{ij}$ is symmetric and thus only admits real eigenvalues. When the non-reciprocal activity is turned on, however, $\mathcal{D}_{ij}$ is no longer symmetric and its eigenvalues may become complex, signaling the possibility of oscillations in the NRCH model.

Indeed, a non-oscillatory instability will take place when one of the eigenvalues $\lambda_{1,2}$ is real and positive, whereas an oscillatory instability is expected when $\lambda_{1,2}$ are a complex conjugate pair with positive real part. To study the phase diagrams of the system, we define $\mathcal{C}_r$ as the region of the parameter space where either $\mbox{Re}(\lambda_1)>0$ or $\mbox{Re}(\lambda_2)>0$, and $\mathcal{C}_i$ as the region where $\mbox{Im}(\lambda) \neq 0$. A non-oscillatory instability will occur in regions of $\mathcal{C}_r$ that do not intersect with $\mathcal{C}_i$, whereas the instability will be oscillatory at the intersection between $\mathcal{C}_r$ and $\mathcal{C}_i$.

\subsection{Exceptional points}
Consider the case for which $\chi' = 0$.  

\subsection{Oscillatory instability in the composition plane}

\subsection{Some general points about the travelling wave state}

\subsection{Comparison with non-conserving dynamics and the minimal oscillator}
It is useful to compare the equations of NRCH with the non-reciprocal model A 
\begin{eqnarray}
    \partial_t \phi_1 &=& - \mu_1 + \alpha \phi_2 + K \nabla^2 \phi_1\\
    \partial_t \phi_2 &=& - \mu_2 - \alpha \phi_1 + K \nabla^2 \phi_2.
    \label{eq:NonReciprocalModelA}
\end{eqnarray}
Let us first look at the dynamical system described by $\dot{x_i} = - \mu_i$. 




\subsection{Stability of the plane waves}
In this subsection we will choose a different form for the free energy $f_I = \phi_1^2 \phi_2^2 /2 $. For this choice we can write down an exact form for the traveling waves and check their stability to linear perturbations. 
\begin{eqnarray}
\psi(\bm{r}, \bm{q},t) = R \exp^{i( \bm{q} \cdot \bm{r} - \omega t)}.
\label{eq:planeWave}
\end{eqnarray}
is substituted in \eqref{variantNRCH} to obtain expressions for the amplitude $R(q)$ and the dispersion relation $\omega(q)$
\begin{eqnarray}
R &=& c \sqrt{1-\frac{q^2}{q_0^2}}, \; \forall q < q_0, \nonumber \\
\omega(q) &=& {\Gamma} q^2 \left[- \alpha_0 + \alpha_1 c^2\left( 1- \frac{q^2}{q_0^2} \right) \right], 
\label{eq:dispersion}
\end{eqnarray}



    \chapter{Phoretic active matter - Lectures 7\&8}
    \label{chap_phoretic}
    Phoresis comes from Greek and means to be carried.
There are several types of phoresis, but common for all is that there is a gradient in some fields that carries particles around.
Some kinds of phoresis, and the corresponding gradients are
\begin{itemize}
    \item Diffusiophoresis, $\bm \nabla C$,
    \item Electrophoresis, $\bm \nabla \varphi = \bm E$,
    \item Thermophoresis, $\bm \nabla T$.
\end{itemize}
%
This gradient gives rise to a \emph{slip velocity} $\bm v_s$ along the surface of particles and allows for motion.


\section{Diffusiophoresis}

We will now consider a particle of radius $R$ suspended in a liquid, surrounded by a chemical density $c$.
In this section, we will show that, given some criteria for the interaction of the particle with the chemical, this gives rise to a slip-velocity $\bm v_s$ of the fluid around the particle, which is given by
%
\begin{align}
    \bm v_s  = \mu \bm \nabla_\parallel c_{\mathrm{out}},
\end{align}
%
where $c_{\mathrm{out}}$ is the gradient surronding sphere, outside the slip region(?).
\todo[inline]{figure}

The slip region is a thin layer of width $\sigma \ll R$ around the sphere, where the chemical and fluid is interacting with the surface of the sphere.
\todo[inline]{figure}

We assume the particle interacts with the chemical through a potential of the form
%
\begin{align}
    W(\bm x) = 
    \begin{cases}
        0 , & z \geq \sigma \\
        W(z), & z < 0.
    \end{cases}
\end{align}
%
This potential must diverge as we approach the sphere, $W(z\rightarrow 0) = \infty$, as we model the particle as a hard sphere.

We assume the solute follow
%
\begin{align}
    \partial_t c &= \bm \cdot \bm J,\\
    \bm J &= - \bm \nabla c + \beta D C \bm F + c \bm v,
\end{align}
%
and the fluid is Stoksean,
%
\begin{align}
    - \eta \nabla^2 \bm v &= - \bm \nabla p + \bm f, \quad\quad \bm f = - c \bm \nabla W,\\
    \bm \nabla \cdot \bm v &= 0.
\end{align}
%
We will assume the chemical relaxes fast, so $\partial_t c = 0$.
Then we consider the flow in the $z$-direction, perpendicular to surface of the sphere, and the parallel flow seperatly.
This leaves the following equations,
%
\begin{align}
    D \nabla^2 c &= \beta D \bm \nabla \cdot (c \bm \nabla W) - \bm v \cdot \bm \nabla c,\\
    - \eta \nabla^2 v_z &= - \partial_z p - c \partial_z W, \\
    - \eta \nabla^2 \bm v_\parallel & = - \bm \nabla_\parallel p.
\end{align}
%


    \chapter{Polar active matter  - Lectures 9\&10}
    \label{chap_polar}
    
\section{Introduction}

So far, we have mostly focused on `scalar' active matter whose continuous description
is generally expressed in terms of a scalar density field.
In this chapter, on the contrary, we will address cases where active systems present large scale orientational order.
The goal of these lectures is to present a minimalist approach to this problem, which consists of building simple models 
including relevant physical ingredients so as to capture physical features common to a broad variety of active matter systems.
We will start by adopting a microscopic point of view, i.e. by constructing a particle model for collective motion.
Then, as we will be interested in collective effects involving many individuals, 
we will coarse grain our model using kinetic theory in order to obtain its hydrodynamic description
which will allow us to theoretically characterize the large scale physics at play.

\section{A minimal approach to collective motion} 

Collective motion can be described as the ability of some self propelled agents to move coherently at the level of many individuals without the need of a leader or external influence.
Such definition encompasses various examples found in nature, 
most obvious are human crowds, bird flocks, or insect swarms.
At the micron scale one can moreover mention bacterial colonies, or cellular migration which is involved in various biological phenomena such as morphogenesis, wound healing, or cancer development.
Cells are actually put into motion thanks to an elaborate machinery involving various types of molecular motors, so that collective motion is also found at sub-cellular scales (cf~\autoref{chap_intro}).

A natural question for a physicist is then whether all the above systems, even though their size spans many orders of magnitude, share common {\it universal} properties which could be captured by a unified theoretical framework.
To answer this question, we will adopt a minimalist approach that will discard the details of each system, but only include the physically relevant ingredients necessary for collective motion to arise.

\paragraph*{How physicists see birds}
In the past thirty years, flocks of birds became one of the paradigmatic examples of biological active matter. A combination of tracking experimental data and theoretical approaches based on statistical physics allowed to unveil and quantify many properties of their collective behavior. 
When moving along a fixed direction, 
the large cohesive structures formed by birds can be seen as breaking the global symmetry of rotation.
In analogy with liquid crystal physics, the velocities $\{\bm v_i\}_{i = 1,...N}$ of birds 
thus exhibit a certain amount of order,
which can be quantified computing the average polarization
\begin{equation}
\Pi(t) = \left | \frac{1}{N} \sum_{i}^N \frac{\bm v_i(t)}{|\bm v_i(t)|} \right |
\end{equation}
which is $1$ for a perfectly aligned system, and it is $0$ for a disordered one.
The average value for flocks is around $\langle \Pi \rangle \sim 0.96$, reflecting a high degree of polar order. However, this feature is not enough to ensure a self-organized collective behavior of the group. Birds move without a leader, interact with only a few neighbors, but their motion is correlated on length scales much larger than the typical inter-element distance. This has been understood looking at the individual velocity fluctuation around the mean velocity of the group,
$$
\delta \bm v_i (t) = \bm v_i(t) - \frac{1}{N} \sum_k \bm v_k(t)
$$
and at its correlation function,
\begin{equation}
 C(r) = \frac{1}{C_0} \left\langle \frac{\sum_{ij} \delta \bm v_i(t) \cdot \delta \bm v_j(t) \delta(r-r_{ij}(t))}{\sum_{ij} \delta(r-r_{ij}(t))} \right\rangle_t
\end{equation}
that measures how much the fluctuation in the moving direction of bird $i$ influences that of bird $j$, when they are separated by a distance $r_{ij}$. $C(r)$ is a static quantity, thus averaged over all the trajectories. 
\autoref{scalefree} shows this function obtained from experimental data from real flocks: the correlation is positive for short distances, then it decays until crossing the zero at a point $\ell_c$, and finally it becomes negative. This last feature is due to the definition of velocity fluctuation, but this trend highlights that there are only two large correlated domains in the flock and they are of size $\approx \ell_c$. Adopting $\ell_c$ as a measure of the correlation length, \autoref{scalefree}(b) shows that it scales linearly with the flock's size: $\ell_c \sim L$.
This reflects a scale-free behavior similar to the ones we have encountered in~\autoref{chap_intro} for the study of equilibrium magnetized systems.
It can indeed be shown that $\ell_c \sim L$ implies power law velocity correlations,
$$
C(r) = r^{-\gamma} g \left( \frac{r}{L}\right)
$$
typical of strongly correlated systems near-criticality ($T=T_c$) or in a continuous symmetry-broken phase ($T\to 0$, $O(n)$).
As these systems, assemblies of collectively moving birds undergo system spanning correlations, thus ensuring high sensitivity to external perturbations.

\begin{figure}[t]
\begin{center}
	\includegraphics[width=.75\textwidth]{Figures/scalefree.pdf}
	\caption{(a) The correlation function $C(r)$ is the average inner product of the velocity fluctuations of pairs of birds at mutual distance $r$. This correlation function therefore measures to which extent the orientations of the velocity fluctuations are correlated. The function changes sign at $r=\ell_c$, which gives a good estimate of the average size of the correlated domains.(b) The orientation correlation length $\ell_c$ is plotted as a function of the linear size $L$ of the flock. Each point corresponds to a specific flocking event and it is an average over several instants of time in that event. The figure was reproduced from Ref.~\cite{cavagna2010scale}.
	}
	\label{scalefree}
	\end{center}
\end{figure}

Our purpose here will be to develop minimal models able to explain these features found not only in birds, but also in a variety of other systems.
Therefore, we will build our model from the example of birds for illustration purposes, but we won't consider the details of their ethology or morphology.
From far away, a flying bird is nothing but a polar object, meaning that is moves along a preferred direction set by its body axis.
Such feature is common in the systems we are interested in, so that in the following we will adopt an abstract representation
for which the bird is a simple arrow pointing along its direction of motion.

\paragraph*{Interactions with the environment}
Interactions between a biological systems and its environment are generally very complex and their quantitative description is here out of reach. 
Rather than solving the Navier-Stokes equations
to characterize the interaction of the bird with the surrounding air, 
we will here neglect inertia and the aerodynamics by assuming that the animal self-propels with a constant speed.
Such an approximation is arguably not so good for birds, but it turns out to be well verified for microscopic objects such as bacteria moving at low Reynolds numbers, or for any two dimensional active motion in the presence of a substrate.
Although setting the bird self propulsion speed to a constant leads to neglecting speed fluctuations, fluctuations in the direction of motion of the bird are generally still possible such that our model will retain orientational noise. 
 
 \paragraph*{Interactions with neighbors}
 Above, we have described the motion of a single bird, but below we will describe collective behaviors which implies to describe interactions between individuals.
 A prerequisite for flocking is that the birds seek to move together. 
 In the context of polar objects moving at constant speed, this effect is easily achieved via social rules leading to local alignment of velocities.
 Other types of interactions can of course be implemented. 
 For example, animals generally try to avoid colliding with each other, which can be described as a kind of short-ranged repulsion. Moreover, swarms or flocks usually show a certain degree of cohesion, which also suggests the presence of an effective attraction at long distances. 
 Another type of effective interactions may come from the medium our agents are moving into. An example of such interactions which we have studied in~\autoref{chap_phoretic} are phoretic interactions mediated by production/consumption and motility response to a chemical concentration. 
 Another prominent example are of course hydrodynamic interactions which play a central role in biological systems (cf~\autoref{chap_hydro}).
 Although these effects need to be considered when studying specific systems in details, they will enrich\footnote{Understand `complicate'.} the physics and may hide some of the universal features we are looking for. 
 In this lecture, we will thus restrict to the simplest setting where the agents of the model only interact via pairwise velocity alignment.


\section{The Vicsek model}

\subsection{Definition of the Vicsek class}

We now have all the necessary ingredients for a minimal model of collective motion.
The most famous one is arguably the Vicsek model~\cite{chate2020dry}, as it was first formulated by T. Vicsek and collaborators.
In the Vicsek setting, point-like particles move in two dimensions with constant speed $v_0$ and align their velocities with their neighbors in the presence of noise.
These rules mathematically translate in two dimensions into the following discrete time dynamics for the $i^{\rm th}$ particle's position $\bm r_i$ and velocity orientation $\theta_i$
\begin{subequations}
\label{eq_VM}
\begin{align}
\label{eq_VM_r}
\bm r_i^{t + \Delta t} & = \bm r_i^{t} + v_0 \Delta t \hat{\bm e}(\theta_i^{t + \Delta t}) \, , \\
\label{eq_VM_theta}
\theta_i^{t + \Delta t} & = {\rm Arg}\left[ \langle \hat{\bm e}(\theta_j^{t}) \rangle_{j \in \Omega_i} \right] + \eta \xi_i^t , 
\end{align}
\end{subequations} 
where $\hat{\bm e}(\theta) = (\cos(\theta),\sin(\theta))$ is the unit vector with orientation $\theta$, 
the average $\langle\cdot\rangle_{j \in \Omega_i}$ is done over the disk of radius $r_0$ centered in $i$: $\Omega_i \equiv \{ j ; \|\bm r_i  - \bm r_j\| \le r_0 \}$ 
(it therefore includes $i$ itself), and ${\rm Arg}[\bm v]$ returns the orientation of the vector $\bm v$.
The second term in the rhs of Eq.\eqref{eq_VM_theta} accounts for the angular fluctuations experienced by the particles, 
whose strength are set by the parameter $\eta$. 
$\xi_i^t$ is therefore a white noise with zero mean and unit variance:
$\langle \xi_i^t \xi_j^{t'} \rangle = \delta_{ij} \delta_{t t'}$ --hereafter angular brackets without indices are meant as averages over the noise.
In numerical simulations, the distribution of $\xi$ is usually taken uniform in $(-\pi;\pi]$ for simplicity, 
but other choices are possible (for instance Gaussian) and they do not qualitatively modify the dynamics 
so long as clockwise and counter-clockwise fluctuations remain equiprobable.
Similarly, different implementations of the Vicsek model have been proposed (e.g.\ considering continuous time dynamics, or with different forms of the noise etc...),
but those lead to large-scale collective features qualitatively similar to those of Eqs.~\eqref{eq_VM}.
For more details on the different numerical implementations of the Vicsek dynamics, see Ref.\cite{chate2020dry} and references therein.
Here, for practical purposes we'll consider the case where Eqs.~\eqref{eq_VM} are simulated in a square domain of linear size $L$ with periodic boundary conditions on all sides.

As for the Ising or $XY$ models for ferromagnets, the importance of the role of the Vicsek model in active matter studies stems from its simplicity and genericity.
Indeed, numerical and theoretical studies of the model allowed to uncover general physical principles which can be applied to a broad range of active matter systems.
In this context, Eqs.~\eqref{eq_VM} thus define a {\it Vicsek class} (or polar class), whose members all satisfy the following requirements:
\begin{itemize}
\item The alignment interactions between the particles are {\it local} and {\it polar}. 
Particles moreover do not point towards a preferred direction so that the dynamics is {\it isotropic}.  
\item {\it The particles are advected by their polarities.} 
While the polarity and velocity of Vicsek particles are identical, this constraint can be relaxed to some extent if the two remain coupled.
\item As a consequence of the previous point, the Vicsek class is inherently out of equilibrium. 
In particular, the dynamics does not conserve momemtum, so that it is not invariant by change of inertial frame and {\it Galilean symmetry is broken}.
\item The dynamics~\eqref{eq_VM} nevertheless {\it conserves the total particle number $N$} (there are no particle creation or annihilation).
\end{itemize}

\subsection{The phase diagram}

\label{sec_pd}

Rescaling space and time, we set $\Delta t$ and $r_0$ to unity. 
Therefore, the Vicsek model has three control parameters: $v_0$, $\eta$, and the mean particle density $\bar{\rho} \equiv N / L^2$.
Considering $v_0$ in a `reasonable' range --meaning not too small so that activity appears on scales accessible via simulations, 
or not too large so that the neighborhood of each particle is not completely randomized at each simulation step-- its specific value does not qualitatively affect the dynamics\footnote{Typically $v_0$ is taken between 0.1 and 1.}.
Hence, the dynamical behaviors of Vicsek particles can be explored tuning the remaining two parameters: the average particle density $\bar{\rho}$ and the strength of the noise $\eta$.

The phase diagram of the Vicsek model in the ($\bar{\rho}$,$\eta$) plane is shown in Fig.~\ref{figVM}(a). It includes three phases:
\begin{itemize}
\item At low particle densities and high noises, fluctuations dominate over alignment interactions and the system is thus globally disordered. 
Namely, the polarity and density correlations are exponential with a finite associated correlation length $\ell_c$. 
The system can thus be split into uncorrelated domains of linear size $> \ell_c$ so that from the law of large numbers the global polar order vanishes over large scales as
\begin{equation}
\Pi \equiv \langle \|\langle \hat{\bm e}(\theta_i^{t}) \rangle_{i=1\ldots N}\|\rangle_t \underset{L \gg \ell_c}{\sim} N^{-1/2} \sim L^{-1} .
\end{equation} 
\item At high densities and low noises, the mean number of neighbors of each particles is large enough so that the system forms 
a homogeneous ordered state in which particles move collectively along a randomly selected direction. 
This symmetry broken phase exhibits characteristic features such as scale free (power law) density and order correlations, 
as well as long-range orientational (polar) order.
Namely, as shown in Fig.~\ref{figVM}(c) the system's global polarization is found to decay with system size to a nonzero value $\Pi_\infty$, with power law finite-size corrections:
\begin{equation}
\label{eq_LRO}
\Pi \sim \Pi_{\infty} + A L^{-\omega} .
\end{equation} 
Here $\Pi_\infty$ and $A$ depend on the parameters $\bar{\rho}$ and $\eta$ and the details of the microscopic dynamics, 
while the exponent $\omega \simeq 2/3$~\cite{chate2020dry} is universal so that its value is the same throughout the homogeneous ordered phase and for all two dimensional systems in the Vicsek class (similarly to the critical exponents that we have used in~\autoref{chap_intro} to describe the transition to order in the Ising model).

Another peculiar feature of the ordered phase can be observed by measuring the statistics of the number of particle $n(t) \equiv \int_{\cal V}{\rm d}\bm r \, \sum_i \delta(\bm r - \bm r_i^t)$ into a sub-domain of volume $\cal V$.
Namely, measuring the mean and variance of $n$ in domains of increasing sizes one finds that
\begin{equation}
\langle \Delta n^2 \rangle \sim \langle n \rangle^\phi ,
\end{equation}
with $\phi \simeq 1.6$ (see Fig.~\ref{figVM}(c))~\cite{chate2020dry}. 
In a system with `normal' density fluctuations, the law of large number would impose that $\phi = 1$. 
Here, however, $\phi > 1$, i.e.\ the variance of $n$ grows faster than its mean, 
so that density fluctuations in the homogeneous ordered phase are deemed as `anomalous', or `giant'\footnote{Note that these giant density fluctuations are not related to any clustering phenomenon (e.g.\ like the motility induced phase separation discussed in~\autoref{chap_scalar}),
but instead due to the infinite correlation length of density fluctuations in an overall spatially homogeneous system.}.
\item Finally, contrary to magnetic systems at equilibrium (i.e.\ described by the Ising or $XY$ models) the transition to collective motion is not second order, but is best described as a phase separation scenario where elongated dense ordered domains --which are usually referred to as \textit{bands}, see Fig.~\ref{figVM}(b)-- coexist with a dilute disordered gas.
Because of the coupling between density and order, these bands travel in the direction orthogonal to their axis.
Contrary to usual phase separation between two macroscopic domains, bands here are quantized, i.e. they have a finite width $\lambda_{\rm b}$ selected by the parameters of the model.
In this case, conservation of volume and mass thus imply that 
\begin{equation*}
    L_\| = n_{\rm b}\lambda_{\rm b} + L_{g}, \qquad
    N = \bar\rho L_\| = n_{\rm b}\langle \rho \rangle_{\rm b} + \rho_g L_g,
\end{equation*}
where we have assumed that the dynamics can be reduced to a quasi-one-dimensional problem. $L_\|$ is then the system size along the direction of order, $n_{\rm b}$ is the number of bands and $\langle \rho \rangle_{\rm b}$ is their mean density, while $\rho_g$ and $L_g$ are respectively the density and volume of the gas phase.
We then find that the number of bands is given by
\begin{equation}
    n_{\rm b} = \frac{L_\| (\bar\rho - \rho_g)}{\lambda_{\rm b}(\langle \rho \rangle_{\rm b} - \rho_g)}.
\end{equation}
In contrast with the lever rule that we introduced in~\autoref{chap_scalar}, the volume fraction of ordered liquid here is set by the number of quanta of liquid $n_{\rm b}$, which grows linearly with the mean particle density $\bar\rho$ and system size $L_\|$.
Such scenario is usually named {\it micro-phase separation}, and here results in phase-separated configurations that take the form of regular arrangements of traveling bands.
\end{itemize}

%%%%%%%%%%%%%%%%%%%%%%%%%%%%
\begin{figure}[t!]
	\includegraphics[width=\textwidth]{Figures/figVM.pdf}
	\caption{(a) The stylized phase diagram of the Vicsek model in the density-noise plane showing the three phases described in the text: disordered, homogeneously ordered, and the phase separated phase separating the two.
	(b) A snapshot of a regular smectic band configuration found in the phase-separated phase, the white arrow indicates the mean order direction and the color codes the local particle density. 
	Parameters are $\bar{\rho} = 1/2$, $\eta = 0.2$ and $v_0 = 0.5$.
	(c, top) Global polarization in the homogeneous ordered phase (in double-log representation) decreasing slower than a power law, the solid black line is a fit of the data by Eq.~\eqref{eq_LRO} giving $\Pi_\infty = 0.8685(2)$ and $\omega \simeq 0.64$. The inset shows the power law scaling of $\Pi - \Pi_\infty$ with system size.
	(c, bottom) $\langle \Delta n^2 \rangle/\langle n \rangle$ (see Sec.~\ref{sec_pd} for a definition) as a function of $\langle n \rangle$ showing anomalous density fluctuations in the ordered phase (filled circles) and normal fluctuations in the disordered phase (hollow squares). Different colors indicate different system sizes.
	Parameters for (c) are $\bar{\rho} = 2$, $v_0 = 0.5$, $\eta = 0.2$(ordered) and 0.6(disordered).
	All figures are reproduced from~\cite{DADAM_LesHouches}. 
	}
	\label{figVM}
\end{figure}
%%%%%%%%%%%%%%%%%%%%%%%%%%%%

\subsection{The hydrodynamic description of the Vicsek class}

It can be shown that a microscopic dynamics similar to that defined by Eqs.~\eqref{eq_VM} can be coarse-grained into hydrodynamic equations for the relevant physical fields which are the particle density $\rho$ and momentum $\bm w$ (or polarity):
\begin{align*}
\rho(\bm r,t) \equiv \left\langle \sum_i \delta(\bm r - \bm r_i^t) \right\rangle , \qquad 
\bm w(\bm r,t) = \rho(\bm r,t) \bm v(\bm r,t) \equiv \left\langle \sum_i \hat{\bm e}(\theta_i^t)\delta(\bm r - \bm r_i^t) \right\rangle .
\end{align*}
Using the Boltzmann-Ginzburg-Landau (BGL) approach detailed in Sec.~3 of Ref.~\cite{DADAM_LesHouches}, 
one can derive the set of partial differential equations governing the dynamics of $\rho$ and $\bm w$, known as the Toner Tu equations, which read
\begin{subequations}
\label{eq_TT}
\begin{align}
\label{eq_TT_rho}
& \partial_t \rho + \nabla \cdot \bm w = 0 , \\
\label{eq_TT_w}
& \partial_t \bm w + \lambda_1 (\bm w \cdot \nabla)\bm w + \lambda_2 (\nabla \cdot \bm w) \bm w + \lambda_3 \nabla |\bm w|^2 = 
-\frac{1}{2}\nabla \rho + \left( \mu(\rho) - \xi |\bm w|^2 \right) \bm w + \nu \Delta \bm w .
\end{align}
\end{subequations}
Thanks to the coarse-graining procedure, all coefficients of Eqs.~\eqref{eq_TT} can be expressed as functions of the microscopic dynamics parameters, i.e.\ density, noise and velocity.
This way it is possible to check from linear stability analysis and numerical simulations that Eqs.~\eqref{eq_TT} reproduce qualitatively the 
Vicsek model phase diagram of Fig.~\ref{figVM}(a) with the transition to collective motion occurring through a phase separated phase.
 
Although Eqs.~\eqref{eq_TT} were obtained starting from specific dynamical rules for the particles position and orientations, 
they contain almost all possible terms up to order 3 in fields and gradients that satisfy the constraints set by the Vicsek class,
so that their structure turns out to be quite generic and would not qualitatively change with the details of the dynamics.
In fact, Eqs.~\eqref{eq_TT} were first written by considering all relevant terms which satisfy the symmetries and conservation rules 
set by Eqs.~\eqref{eq_VM}:
\begin{itemize}
\item As a consequence of the particle number conservation, $\rho$ is a conserved field and Eq.~\eqref{eq_TT_rho} takes the form of a continuity equation with a mass flux $\bm w$, reflecting the fact that density is advected by order.
\item In Eq.~\eqref{eq_TT_w} the symmetry breaking associated with collective motion is accounted for by the Landau cubic term on the r.h.s. which, for $\mu(\bar{\rho}) > 0$, leads to the homogeneous ordered solution $\bm w = \sqrt{\mu(\bar{\rho}) / \xi} \hat{\bm e}_\|$ and $\hat{\bm e}_\|$ a unit vector pointing along an arbitrary direction.
\item The nonlinear terms on the l.h.s.\ of Eq.~\eqref{eq_TT_w} are moreover all allowed with arbitrary coefficients due to the absence 
of momentum conservation\footnote{In a compressible system, the advective term conserving momentum takes the form $\partial_t \bm w + \nabla \cdot (\bm w \bm w / \rho)$.}. 
Indeed, it is easy to check that they generally beak Galilean symmetry which would require the equations to be invariant under the transformation
\begin{equation*}
\bm w \to \bm w + \rho \bm u , \qquad \bm r \to \bm r - \bm u t ,
\end{equation*}
with $\bm u$ a constant velocity.
Moreover, because of these terms Eqs.~\eqref{eq_TT} generally cannot be written from minimizing of a free energy-type functional,
which highlights again the nonequilibrium nature of the underlying dynamics.
\end{itemize}
%Finally, we shall specify that the `simple' form of the pressure term ($\sim \nabla\rho$) in Eq.~\eqref{eq_TT_w} is a consequence of the point-wise nature of the particles, and considering short-range repulsion would lead to more a complicated expression. \\


In the remainder of these notes, we are going to focus on the derivation of large scale and long time universal properties of the fluctuating homogeneous ordered solution of Eqs.~\eqref{eq_TT}.
The approach presented below follows the seminal work by Toner and Tu~\cite{toner1995long,toner2012reanalysis}.
For readers interested in more details about the topic, we recommend the more recent treatment of Ref.~\cite{chate2024dynamic}.

\section{The transition to collective motion: linear stability analysis}

As Eqs.~\eqref{eq_TT} contain many terms which will not qualitatively affect the results derived below, 
for the sake of pedagogy we will study here simplified equations containing all relevant physical ingredients.
Namely, let us consider
\begin{subequations}
\label{eq_TT_new}
\begin{align}
\label{eq_TT_rho_new}
& \partial_t \rho + \nabla \cdot \bm w = 0 , \\
\label{eq_TT_w_new}
& \partial_t \bm w + \lambda (\bm w \cdot \nabla)\bm w = 
-\frac{1}{2}\nabla \rho + \left( \mu(\rho) - \xi |\bm w|^2 \right) \bm w + \nu \Delta \bm w ,
\end{align}
\end{subequations}
with now only one advection term in~\eqref{eq_TT_w_new}, but still with the coefficient $\lambda$ kept arbitrary.
Moreover, we will restrict most of the calculations to two dimensions.
As in the classical Ginzburg-Landau model, a change of sign of the coefficient $\mu(\rho)$ defines a phase transition where the system becomes macroscopically ordered.
Namely, Eqs.~\eqref{eq_TT_new} admit the homogeneous solution
\begin{equation} \label{eq_hos}
\rho = \bar{\rho}, \qquad \bm w = \begin{cases}
    \bm 0 & \mu(\bar\rho) \le 0 \\
    \sqrt{\mu(\bar{\rho})/\xi} \hat{\bm e}_\| & \mu(\bar\rho) > 0
\end{cases} ,
\end{equation}
where $\hat{\bm e}_\|$ is an arbitrary unit vector.
% For $\mu(\bar{\rho}) > 0$, the continuous theory admits the homogeneous ordered solution 
% \begin{equation} \label{eq_hos}
% \rho = \bar{\rho}, \qquad \bm w = \sqrt{\mu(\bar{\rho})/\xi} \hat{\bm e}_\| .
% \end{equation}

In this section, we are interested in the stability of the ordered solution and thus assume that $\mu(\bar\rho) > 0$.
We write the fields $\rho$ and $\bm w$ in terms of small perturbations around~\eqref{eq_hos}:
\begin{equation} 
\rho(\bm r,t) = \bar{\rho} + \delta\rho(\bm r,t), \qquad \bm w(\bm r,t) = \left( w_0 + \delta w_\|(\bm r,t) \right) \hat{\bm e}_\| + \delta w_\perp(\bm r,t) \hat{\bm e}_\perp ,
\end{equation}
with $w_0 \equiv \sqrt{\mu(\bar{\rho}) / \xi}$ and $\hat{\bm e}_\| \cdot \hat{\bm e}_\perp = 0$.
Keeping terms up to linear order, we obtain for the perturbations
\begin{subequations}
\label{eq_TT_lin}
\begin{align}
\label{eq_TT_lin_rho}
&\partial_t \delta \rho + \partial_\| \delta w_\| + \partial_\perp \delta w_\perp = 0, \\
\label{eq_TT_lin_para}
&\partial_t \delta w_\| + \lambda w_0  \partial_\| \delta w_\| = \left( \mu' w_0 - \frac{1}{2}\partial_\| \right)\delta \rho - 2\mu(\bar{\rho}) \delta w_\| + \nu \Delta \delta w_\| , \\
\label{eq_TT_lin_perp}
&\partial_t \delta w_\perp + \lambda w_0  \partial_\| \delta w_\perp =  - \frac{1}{2}\partial_\perp \delta \rho + \nu \Delta \delta w_\perp ,
\end{align}
\end{subequations}
where we have defined $\mu' \equiv {\rm d}\mu /{\rm d}\rho|_{\bar{\rho}}$ and $\partial_{\| , \perp} \equiv \hat{\bm e}_{\| , \perp} \cdot \nabla$.

Going into Fourier space, we recast Eqs.~\eqref{eq_TT_lin} into the linear system
\begin{equation*}
    \partial_t \begin{pmatrix}
        \delta\hat{\rho} \\
        \delta\hat{w}_\| \\
        \delta\hat{w}_\perp
    \end{pmatrix} = 
    \begin{pmatrix}
        0 & i q_\| & i q_\perp \\
        \mu'(\bar\rho)w_0 + \tfrac{1}{2}i q_\| & i\lambda w_0 q_\| - [2\mu(\bar\rho) + \nu q^2] & 0 \\
        \tfrac{1}{2}i q_\perp & 0 & i\lambda w_0 q_\| - \nu q^2
    \end{pmatrix}
    \begin{pmatrix}
        \delta\hat{\rho} \\
        \delta\hat{w}_\| \\
        \delta\hat{w}_\perp
    \end{pmatrix},
\end{equation*}
where $\delta\hat{\rho}(\bm q,t) = \int\dd\bm r \, \delta\rho(\bm r,t)\exp(i\bm q\cdot\bm r)$ and the other fields are defined in a similar way, while $\bm q = q_\|\hat{\bm e}_\| + q_\perp\hat{\bm e}_\perp$ with $q^2 = q_\|^2 + q_\perp^2$.
The stability of the homogeneous ordered solution is then set by the sign of the real part of the eigenvalue of the above matrix.

Given the configuration of the Vicsek bands (see~\autoref{figVM}), we moreover focus on perturbations in the direction longitudinal to the order here by setting $q_\perp = 0$.
This way, we directly obtain the eigenvalue associated to $\delta\hat{w}_\perp$: ${\cal Y}_\perp = -\nu q_\|^2 + i \lambda w_0 q_\|$,
whose real part is always negative.
The other two eigenvalues solve the second order polynomial equation
\begin{equation}\label{eq_poly_Y}
    {\cal Y}({\cal Y} + 2\mu(\bar\rho) + \nu q_\|^2 - i \lambda w_0 q_\|) - i q_\|\left(\mu'(\bar\rho) w_0 + \tfrac{1}{2}i q_\|\right) = 0.
\end{equation}
For $q_\| = 0$, it is straightforward to show that the solutions of this equation are ${\cal Y}_+ = 0$ and ${\cal Y}_- = -2\mu(\bar\rho)$.
As we are interested in long-wavelength instabilities, we focus here on ${\cal Y}_+$ and make use of the ansatz
\begin{equation*}
    {\rm Re}({\cal Y}) \underset{q_\| \to 0}{\sim} q_\|^2, \qquad
    {\rm Im}({\cal Y}) \underset{q_\| \to 0}{\sim} q_\|.
\end{equation*}
Replacing this ansatz in Eq.~\eqref{eq_poly_Y}, we obtain at order $q_\|^2$
\begin{equation*}
    2 \mu(\bar\rho){\rm Re}({\cal Y}) - {\rm Im}^2({\cal Y}) + \lambda w_0 q_\| {\rm Im}({\cal Y}) + \tfrac{1}{2}q_\|^2 + i\left[ 2\mu(\bar\rho) {\rm Im}({\cal Y}) - q_\| \mu'(\bar\rho) w_0 \right] = 0,
\end{equation*}
from which we deduce
\begin{equation}
    {\rm Im}({\cal Y}) = \frac{\mu'(\bar\rho)w_0}{2\mu(\bar\rho)}q_\|, \qquad
    {\rm Re}({\cal Y}) = \frac{q_\|^2}{4\mu(\bar\rho)}\left[ \frac{(\mu'(\bar\rho))^2}{2 \mu(\bar\rho)\xi} - \frac{\lambda\mu'(\bar\rho)}{\xi} - 1 \right].
\end{equation}
In particular, at the onset of orientational order $\mu(\bar\rho)\to0$, such that the real part of the growth rate simplifies to ${\rm Re}({\cal Y}) \sim (\mu'(\bar\rho))^2 q_\|^2/(8\mu^2(\bar\rho)\xi)$, which is always positive.
We thus conclude from this analysis that the homogeneous polar ordered solution of Eqs.~\eqref{eq_TT_new} is always unstable to longitudinal perturbations as one approaches the transition.
This instability in fact leads to the emergence of polar bands similar to that observed in simulations of the microscopic dynamics.
As it appears clearly from the expression of ${\rm Re}({\cal Y})$, this instability results from the local coupling between density and order, which translates to $\mu'(\bar\rho) > 0$. 
Physically, the instability arises because denser regions are more ordered (order grows with density), while the presence of order leads the particles to travel together for longer (density grows with order).
The combination of these two effects creates a self-amplifying feedback loop leading to an instability at the ordering transition where the system is most sensitive to fluctuations. 

An analytical description of the one-dimensional band solutions of Eqs.~\eqref{eq_TT_new} can be achieved via a mapping of the dynamics to a nonlinear dynamical system (for details on the procedure, see Ref.~\cite{DADAM_LesHouches}).
This analysis notably shows that the band solutions indeed correspond to phase-separated configurations, and allows to calculate the corresponding binodal densities.
Hence, both simulations of the microscopic Vicsek model and the analysis of the generic continuous theory point to a liquid-gas scenario for the transition to collective motion. 

\section{The Toner Tu theory: linearized hydrodynamics}

\subsection{Identification of the hydrodynamic modes}

Here, we consider a set of coefficients ($\lambda,\mu(\bar{\rho}),\xi,\nu$) so that we work deep enough in the ordered phase where~\eqref{eq_hos} is stable.
Comparing the r.h.s.\ of Eq.~\eqref{eq_TT_lin_para} with that of Eqs.~(\ref{eq_TT_lin_rho},\ref{eq_TT_lin_perp}), 
we find that $\delta w_\|$ is the only perturbation whose equation presents a damping term with coefficient $- 2\mu(\bar{\rho})$.
Consequently, longitudinal order perturbations typically relax on a finite timescale $\tau_\| \sim (2\mu(\bar{\rho}))^{-1}$,
while for density and transverse order perturbations are massless.
Therefore, $\delta w_\|$ can be considered as fast, 
while $\delta \rho$ and $\delta w_\perp$ are the hydrodynamic modes.

Using this separation of timescales between the fields, 
we neglect the time derivative in Eq.~\eqref{eq_TT_lin_para} and keep only the leading order terms in fields and gradients, which leads to
\begin{equation*}
\delta w_\| \simeq \frac{1}{2\mu(\bar{\rho})}\left( \mu' w_0 - \frac{1}{2}\partial_\| \right) \delta \rho .
\end{equation*}
Replacing this expression in Eq.~\eqref{eq_TT_lin_rho}, we finally get the closed set of equations
\begin{subequations}
\label{eq_TT_lin_closed}
\begin{align}
\label{eq_TT_lin_rho_closed}
&\partial_t \delta \rho + \tilde{v}_0\partial_\| \delta \rho + \partial_\perp \delta w_\perp = D_\rho \partial_\|^2 \delta\rho + \nabla \cdot \bm f_\rho, \\
\label{eq_TT_lin_perp_closed}
&\partial_t \delta w_\perp + \lambda w_0  \partial_\| \delta w_\perp =  - \frac{1}{2}\partial_\perp \delta \rho + \nu \Delta \delta w_\perp + f_\perp,
\end{align}
\end{subequations}
with $\tilde{v}_0 \equiv \mu' w_0 /  2\mu(\bar{\rho})$ and $D_\rho \equiv 1/4\mu(\bar{\rho})$.
As we are interested in calculating the response of the system to fluctuations, 
we have moreover included the noise terms $\bm f_\rho$ and $f_\perp$ in Eqs.~\eqref{eq_TT_lin_closed}.
We consider them as independent Gaussian white noises satisfying
\begin{equation} \label{eq_varf}
\langle f_{\rho,i}(\bm r,t)  f_{\rho,j}(\bm r',t') \rangle = \Delta_\rho \delta_{ij}\delta(\bm r - \bm r')\delta(t - t') , \qquad
\langle f_\perp(\bm r,t)f_\perp(\bm r',t') \rangle = \Delta_\perp \delta(\bm r - \bm r')\delta(t - t') .
\end{equation}
Note that the noise in Eq.~\eqref{eq_TT_lin_rho_closed} has to appear under a gradient term due to the particle number conservation constraint.

\subsection{Space-time correlations}

%\textcolor{blue}{I think that the FT has to be defined as $e^{-i(q \cdot r..)}$ to be consistent with the other formulas. I change it, in any case correct if it wrong. Also the term of the density noise misses an $i$.}

We now go to Fourier space:
\begin{equation*}
\delta \hat{\rho}(\bm q,\omega) = \int{\rm d}\bm r \int {\rm d}t \,  \delta\rho(\bm r,t)e^{i \bm q \cdot r - i \omega t} , \qquad
\delta \hat{w}_\perp(\bm q,\omega) = \int{\rm d}\bm r \int {\rm d}t \,  \delta w_\perp(\bm r,t)e^{i \bm q \cdot r - i \omega t},
\end{equation*}
where Eqs.~\eqref{eq_TT_lin_closed} are recast into the following linear system
\begin{equation}
\label{eq_TT_lin_fourier}
\begin{pmatrix}
\Gamma_\rho(\bm q,\omega) & -i q_\perp \\
-i\frac{q_\perp}{2} & \Gamma_\perp(\bm q,\omega) 
\end{pmatrix}
\begin{pmatrix}
\delta \hat{\rho}(\bm q,\omega) \\ \delta \hat{w}_\perp(\bm q,\omega)
\end{pmatrix}
=
\begin{pmatrix}
 -i\bm q \cdot \hat{ \bm f}_\rho(\bm q,\omega) \\ \hat{f}_\perp(\bm q,\omega)
\end{pmatrix} ,
\end{equation}
where $q_\|$ and $q_\perp$ are the wavenumbers associated respectively with the directions longitudinal and transverse to the global order, 
$\hat{ \bm f}_\rho$ and $\hat{f}_\perp$ are the Fourier transforms of the noises,
and we have defined 
\begin{equation*}
\Gamma_\rho(\bm q,\omega) \equiv i(\omega - \tilde{v}_0 q_\|) + D_\rho q_\|^2 , \qquad 
\Gamma_\perp(\bm q,\omega) \equiv i(\omega - \lambda w_0 q_\|) + \nu (q_\|^2 + q_\perp^2) .
\end{equation*}
In the following, we will focus on the scaling of the fields correlation functions in the large scale and long time limits, i.e.\ in the limits of vanishing $\omega$ and $q \equiv |\bm q|$.
In this regime the conserved noise term in Eq.~\eqref{eq_TT_lin_fourier} is subdominant w.r.t.\ $\hat{f}_\perp$, and we neglect it.
The system~\eqref{eq_TT_lin_fourier} is easily solved formally, and we find
\begin{equation}
\begin{pmatrix}
\delta \hat{\rho}(\bm q,\omega) \\ \delta \hat{w}_\perp(\bm q,\omega)
\end{pmatrix} = {\cal D}^{-1}(\bm q,\omega) 
\begin{pmatrix}
i q_\perp \\ \Gamma_\rho(\bm q,\omega)
\end{pmatrix} \hat{f}_\perp(\bm q,\omega) ,
\end{equation}
with ${\cal D}(\bm q,\omega) \equiv \Gamma_\rho(\bm q,\omega)\Gamma_\perp(\bm q,\omega) + q_\perp^2/2$.
In the limit where $\omega$ and $q$ go to zero, we simplify the expression of ${\cal D}(\bm q,\omega)$ by considering the following factorization
\begin{equation*}
{\cal D}(\bm q,\omega) \underset{\omega, q \to 0}{\simeq} -(\omega - \omega_+(\bm q))(\omega - \omega_-(\bm q)) ,
\end{equation*}
and where we assume that ${\rm Re}(\omega_\pm) \sim q$ and ${\rm Im}(\omega_\pm) \sim q^2$.
Replacing $w$ by $w_\pm(\bm q)$ in the expression of ${\cal D}(\bm q,\omega)$ and keeping terms up to order $q^2$ we end up with the dispersion relations 
$\omega_\pm(\bm q) = c_\pm(\theta_{\bm q}) q - i \varepsilon_\pm(\bm q)$, with
\begin{subequations}
\label{eq_speed_damp}
\begin{align}
\label{eq_speeds}
c_\pm(\theta_{\bm q}) &\equiv \frac{\lambda w_0 + \tilde{v}_0}{2}\cos(\theta_{\bm q}) \pm \sqrt{ \frac{(\lambda w_0 - \tilde{v}_0)^2}{4}\cos^2(\theta_{\bm q}) + \frac{\sin^2(\theta_{\bm q})}{2}} , \\
\label{eq_dampings}
\varepsilon_\pm(\bm q) &\equiv \frac{ [\tilde{v}_0\nu + D_\rho\lambda w_0\cos^2(\theta_{\bm q})  ]\cos(\theta_{\bm q}) - c_\pm(\theta_{\bm q})[\nu + D_\rho \cos(\theta_{\bm q}) ] }
{2 c_\pm(\theta_{\bm q}) + (\lambda w_0 + \tilde{v}_0)\cos(\theta_{\bm q}) } q^2 \sim q^2 ,
\end{align}
\end{subequations}
and where we have defined $\theta_{\bm q}$ as the angle between $\bm q$ and the global order ($q_\| = q\cos\theta_{\bm q}$ and $q_\perp = q\sin\theta_{\bm q}$). 
Using the expression of the noise variance in Fourier space $\langle \hat{f}_\perp(\bm q,\omega) \hat{f}_\perp(\bm q',\omega') \rangle = \Delta_\perp (2\pi)^3 \delta(\bm q + \bm q') \delta(\omega + \omega')$, we get the following expressions of the density and order correlation functions
\begin{subequations}
\label{eq_space_tim_cfs}
\begin{align}
\label{eq_space_tim_cfs_rho}
\left\langle \left|\delta \hat{\rho}(\bm q,\omega)\right|^2 \right\rangle &\underset{\omega, q \to 0}{\simeq} \frac{\Delta_\perp q_\perp^2}
{\left[ ( \omega - c_+(\theta_{\bm q}) q)^2  + \varepsilon^2_+(\bm q)   \right]\left[ ( \omega - c_-(\theta_{\bm q}) q)^2  + \varepsilon^2_-(\bm q)   \right]} , \\
\label{eq_space_tim_cfs_w}
\left\langle \left|\delta \hat{w}_\perp(\bm q,\omega)\right|^2 \right\rangle &\underset{\omega, q \to 0}{\simeq} \frac{\Delta_\perp (\omega - \tilde{v}_0 q_\|)^2}
{\left[ ( \omega - c_+(\theta_{\bm q}) q)^2  + \varepsilon^2_+(\bm q)   \right]\left[ ( \omega - c_-(\theta_{\bm q}) q)^2  + \varepsilon^2_-(\bm q)   \right]} .
\end{align}
\end{subequations}
As shown in Figs.~\ref{figmodes}(a,b), the density and order correlation functions plotted as function of $\omega$ at fixed $\bm q$ thus exhibit two peaks centred in $c_\pm(\theta_{\bm q})q$ and of widths $\varepsilon_\pm(\bm q)\sim q^2$.
Such structure reflects the presence of propagating density and order waves in the system with dispersion relations $\omega = \omega_\pm(\bm q)$.
Namely, these modes propagate with characteristic speeds $c_\pm(\theta_{\bm q})$ that depend on the orientation of $\bm q$ w.r.t.\ the mean order direction (see Fig.~\ref{figmodes}(c)). 
This is a direct consequence of fact that we are analyzing a symmetry broken phase.
Moreover, measuring the two different speeds of the modes in the direction longitudinal to order gives 
a direct indication of the absence of Galilean invariance in the system ($\tilde{v}_0 \ne \lambda w_0$).
The predicted life time of the modes takes a more conventional form set by $\varepsilon^{-1}_\pm(\bm q)\sim q^{-2} \sim \lambda^2$ with $\lambda$ the mode wavelength, which corresponds to the expected scaling for diffusive damping.

As we derived the structure of the propagating density and order modes from the generic hydrodynamic theory, 
we expect it to hold for a broad variety of systems.
Simulations of the Vicsek model~\eqref{eq_VM} indeed confirm the two-peaks structure predicted by Eqs.~\eqref{eq_space_tim_cfs}.
Measuring the peaks positions, one finds that they indeed scale linearly with $q$. 
The associated slope corresponds to the modes speeds which are found to vary with the orientation $\theta_{\bm q}$ 
in perfect agreement with Eq.~\eqref{eq_speeds} (as shown in Fig.~\ref{figmodes}(c)).
This phenomenology moreover holds beyond the Vicsek setting, as found in numerical simulations of a realistic model of self-propelled polar rods~\cite{Soni2020}
as well as in colloidal rollers experiments~\cite{geyer2018sounds}.

%%%%%%%%%%%%%%%%%%%%%%%%%%%%
\begin{figure}[t!]
	\includegraphics[width=\textwidth]{Figures/figmodes.pdf}
	\caption{(a,b) density(blue) and order(red) space time correlation functions as function of the frequency $\omega$ for $\theta_{\bm q} = \pi/2$(a) and $\pi/4$(b),
	respectively showing two symmetric and asymmetric peaks.
	When present, the continuous lines are fits of the numerical data (points) with Eqs.~\eqref{eq_space_tim_cfs}.
	(c) Polar representation of the mode speeds obtained from the scaling of the density and order correlation functions peaks position with $q$ at various values of $\theta_{\bm q}$. 
	The full line shows a fit of the data with Eq.~\eqref{eq_speeds}.}
	\label{figmodes}
\end{figure}
%%%%%%%%%%%%%%%%%%%%%%%%%%%%

\subsection{Equal-time correlation functions}

We now investigate on the nature of order as well as the presence of giant density fluctuations, which are both related to the equal-time correlation functions.
Equal-time density and order correlations are simply obtained by integrating Eqs.~\eqref{eq_space_tim_cfs} over the frequency $\omega$.
The full calculation can be carried out explicitly, but can be simplified considering the limit of small wavenumbers.
Indeed, although the two peaks in the space-time correlation functions are separated by $\sim q$, their respective widths $\sim q^2$ and their heights $\sim q^{-2}$.
Therefore, for $q \to 0$ the integral over $\omega$ of Eqs.~\eqref{eq_space_tim_cfs} will be dominated by the contributions from the individual peaks, 
which can be treated independently.
Using the relation $\int_{-\infty}^{+\infty} {\rm d}x \, [(x - x_0)^2 + \alpha^2]^{-1} = \pi/\alpha$, we thus get after little algebra
%\begin{subequations}
%\label{eq_eq_tim_cfs}
\begin{align*}
%\label{eq_eq_tim_cfs_rho}
\left\langle \left|\delta \hat{\rho}(\bm q)\right|^2 \right\rangle &\underset{q \to 0}{\simeq} \frac{\Delta_\perp \sin^2(\theta_{\bm q})}{ 2[c_+(\theta_{\bm q}) - c_-(\theta_{\bm q})]^2 }
\left( \frac{1}{\varepsilon_+(\bm q)} + \frac{1}{\varepsilon_-(\bm q)}  \right) , \\
%\label{eq_eq_tim_cfs_w}
\left\langle \left|\delta \hat{w}_\perp(\bm q)\right|^2 \right\rangle &\underset{q \to 0}{\simeq} \frac{\Delta_\perp}{ 2[c_+(\theta_{\bm q}) - c_-(\theta_{\bm q})]^2}
\left( \frac{ [ c_+(\theta_{\bm q}) - \tilde{v}_0\cos(\theta_{\bm q}) ]^2 }{\varepsilon_+(\bm q)} + \frac{[ c_-(\theta_{\bm q}) - \tilde{v}_0\cos(\theta_{\bm q}) ]^2}{\varepsilon_-(\bm q)}  \right) .
\end{align*}
%\end{subequations}
Using the expressions of the $c_\pm(\theta_{\bm q})$ and $\varepsilon_\pm(\bm q)$ given in Eq.~\eqref{eq_speed_damp} we finally rewrite these expressions into a more compact form:
\begin{equation}
\label{eq_eq_tim_cfs_simple}
\left\langle \left|\delta \hat{\rho}(\bm q)\right|^2 \right\rangle \underset{q \to 0}{\simeq} \frac{P_\rho(\theta_{\bm q})\sin^2(\theta_{\bm q})}{q^2} , \qquad
\left\langle \left|\delta \hat{w}_\perp(\bm q)\right|^2 \right\rangle \underset{q \to 0}{\simeq} \frac{P_\perp(\theta_{\bm q})}{q^2} ,
\end{equation}
with $P_\rho(\theta_{\bm q})$ and $P_\perp(\theta_{\bm q})$ two ${\cal O}(1)$ functions.
In agreement with numerical and experimental observations, density and order correlations are scale free (they scale like a power of $q$).
Moreover, although we derived Eqs.~\eqref{eq_eq_tim_cfs_simple} starting from a two dimensional theory, this result holds in arbitrary dimensions $d \ge 1$~\cite{toner2012reanalysis}.

\subsubsection{Giant density fluctuations}

The scaling of density fluctuations is obtained through the relation $\langle \Delta n^2 \rangle / \langle n \rangle \sim \lim_{q\to 0} \langle |\delta \hat{\rho}(\bm q)|^2 \rangle$.
Conventional density fluctuations are thus found when the Fourier space density correlation saturates to a constant value as $q \to 0$.
Here, on the contrary, $\langle |\delta \hat{\rho}(\bm q)|^2 \rangle$ diverges at vanishing wavenumbers like $q^{-2}$.
Using that $q \sim 1/L \sim 1/\langle n \rangle^{1/d}$, this implies
\begin{equation}
\langle \Delta n^2 \rangle \sim \langle n \rangle^\phi ,
\end{equation}
with $\phi = 1 + 2/d$.
Therefore, although our theory correctly predicts the existence of giant density fluctuations in all dimensions $d \ge 1$, 
the corresponding predicted exponent $\phi = 2$ for $d=2$ is larger than that measured in simulations or experiments (see e.g.\ Fig.~\ref{figVM}(c)).

\subsubsection{The nature of order}

For any dimension $d$, the typical strength of order fluctuations is set by 
\begin{equation}
\label{eq_order_flucts}
\left\langle \left|\delta {w}_\perp(\bm r)\right|^2 \right\rangle = \int \frac{{\rm d}\bm q}{(2\pi)^d} \left\langle \left|\delta \hat{w}_\perp(\bm q)\right|^2 \right\rangle 
\underset{q\to 0}{\sim} \int_{1/L}^{1/\Lambda} {\rm d}q \, q^{d-3} \underset{L\to +\infty}{\sim} 
\begin{cases}
L & (d=1) \\
\ln(L / \Lambda) & (d=2) \\
\Lambda^{2-d} & (d \ge 3)
\end{cases},
\end{equation}
with $\Lambda$ and ultra-violet cutoff length.
Eq.~\eqref{eq_order_flucts} predicts that the global polar order is destroyed by fluctuations in the limit of infinite system sizes for dimensions $d \le 2$.
More specifically, the logarithmic divergence predicted in $d=2$ is `slow enough' that, although the order vanishes asymptotically, the associated correlation length remains infinite.
Consequently, the global polarization decays with system size as $L^{-\kappa}$ with the exponent $0 < \kappa < 1$ usually nonuniversal.
This regime corresponds e.g.\ to that of the ordered phase of the two dimensional $XY$ model at equilibrium, and is named `quasi-long-range order'.
In $d \ge 3$, however, Eq.~\eqref{eq_order_flucts} predicts bounded polar order fluctuations so that order should remain long-ranged.\\
\\
In summary, we have derived the scaling of density and order correlation functions around the fluctuating homogeneous ordered solution of the Toner Tu equations~\eqref{eq_TT_new}.
The theory succeeds in predicting the presence of propagating density and order modes, giant density fluctuations, as well as the scale free correlations observed in numerical simulations and experiments.
However, some discrepancies persist: the exponent associated with giant density fluctuations is measured to be smaller than its predicted value, 
and most importantly the theory predicts a quasi-long-range order in two dimensions, 
at odds with observations of long-range orientational order~\eqref{eq_LRO} in numerical simulations of the two dimensional Vicsek model (see Fig.~\ref{figVM}(c)).
As we show below, these differences can be qualitatively explained by the fact that the effect of nonlinearities in the Toner Tu equations cannot be neglected in low dimensions, 
so that the linearized theory effectively breaks down.

\section{Some elements of the nonlinear theory}

To evaluate the importance of nonlinearities, we need to derive a nonlinear version of Eqs.~\eqref{eq_TT_lin_closed}.
Then, the question arises of which nonlinear terms should be kept and which should be discarded. The method to evaluate the relevance of these additional terms is called scaling analysis, or more formally the Renormalization Group (RG). 

The main idea behind this is the validity of the scaling hypothesis: when the system is largely correlated, i.e. $\ell_c \gg \Lambda$, the correlation length is the only relevant length scale of the dynamics, and it determines the macroscopic behavior through universal critical exponents. Indeed, all the microscopic details on scales smaller than $\ell_c$ do not really matter in a hydrodynamic description of the system. This is even more evident when $\ell_c \to \infty$, namely at the critical point of a phase transition or in the presence of spontaneous symmetry breaking with a Goldstone mode. In these cases, the system is scale invariant: the fluctuations extend on all the length scales (scale-free), and the physical properties are preserved when the observation scale is changed. To understand this, let's change the unit of space $r \to a r$, shrinking or dilating it with a scale factor $a$. We learned that when a system is scale-free, the correlation function of the order parameter decays with a power law $C(r) \sim r^{-\gamma}$. Performing the rescaling, we obtain
$$
C(ar) = \frac{1}{(ar)^{\gamma}} = a^{-\gamma} C(r) \ .
$$
The correlation function of the rescaled system is thus equal to the old correlation function multiplied by a constant factor: space rescaling does not change the physics when scale invariance is satisfied.  The RG is the set of transformations that allows to find the points of scale invariance (fixed points) of a specific field theory. It provides the information on how all the involved parameters have to change when space-time is rescaled, and the system is observed at larger and larger scales. 

We are now going to use RG arguments to understand which nonlinearities are relevant. For simplicity, we will consider here only the leading order nonlinearities, which are quadratic in the fields and linear in gradients. 
Carrying out the procedure detailed above, we find after enslaving the longitudinal order fluctuations\footnote{For the full derivation in the general case, see~\cite{toner2012reanalysis}.}
\begin{subequations}
\label{eq_TT_nlin_closed}
\begin{align}
\label{eq_TT_nlin_rho_closed}
&\partial_t \delta \rho + \tilde{v}_0\partial_\| \delta \rho + \partial_\perp \delta w_\perp + w_1 \partial_\perp( \delta \rho \delta w_\perp ) = 
D_\rho \partial_\|^2 \delta\rho + w_2\partial_\|  \delta \rho^2 + w_3 \partial_\|  \delta w_\perp^2 , \\
\label{eq_TT_nlin_perp_closed}
&\partial_t \delta w_\perp + \lambda w_0  \partial_\| \delta w_\perp + \lambda\delta w_\perp  \partial_\perp \delta w_\perp =  
- \frac{1}{2}\partial_\perp \delta \rho + \nu \Delta \delta w_\perp + g_1 \delta \rho \partial_\|  \delta w_\perp + g_2\delta w_\perp\partial_\|\delta \rho  + f_\perp,
\end{align}
\end{subequations}
and where, as previously, we have discarded the conserved noise term in Eq.~\eqref{eq_TT_nlin_rho_closed}.
Note, moreover, that all nonlinear terms in Eq.~\eqref{eq_TT_nlin_rho_closed} are total derivatives as a consequence of the particle number conservation.

We have shown from the linearized theory that the correlation functions of density and order in the ordered phase are scale free in $d\ge 2$,
therefore 
%from usual statistical mechanics arguments 
we assume the following scaling ansatz
\begin{equation} \label{eq_rescaling}
\bm r_\perp \to a \bm r_\perp , \quad r_\| \to a^\xi r_\| , \quad t \to a^z t , \quad \delta\rho \to a^{\chi_\rho}\delta\rho , \quad \delta w_\perp \to a^{\chi}\delta w_\perp ,
\end{equation}
with $a$ the given scaling factor.
The exponents $\xi$, $z$ $\chi_\rho$ and $\chi$ are {\it a priori} unknown, but as we will show below they can be fully determined for the linear theory.
$\xi$ measures the anisotropy of the scaling, while $\xi = 1$ corresponds to isotropy.
$z$ is called the dynamical exponent, and sets how the typical life time of perturbations scales with distances. 
$\chi_\rho$ and $\chi$ are the roughness exponents associated with density and order, respectively. 
They measure how the amplitude of fluctuations scales with distances. 

Applying the rescaling~\eqref{eq_rescaling} to Eqs.~\eqref{eq_TT_nlin_closed}, we get
\begin{subequations}
\label{eq_TT_rescalednonlin_closed}
\begin{align}
&\partial_t \delta \rho + \tilde{v}_0 a^{z-\xi} \partial_\| \delta \rho + a^{z + \chi - \chi_\rho -1} \partial_\perp \delta w_\perp + w_1 a^{z + \chi - 1} \partial_\perp( \delta \rho \delta w_\perp ) = \nonumber \\
& \qquad\qquad\qquad\qquad\qquad\qquad\qquad\qquad\qquad\qquad 
D_\rho a^{z-2\xi} \partial_\|^2 \delta\rho + w_2 a^{z + \chi_\rho - \xi}  \partial_\|  \delta \rho^2 + w_3 a^{z + 2\chi - \chi_\rho - \xi} \partial_\|  \delta w_\perp^2 , \\
&\partial_t \delta w_\perp + \lambda w_0 a^{z-\xi} \partial_\| \delta w_\perp + \lambda a^{z + \chi -1} \delta w_\perp  \partial_\perp \delta w_\perp  =  
- \frac{ a^{z + \chi_\rho - \chi - 1} }{2}\partial_\perp \delta \rho + \nu \left( a^{z-2\xi} \partial_\|^2 + a^{z-2} \partial_\perp^2 \right) \delta w_\perp \nonumber \\
& \qquad\qquad\qquad\qquad\qquad\qquad\qquad\qquad 
+ g_1 a^{z + \chi_\rho - \xi} \delta \rho \partial_\|  \delta w_\perp + g_2  a^{z + \chi_\rho - \xi} \delta w_\perp\partial_\|\delta \rho  + a^{\tfrac{z - 2\chi - (d-1) - \xi}{2}} f_\perp.
\end{align}
\end{subequations}
To calculate how the noise $f_\perp$ renormalizes upon the rescaling~\eqref{eq_rescaling}, we have used the relation~\eqref{eq_varf} 
as well as the fact that $\int {\rm d}\bm r' \, \delta(\bm r - \bm r') = 1$ and $ \int {\rm d}t' \, \delta(t - t') = 1$ must hold independently of $a$,
which implies $f_\perp^2 \to a ^{-(z + d-1 + \xi)} f_\perp^2$.
Each parameter of the theory could then be rescaled to preserve the same effective theory, for instance,
\begin{align*}
    D_\rho' &= D_\rho a^{z-2 \xi}, \\
    \nu' &= \nu a^{z-2}, \\
    \Delta_\perp' &= \Delta_\perp a^{(z-2\chi -(d-1)-\xi)/2}.
\end{align*}

To determine how the nonlinearities in Eqs.~\eqref{eq_TT_rescalednonlin_closed} renormalize upon the rescaling, 
we now evaluate the exponents in the linear theory. First, the continuity equation~\eqref{eq_TT_rho_new} combined with the dispersion relation $\omega \simeq c_\pm(\theta_{\bm q}) q$ implies that 
$c_\pm(\theta_{\bm q}) q \delta\hat{\rho} \sim q_\perp \delta \hat{w}_\perp$, which leads to $\chi_\rho = \chi$.
Secondly, the rescaling~\eqref{eq_rescaling} must leave the equal-time correlation functions~\eqref{eq_eq_tim_cfs_simple} invariant.
This can be achieved if the diffusion constants $\nu$ and $D_\rho$, as well as the noise $\Delta_\perp$, do not renormalize,
which from~\eqref{eq_TT_rescalednonlin_closed} implies the following hyperscaling relations
\begin{equation*}
z - 2 = 0, \qquad z - 2\xi = 0, \qquad z - 2\chi - (d-1) - \xi = 0 .
\end{equation*}
Solving for the three exponents, we then obtain their value given by the linear theory
\begin{equation}
z_{\rm lin} = 2, \qquad \xi_{\rm lin} = 1, \qquad \chi _{\rm lin} = 1 - \frac{d}{2} .
\end{equation}
In agreement with the results of the previous section, the linear theory thus predicts an isotropic scaling ($\xi = 1$), with diffusive damping of the modes ($z =2$)
and long-range order in $d \ge 3$ only ($\chi < 0$)\footnote{In two dimensions $\chi = 0$ but velocity correlations still diverge due to logarithmic corrections which cannot be captured by the scaling argument.}.

We now replace these values of the exponents in the nonlinear terms of  Eq.~\eqref{eq_TT_rescalednonlin_closed} to understand if the corresponding parameters grow or decrease upon rescaling. If they grow, the nonlinear terms are said to be {\it relevant} (positive scaling dimension), in the sense that they carry the system away from the linear fixed point. On the other hand, if they are irrelevant (negative scaling dimension) a small perturbation in their direction will be stable for the linear fixed point. We find that all nonlinearities renormalize as
\begin{equation}
\lambda; w_{1,2,3}; g_{1,2} \to a^{ \tfrac{4-d}{2} } \lambda; w_{1,2,3}; g_{1,2} .
\end{equation}
Therefore, for $d \ge 4$ all nonlinearities won't grow upon looking at the dynamics on larger scales. 
On the contrary, for $d < 4$ the strength of these nonlinearities grows as the theory is applied to larger scales and longer times, 
they are therefore relevant.
The upper critical dimension of the model, i.e.\ the dimension below which nonlinearities are important, is thus $d_c = 4$.

The the full nonlinear problem could be in principle addressed via renormalization group techniques, 
this is however a notoriously difficult task~\cite{toner2012reanalysis} which has not been performed so far.
Toner and Tu nevertheless managed to derive an expression for the scaling of the correlation functions using a matching procedure between the linear and nonlinear problems~\cite{toner1998flocks}.
They could show that the scaling of correlation functions could be obtained simply from their expressions~(\ref{eq_space_tim_cfs},\ref{eq_eq_tim_cfs_simple}) 
in the nonlinear theory with the renormalized noise variance and dampings
\begin{equation} \label{eq_renormalized_params}
\Delta_\perp(\bm q) = \Delta_\perp^0 q_\perp^{z - (d-1 + 2\chi + \xi)} {\cal F}_\Delta \left( \frac{q_\| }{q_\perp^\xi} \right) , \qquad
\varepsilon_\pm(\bm q) = \varepsilon_\pm^0 q_\perp^z {\cal F}_\pm \left( \frac{q_\| }{q_\perp^\xi} \right) ,
\end{equation}
while the modes speeds $c_\pm(\theta_{\bm q})$ remain those given by the linear theory, 
and where the functions ${\cal F}_\perp$ and ${\cal F}_\pm$ are unknown but universal and satisfy
\begin{equation*}
{\cal F}_\Delta(x) \underset{x \to 0}{\to} {\rm const} , \qquad {\cal F}_\Delta(x) \underset{x \to +\infty}{\to} x^{\tfrac{z - (d-1 + 2\chi + \xi)}{\xi}} ,\qquad
{\cal F}_\pm(x) \underset{x \to 0}{\to} {\rm const} , \qquad {\cal F}_\pm(x) \underset{x \to +\infty}{\to} x^{\tfrac{z}{\xi}} .
\end{equation*}
Replacing Eq.~\eqref{eq_renormalized_params} into Eq.~\eqref{eq_eq_tim_cfs_simple}, for instance, leads up to some ${\cal O}(1)$ factors
to the following expression of the equal-time correlation functions
\begin{equation} \label{eq_nl_scaling}
\left\langle \left|\delta \hat{\rho}(\bm q)\right|^2 \right\rangle \underset{q \to 0}{\simeq} \frac{q_\perp^{2 - (d-1 + 2\chi + \xi)}}{q^2}{\cal F}_\rho \left( \frac{q_\| }{q_\perp^\xi} \right)  , \qquad
\left\langle \left|\delta \hat{w}_\perp(\bm q)\right|^2 \right\rangle \underset{q \to 0}{\simeq} q_\perp^{- (d-1 + 2\chi + \xi)}{\cal F}_\perp \left( \frac{q_\| }{q_\perp^\xi} \right) ,
\end{equation}
with ${\cal F}_{\rho,\perp}(x) \underset{x \to 0}{\to} {\rm const}$ and  ${\cal F}_{\rho,\perp}(x) \underset{x \to 0}{\to} x^{\tfrac{- (d-1 + 2\chi + \xi)}{\xi}}$.
The scaling form~\eqref{eq_nl_scaling} was later confirmed by numerical simulations of the Vicsek model (for the full analysis, see~\cite{mahault2019TT}),
with the numerically determined exponents in two dimensions
\begin{equation} \label{eq_exps}
\chi = -0.31(2) , \qquad \xi = 0.95 (2) , \qquad  z = 1.33(2) .
\end{equation}
Therefore, it turns out that the scaling of correlation functions is nearly isotropic ($\xi \lesssim 1$), 
while the true long range order is confirmed ($\chi < 0$)\footnote{Actually, the exponent $\omega$ defined in Sec.~\ref{sec_pd} can be shown to be equal to $-2\chi$.}.
The fact that $z< 2$ moreover indicates that density and order fluctuations are damped 
on large scales more heavily than predicted by the linear theory (their life time scales as $\lambda^z$, with $\lambda$ their wavelength).
Going back to the giant density fluctuations, the exponents~\eqref{eq_exps} moreover imply that $\phi \simeq 1.67$, 
which is a value compatible with that usually directly measured from the scaling of $\langle \Delta n^2\rangle$ with $\langle n \rangle$ (see Fig.~\ref{figVM}(c)).

Although no proper RG treatment of Eqs.~\eqref{eq_TT_nlin_closed} has been carried out yet, recently alternative approaches based on scaling arguments have been proposed to predict the values of the exponents in dimension $d=2$.
Namely, the idea is first to note that for $\delta w_\perp$ to be a Goldstone mode, the deterministic part of its dynamics must be written as a total derivative (otherwise, the nonlinear terms that are not total derivatives effectively generate a mass), which implies in practice that $g_1 = g_2$.
As a consequence, the noise in Eq.~\eqref{eq_TT_nlin_perp_closed} cannot be renormalized by nonlinearities as the short scales of the theory are integrated out, since it is the only contribution on the r.h.s.\ that is not a total derivative. 
Hence, at the RG fixed point the hyperscaling relation
\begin{equation*}
    z = 1 + 2\chi + \xi,
\end{equation*}
must be satisfied.
Two other scaling relations can moreover be obtained by noting that Eqs.~\eqref{eq_TT_nlin_closed} remain invariant under a generalized Galilean transformation.
The details of this argument are quite subtle, so we refer to~\cite{chate2024dynamic} for details, but as a result the coefficients $w_{1,2,3},\lambda,g_1$ in front of the nonlinearities in Eqs.~\eqref{eq_TT_nlin_closed} are also not renormalized under RG transformation.
This implies two other scaling relations at the fixed point:
\begin{equation*}
    \chi + z - 1 = 0, \qquad
    \chi + z - \xi = 0.
\end{equation*}
Solving for $\chi$, $\xi$ and $z$, we then conclude that
\begin{equation}
    \chi = -\frac{1}{3}, \qquad
    \xi = 1, \qquad
    z = \frac{4}{3}.
\end{equation}
Note that these values are indeed very close to the ones obtained from numerical simulations (Eq.~\eqref{eq_exps}).

 % Appart from the giant density fluctuations, the scaling laws giving access to the values of the exponents $\chi$, $\xi$ and $z$ are difficult to measure
 % due to the strong finite size effects present in the Vicsek model. 
 % These effects are moreover amplified considering more realistic models with repulsion, or in experiments.
 % Therefore, for far there has not been any convincing experimental confirmation of the values of the exponents~\eqref{eq_exps}.


    \chapter{Nematic active matter  - Lectures 11\&12}
    \label{chap_nematic}
    Notes from Navdeep Lectures 11 \& 12

    \appendix

    \chapter{Some basics on equilibrium phase separation}
    \label{chapter: phase sep}

\section{Passive model B and Cahn-Hilliard dynamics}

\label{sec_PMB}

\subsection{The dynamical theory for conserved scalar order parameter} 

We have seen in~\autoref{chap_thermo} that the large-scale dynamics of systems that microscopically obey detailed balance minimize a free energy.
Therefore, model B that describes the dynamics of systems with a conserved scalar order parameters $\phi$ is generally written in terms of a functional ${\cal F}[\phi]$ that we express as
%
\begin{equation} \label{eq_F}
{\cal F}[\phi] = \intd{r} \left[ f(\phi) + \frac{\kappa(\phi)}{2}|\nabla\phi|^2\right],
\end{equation} 
%
where we have kept terms up to order $|\nabla \phi|^2$.
In Eq.~\eqref{eq_F} $f$, denotes the `bulk' free energy density and $\kappa(\phi) > 0$ is a generic function.
The bulk contribution consists of a free energy landscape due to e.g.\ entropic effects and interactions between microscopic elements, 
while the $\kappa$ term describes the cost of interfaces.

The dynamical equation for $\phi$ takes the general form
\begin{equation} \label{eq_phi}
\partial_t \phi(\bm r,t) = - \nabla \cdot \bm J(\phi) ,
\end{equation}
where the current $\bm J = \bm J_{\rm D} + \bm J_{\rm S}$ has a deterministic and stochastic contributions.
Up to some mobility $\bm M$, the deterministic part of the current $\bm J_{\rm D}$ is given by the chemical potential $\mu$:
\begin{equation} \label{eq_JD}
\bm J_{\rm D}(\phi) = - \bm M(\phi) \cdot \nabla \mu(\phi), \qquad \mu(\phi) = \frac{\delta {\cal F}}{\delta \phi} = f'(\phi) - \kappa(\phi) \nabla^2\phi - \frac{\kappa'(\phi)}{2}|\nabla\phi|^2 .
\end{equation}
The stochastic part, $\bm J_{\rm S}$, can be determined using the fluctuation dissipation relation:
\begin{equation} \label{eq_JS}
\bm J_{\rm S}(\bm r,t) = \sqrt{2 k_B T} \bm \sigma(\phi) \cdot \bm \Lambda(\bm r,t), \qquad \Lambda_i(\bm r,t)\Lambda_j(\bm r',t') = \delta_{ij}\delta^d(\bm r - \bm r')\delta(t - t'),
\end{equation}
with $\sigma_{ik}(\phi)\sigma_{jk}(\phi) = M_{ij}(\phi)$ (Einstein summation is implied).

\textit{{\bf Homework:}
Use the material presented in~\autoref{chap_thermo} to show that the choice~\eqref{eq_JS} ensures that the property of detailed balance is satisfied at the level of the dynamics of $\phi$.
Similarly to the Langevin equations satisfied by the particles degrees of freedom, Eq.~\eqref{eq_phi} is associated to a Fokker-Planck equation for the probability functional $\calP[\phi,t]$:
\begin{equation*}
\partial_t {\cal P}[\phi,t] = \intd{x} \frac{\delta}{\delta \phi}\left[ \nabla \cdot \left( {\cal P}[\phi,t] \bm J_{\rm D} - k_B T \bm M(\phi) \cdot \nabla \frac{\delta}{\delta \phi} {\cal P}[\phi,t]  \right) \right] .
\end{equation*}
Then, show that Eqs.~(\ref{eq_JD},\ref{eq_JS}) imply that the corresponding stationary distribution takes the Boltzmann form: ${\cal P}_{\rm s}[\phi] = \exp\left(-\tfrac{{\cal F}[\phi]}{k_B T} \right)$. 
}


The stochastic contribution to Eq.~\eqref{eq_phi} is usually written to study dynamical effects due to fluctuations (such as the roughening of interfaces, nucleation processes or to characterize the properties of the critical point...). Here, we will focus on the mean field properties of the models and thus drop the contribution from the noise ($\bm J_{\rm S} = \bm 0$).
Moreover, for simplicity we will restrict the presentation to the case of a scalar mobility $M(\phi) > 0$.

Although we keep the expression of the free energy density $f(\phi)$ general below,
its expression of course depends on the model of interest. 
Below, we describe two which are often considered: 
\begin{itemize}
\item {\bf The Flory-Huggins theory of polymer solutions.} Considering polymers in a solvent with respective volume fractions $\phi = v \rho$ and $\phi_{\rm sol} = v_{\rm sol} \rho_{\rm sol}$, 
the incompressibility of the solution implies that $\phi + \phi_{\rm sol} = 1$.
The Flory-Huggins free energy is then given by
$f_{\rm FH}(\phi) = k_B T \left[ \tfrac{1}{v} \phi \ln(\phi) + \tfrac{1}{v_{\rm sol}} (1-\phi)\ln(1-\phi) \right] + \chi \phi(1-\phi)$ where the first term accounts for the entropy contribution due to mixing of the polymer in the solvent, while the second term ($\propto \chi$) accounts for interactions. In the case where the latter are attractive, $\chi < 0$. For more details see e.g.~\cite{Eisele1990}. 
\item {\bf Cahn-Hilliard (Landau).} In general, the free energy can expanded in powers of $\phi$: $f_{\rm CH}(\phi) = \sum_i \tfrac{a_k}{k}\phi^k$ with the $\{a_k\}$ real coefficients allowed by the symmetries of the problem. 
Such expansion is generally considered as formally valid close to a critical point where $\phi$ is small (typically $\phi = (\rho - \rho_c)/\rho_c$ with $\rho_c$ the value of the density at the critical point) such that one usually truncates it at order $k = 4$.
Note that $f_{\rm CH}$ can be obtained from $f_{\rm FH}$ by expanding the logarithms.
\end{itemize}



\noindent {\it Spinodal decomposition} First, we investigate the condition for a homogeneous configuration with $\phi = \bphi$ to be stable to small perturbations.
Writing $\phi(\bm r,t) = \bphi + \delta \phi(\bm r,t)$, the perturbation $\delta \phi(\bm r,t)$ obeys from~\eqref{eq_phi} at linear order
\begin{equation} \label{eq_linear_phi}
\partial_t \delta \phi(\bm r,t) = \bar{M} \nabla^2 \left[ \left( \bar{f}'' - \bar{\kappa} \nabla^2\right)\delta \phi(\bm r,t)\right],
\end{equation}
where the bars stand for quantities evaluated at $\bphi$. 
Going into Fourier space, the growth rate associated to the mode with wavenumber $q$ is therefore $\lambda_q = -\bar{M} q^2(\bar{f}'' + \bar{\kappa} q^2)$.
As $\bar{M}$ and $\bar{\kappa}$ are both positive, the homogeneous configuration becomes unstable whenever $\bar{f}'' < 0$. 
This condition defines the so-called spinodal densities in the $(\bphi, k_B T$) phase diagram (Fig.~\ref{figeq}). 
In the region enclosed by the spinodals, any perturbation of homogeneous configurations grows exponentially fast, it is said that the system undergoes spinodal decomposition.

\textit{
{\bf Homework:}
Show that the emergence of phase-separation at equilibrium requires the presence of attractive interactions.
Hint: you can show this by considering the free energy ${\cal F}[\phi] = k_B T\intd{r} \phi (\ln(\phi) - 1) + \intd{r} \chi \phi^2(\bm r)$.
Propose an interpretation for the two integrals contributing to $\cal F$.
}

%%%%%%%%%%%%%%%%%%%%%%%%%%%%
\begin{figure}[b!]
	\includegraphics[width=\textwidth]{Figures/equilibrium_ps.pdf}
	\caption{Left: Typical Landau free energy landscapes above and below the critical point. 
	Center: phase diagram in the plane spanned by density and temperature showing the spinodals, binodals, and the critical point. 
	Right: Common tangent construction allowing to determine the coexistence densities in the phase separated regime.}
	\label{figeq}
\end{figure}
%%%%%%%%%%%%%%%%%%%%%%%%%%%%

\noindent {\it Binodal densities and the composition of phase separated configurations} 
The linear stability analysis does not tell us much about the state of the system long after the instability arises.
After the system undergoes spinodal decomposition, it goes through a coarsening regime until it phase separates into coexisting domains of different volumes and densities.
To determine the coexisting densities, we use the conservations of the total volume ${\cal V}$ and of the field $\phi$. 
We moreover work in the thermodynamic limit ${\cal V} \to \infty$ such that contributions from the interfaces can be safely neglected and we only consider the bulk contribution $f(\phi)$.
In a stationary phase-separated regime where the system is partitioned into two domains with $\phi = \phi_{1,2}$ and of respective volumes ${\cal V}_{1,2}$, 
the free energy~\eqref{eq_F} takes the trivial form
\begin{equation} \label{eq_F_ps}
{\cal F} = f(\phi_1) {\cal V}_1 + f(\phi_2) {\cal V}_2.
\end{equation} 
Denoting the mean value of $\phi$ as $\bphi$, the conservations of the total mass and volume impose
\begin{equation} \label{eq_constraints_binodals}
\phi_1 {\cal V}_1 + \phi_2 {\cal V}_2 = \bphi {\cal V}, \qquad  {\cal V}_1 + {\cal V}_2 = {\cal V} .
\end{equation} 
To determine the values $\phi_1$ and $\phi_2$, we minimize the free energy~\eqref{eq_F_ps} imposing~\eqref{eq_constraints_binodals}.
This is done by adding Lagrange multipliers ($\mu,P$) to ${\cal F}$, such that we define $\tilde{\cal F} = {\cal F} + P({\cal V}_1 + {\cal V}_2) - \mu(\phi_1 {\cal V}_1 + \phi_2 {\cal V}_2)$.
Minimizing $\tilde{\cal F}$ wrt the values of density and volume of each of the two phases, we get
\begin{align} \label{eq_binodal_mu}
\frac{\partial \tilde{\cal F}}{\partial \phi_i}  & = {\cal V}_i (f'(\phi_i) - \mu) = 0 , \\
\label{eq_binodal_P}
\frac{\partial \tilde{\cal F}}{\partial {\cal V}_i}  & = f(\phi_i) - \mu \phi_i + P = 0 ,
\end{align}
with $i = 1,2$.
Eq.~\eqref{eq_binodal_mu} imposes that the chemical potential $\mu = f'(\phi_1) = f'(\phi_2)$ takes identical values in both phases (\emph{diffusive equilibrium}),
while Eq.~\eqref{eq_binodal_P} ensures the equality of pressures (\emph{mechanical equilibrium}).
Therefore, one can determine graphically the values of $\phi_1$ and $\phi_2$ from a common tangent construction 
on the free energy landscape. 
Indeed, the equality of chemical potentials imposes equal slopes of $f$ at $\phi_1$ and $\phi_2$, 
while equality of pressures imposes a common intercept on the vertical axis as shown in Fig.~\ref{figeq}

Varying, e.g.\, the system temperature, the coexistence values $\phi_{1,2}$ define the \emph{binodal curves},
and meet the spinodals we determined previously at the critical point. 
As shown in Fig.~\ref{figeq}, the spinodals and binodals generally do not coincide, such that there are regions of the phase diagram 
where phase separated configurations exist (and are stable) but the homogeneous state at $\phi = \bphi$ is also linearly stable.
In practice, if we now put back noise into the picture in these regions the homogeneous state will typically disappear 
not through the deterministic growth of an infinitesimal perturbation, 
but because of stochastic nucleation events which correspond to large perturbations and are thus captured by the linear stability analysis.
Within the binodal region, the system thus always phase separates over long times into two distinct domains 
where $\phi$ takes values $\phi_{1,2}$ and whose volumes linearly interpolate between 0 and ${\cal V}$ for $\bphi \in [\phi_1;\phi_2]$ due to the condition~\eqref{eq_constraints_binodals}, 
namely
\begin{equation}
{\cal V}_1 = \frac{\phi_2 - \bphi}{\phi_2 - \phi_1} {\cal V}, \qquad {\cal V}_2 = {\cal V} - {\cal V}_1 = \frac{\bphi - \phi_1}{\phi_2 - \phi_1} {\cal V},
\end{equation} 
which is known as the \emph{lever rule}.\\

\noindent {\it The coarsening dynamics} So far, we have discussed the linear instability of homogeneous solutions and the relative composition of phase-separated states.
Of course, another interesting aspect of phase separation concerns how one moves from one to the other. 
Without entering into details (a relevant review paper on the topic is~\cite{Bray1994}), 
the coarsening process leading to phase separation can be understood by taking into account finite size effects, 
i.e.\ by considering the nonlocal contributions to the free energy which we have neglected above.
Doing this, one finds that the pressure inside a spherical droplet increases with its interface curvature, 
leading to a diffusive flux from small to large droplets. 
Over long times, small droplets thus typically shrink at the expense of larger ones, which is known as \emph{Ostwald ripening}.
One can moreover show that under this process the mean droplet radius grows in a universal manner as $\sim t^{1/3}$,
so that the asymptotic ($t \to \infty$) state inevitably consists of two macroscopic phase separated domains. 
























    \chapter{Coarse-graining of the mean field active Brownian particles model}
    \label{appendxi: BBKGY}

To begin with, we denote $N$ the total particle number and $P_N(\{r_i,\theta_i\},t)$ the $N$-body particle distribution.
From standard stochastic calculus, $P_N$ obeys the Fokker-Planck equation
\begin{equation} \label{eq_FPN}
    \partial_t P_N = \sum_{i=1}^{N} \nabla_{\bm r_i}\cdot \left[ \left( M \nabla_{\bm r_i}U - v_0 \hat{\bm e}(\theta_i) + D\nabla_{\bm r_i} \right)P_N\right] + D_r \sum_{i=1}{N} \partial_{\theta_i\theta_i}^2 P_N ,
\end{equation}
where, to lighten notations, we make the dependencies of the distribution in the degrees of freedom and time implicit.

Eq.~\eqref{eq_FPN} is exact. However it is of limited practical use as we went from a description of the system in terms of $3N$ microscopic degrees of freedom to a partial differential equation (PDE) for a distribution with $3N$ independent variables. 
Our goal now is to derive a simpler description of the system, essentially by integrating out `fast' processes which do not affect the dynamics over large scales.
To achieve this, we now consider the single-body distribution
\begin{equation} \label{eq_FP_P1}
    P(\bm r,\theta,t) = N \int \prod_{j=2}^N [\rmd^2 \bm r_j \rmd \theta_j] P_N(\{\bm r,\theta,\bm r_2,\theta_2,\ldots,\bm r_N,\theta_N\},t),
\end{equation}
obtained by integrating $P_N$ over $3(N-1)$ degrees of freedom and where the $N$ factor on the rhs accounts for the fact that the particles are indistinguishable.  
Note that here for simplicity we have dropped the `1' subscript. 
Integrating Eq.~\eqref{eq_FPN} over $3(N-1)$ degrees of freedom, we find that $P$ obeys
\begin{equation} \label{eq_FP1}
    \partial_t P = \nabla \cdot \left[ - \bm J_{\rm eff} - v_0 \hat{\bm e}(\theta) P + D\nabla P \right] + D_r \partial_{\theta\theta}^2 P ,
\end{equation}
where the effective flux coming from the interaction term reads
\begin{align*}
    \bm J_{\rm eff} & = -M N \int \prod_{j=2}^N [\rmd^2 \bm r_j \rmd \theta_j] \, P_N(\{\bm r,\theta,\ldots,\bm r_N,\theta_N\},t) \sum_{j \ne 1} \nabla u(|\bm r - \bm r_j|) \\
    & = -M \int \rmd^2 \bm r' \, \tilde{P}_2(\bm r,\theta,\bm r',t) \nabla u(|\bm r - \bm r'|) \\
    & = -M \int \rmd^2 \bm r' \, \tilde{P}_2(\bm r,\theta,\bm r',t) u'(|\bm r - \bm r'|) \frac{\bm r - \bm r'}{|\bm r - \bm r'|} ,
\end{align*}
where 
\begin{equation}
    \tilde{P}_2(\bm r,\theta,\bm r',t) = N(N-1) \int \rmd\theta' \prod_{j=3}^N [\rmd^2 \bm r_j \rmd \theta_j] P_N(\{\bm r,\theta,\bm r',\theta',\ldots,\bm r_N,\theta_N\},t) ,
\end{equation}
denotes the two body probability density to find a particle at position $\bm r$ with an orientation $\theta$ while another particle is at position $\bm r'$ with an arbitrary orientation. 
    
Due to the pairwise interactions, the equation for $P$ is thus coupled to the two body particle distribution. Similarly, if we had derived the equation for $\tilde{P}_2$ we would have found that it depends on the three body distribution, etc. 
This structure commonly appears when one coarse-grains interacting systems, and is known as the \emph{BBGKY hierarchy}, where the letters stand for Bogoliubov–Born–Green–Kirkwood–Yvon. To proceed further, we thus need to truncate the BBGKY hierarchy in order to get a closed equation for $P$. This can be done by means of various approximation schemes, the most common one being the molecular chaos hypothesis (or `Stosszahlansatz' for German speakers) which can be used to derive e.g.\ the Boltzmann equation for dilute gases. The molecular chaos assumption consists in factorizing the two body particle distribution into the product of two single body distributions.    
This amounts to assume that the positions and/or velocities of two particles on average decorrelate between collisions. This approximation works decently at low particle density, but of course becomes increasingly worse at higher densities.
Here, we therefore write
\begin{equation} \label{eq_P2}
    \tilde{P}_2(\bm r,\theta,\bm r',t) = P(\bm r,\theta,t) \, \phi(\bm r',t) \, g(|\bm r - \bm r'|,\varphi|\bm r,\theta,t),
\end{equation}
where $\phi$ denotes the particle density field, while $g(|\bm r - \bm r'|,\varphi|\bm r,\theta,t)$ 
is the conditional probability to find another 
particle at position $\bm r'$ given that there is another particle at position $\bm r$ self propelling along $\theta$, the angle $\varphi$ being that formed by the vectors $\bm r' - \bm r$ and $\hat{\bm e}(\theta)$.

Setting $g=1$ in Eq.~\eqref{eq_P2} corresponds to the molecular chaos assumption. 
Getting an explicit expression for $g$, however, would require to consider the dynamics of the two body distribution. In general, such approach is quite tedious and we won't pursue it here.
For quantitative descriptions, a simpler approach is to measure $g$ directly from simulations of the microscopic dynamics~\eqref{eq_ABP_micro}. 
Doing this in a dilute suspension, one finds that $g$ is generally maximal 
at $\varphi = 0$, i.e.\ there is more chance for a tagged particle to find another one at its front (where here `front' and `back' are defined wrt the particle polarity $\hat{\bm e}(\theta)$) than behind it~\cite{Bialk_2013}.

Using~\eqref{eq_P2}, we now rewrite the effective current as
\begin{align}
    \bm J_{\rm eff} & = -M P(\bm r,\theta,t) \int \rmd^2 \bm r' \, \phi(\bm r',t) g(|\bm r - \bm r'|,\varphi|\bm r,\theta,t) u'(|\bm r - \bm r'|) \frac{\bm r - \bm r'}{|\bm r - \bm r'|} , \\
    & = M P(\bm r,\theta,t) \int_0^\infty s \rmd s \int_0^{2\pi} \rmd\varphi \, \phi(\bm r + s\hat{\bm e}(\theta + \varphi),t) g(s,\varphi|\bm r,\theta,t) u'(s) \hat{\bm e}(\theta + \varphi) ,
    \end{align}
where $\bm s = \bm r' - \bm r = s \hat{\bm e}(\theta + \varphi)$.
Assuming that $u$ is sufficiently short-ranged such that we can neglect variations of $\phi$ over the length scale of the interaction, while supposing that $g$ is homogeneous and stationary: $g(s,\varphi|\bm r,\theta,t) \simeq g(s,\varphi)$,
we moreover get that $\bm J_{\rm eff} = M P(\bm r,\theta,t) \bm F_{\rm eff}(\bm r,\theta,t)$ with
\begin{equation}
    \bm F_{\rm eff}(\bm r,\theta,t) = -\phi(\bm r,t) \bm \zeta(\theta) = \phi(\bm r,t) \int_0^\infty s \rmd s \, u'(s) \int_0^{2\pi} \rmd \varphi \,  g(s,\varphi) \bm \hat{\bm e}(\varphi+\theta)
\end{equation}
Given the other terms in the positional current on the rhs of Eq.~\eqref{eq_FP1}, we now express $\bm \zeta(\theta)$ in the (nonorthogonal) basis formed by $\hat{\bm e}(\theta)$ and $\nabla P$.
Keeping terms up to ${\cal O}(|\nabla P|^2)$, we get
\begin{equation}
    \bm \zeta(\theta) = \zeta_\| \hat{\bm e}(\theta) + \left( \frac{\bm \zeta \cdot \nabla P - \zeta_\| \hat{\bm e}(\theta) \cdot \nabla P}{|\nabla P|^2}  \right)\nabla P + {\cal O}(|\nabla P|^2),
\end{equation}
where the important parameter $\zeta_\|$ is given by
\begin{equation}
    \zeta_\| = -\int_0^\infty s \rmd s \, u'(s) \int_0^{2\pi} \rmd \varphi \,  g(s,\varphi) \cos(\varphi).
\end{equation}

Putting back the explicit expression of $\bm J_{\rm eff}$ into Eq.~\eqref{eq_FP_P1}, we finally end up with a drift-diffusion-type equation:
\begin{equation} \label{eq_FP_P}
    \partial_t P = -\nabla \cdot \left[ v_{\rm eff} \hat{\bm e}(\theta) P - D_{\rm eff}\nabla P \right] + D_r \partial_{\theta\theta}^2 P ,
\end{equation}
whose effective coefficients read
\begin{equation}
    v_{\rm eff} = v_0 - M \zeta_\| \phi, \qquad
    D_{\rm eff} = D + M \phi P \bm \zeta \cdot \left( \frac{ \nabla P - (\hat{\bm e}(\theta) \cdot \nabla P) \hat{\bm e}(\theta)}{|\nabla P|^2}  \right).
\end{equation}

Some comments are now in order:
\begin{itemize}
    \item Activity shows at the level of $g$ by its anisotropic behavior wrt the angle $\varphi$. Therefore, we find that activity renormalizes both the self propulsion velocity and active fluctuations. In a passive system, one would indeed get $\bm \zeta = 0$ by symmetry.
    \item For repulsive interactions between the particles $u'(r) < 0$. Moreover, we know that $g(r,\varphi)$ is maximum in the sector $-\tfrac{\pi}{2} \le \varphi \le \tfrac{\pi}{2}$ where $\cos(\varphi) \ge 0$, so one expects $\zeta_\| > 0$ in general. Therefore, our derivation reveals that the main effect of coupling self-propulsion with interactions is to slow down particles in dense regions ($v_{\rm eff} < v_0$ for $\phi > 0$). As one can already anticipate, this phenomenon amounts to having effective attractive interactions between the particles despite their absence at the microscopic level.
    Note that the linear decay of $v_{\rm eff}(\phi)$ is actually in agreement with numerical simulations of repelling ABPs~\cite{CatesMIPS}.
    \item In contrast, diffusivity is renormalized in a non-trivial way. In practice, this effect is weak which we can understand from Eq.~\eqref{eq_FP_P} by noting that in steady state $\nabla P$ and $\hat{\bm e}(\theta)$ must be aligned. In what follows, we will thus neglect corrections to the diffusivity and consider that it is given by a constant coefficient.
\end{itemize}


    \chapter{Active model B}
    \label{section: active model B top down}

In~\autoref{chapter: phase sep} we have briefly reviewed the physics of phase separation at equilibirum.
In~\autoref{chap_scalar}, however, we have seen that active phase separation generally leads to a density current with terms that cannot be derived from a free energy functional.
Namely, we found
\begin{align} \label{eq_neq_J}
\bm J[\rho] & = - \rho M(\rho)\nabla\mu_{\rm neq}(\rho) + \zeta(\rho)(\nabla^2\rho)\nabla\rho, \\
\mu_{\rm neq}(\rho) & = f'(\rho) - \kappa(\rho) \nabla^2\rho + \lambda(\rho)|\nabla\rho|^2 . \nonumber
\end{align}
In particular, when the relation $2\lambda(\rho) + \kappa'(\rho) = 0$ is not satisfied, $\mu_{\rm neq}$ cannot be written as $\delta {\cal F}/ \delta \rho$.
Below, we show how when setting $\zeta(\rho) = 0$ one can formally evaluate the binodal densities associated to a phase-separated configuration. 

As was demonstrated in~\cite{Solon2018}, an effective free energy structure can be recovered through a mapping of $\rho$ and ${\cal F}$ to generalized thermodynamic variables.
Namely, let us consider the one to one mapping $\rho\to\psi$ and the functional ${\cal G}$ such that $\mu_{\rm neq} = \delta{\cal G}/\delta\psi$.
Writing, ${\cal G} = \intd{r} [g(\rho) + \tfrac{1}{2}B(\rho)|\nabla\psi|^2]$, we get
\begin{equation}
\frac{\delta {\cal G}}{\delta \psi} = \frac{g'(\rho)}{\psi'(\rho)} +\frac{1}{2}\frac{B'(\rho)}{\psi'(\rho)}|\nabla\psi|^2 - \nabla \cdot ( B(\rho)\nabla\psi) ,
\end{equation}
where primes denote derivatives wrt $\rho$.
Using that $\nabla\psi = \psi'\nabla\rho$, we obtain after straightforward algebra
\begin{align}
\mu_{\rm neq} & = \frac{g'(\rho)}{\psi'(\rho)} - B(\rho) \psi'(\rho) \nabla^2\rho  - \left[\psi''(\rho) B(\rho) + \frac{1}{2} B'(\rho)\psi'(\rho)\right]|\nabla\rho|^2 \\
& = f'(\rho) - \kappa(\rho) \nabla^2\rho + \lambda(\rho) |\nabla\rho|^2. \nonumber
\end{align}
Equating the r.h.s.\ of the above expressions term by term, we thus get
\begin{equation}
g(\rho) = \int^\rho \rmd\tilde{\rho} \, f'(\tilde{\rho}) \psi'(\tilde{\rho}) , \qquad
B(\rho) = \frac{\kappa(\rho)}{\psi'(\rho)} , \qquad
\kappa(\rho)\psi''(\rho) = -[2\lambda(\rho) + \kappa'(\rho)]\psi'(\rho).
\end{equation}
It can easily be check that that in the case where $2\lambda(\rho) + \kappa'(\rho) = 0$, the third equality above implies that $\psi = \rho$ up to a constant term which can be set to zero without loss of generality.
From the above, the dynamics of $\rho$ can now be written as minimizing a free-energy-like functional:
\begin{equation}
\partial_t \rho = \nabla\cdot \left[ \rho M (\rho) \nabla\frac{\delta {\cal G}}{\delta\psi}\right] ,
\end{equation}
but where, in contrast with the equilibrium case, the minimization of ${\cal G}$ is performed over the auxiliary variable $\psi$ and not $\rho$ itself. 

To identify the generalized pressure, we now note that the dynamics of $\rho$ can be generally expressed in terms of the stress tensor $\bm T$:
\begin{align}
\partial_t \rho & = \nabla\cdot \left [\rho M(\rho) \nabla \frac{\delta {\cal F}}{\delta\rho} \right] = -\nabla\cdot\left[ M(\rho) \nabla \cdot \bm T\right] , \nonumber \\
\label{eq_stress}
T_{ij} & = \delta_{ij} \left[ F - \rho\frac{\delta {\cal F}}{\delta\rho} \right] - \frac{\partial F}{\partial (\partial_j \rho)}\partial_i \rho ,
\end{align}
with $F(\rho,\nabla\rho)$ defined from ${\cal F} = \int\dd \bm r F(\rho,\nabla\rho)$.

\textit{{\bf Homework:} Demonstrate Eq.~\eqref{eq_stress}.}\\

It appears clearly from Eq.~\eqref{eq_stress} that the local diagonal part of the stress tensor corresponds to the pressure as defined from Eq.~\eqref{eq_binodal_P},
while in general the nondiagonal part defines an anisotropic pressure.
Therefore, neglecting nonlocal contributions as in the previous section we conclude from the above mapping that the generalized pressure for active model B reads
\begin{equation} \label{eq_genP}
\Pi = \psi \mu  - g(\psi) =  \psi \frac{\rmd g}{\rmd \psi} - g(\psi) .
\end{equation}
Together, equalities of pressure and chemical potential between both phases define a common tangent construction in terms of the variables $\psi$ and $g$, which
allows to determine the binodal densities and consequently the volume of each phase. 
The resulting phase diagram for active model B is therefore similar to the equilibirum one shown in Fig.~\ref{figeq}.



    \bibliographystyle{ieeetr}
    \bibliography{ref.bib}

    \phantomsection
    \addcontentsline{toc}{section}{To do/Notes}
    
    %\setcounter{tocdepth}{1}
    %\listoftodos
    
    % \section*{Notes}
    % \begin{itemize}
    %     \item Should we use $\bm \nabla$, for consistent vector notation?
    % \end{itemize}

\end{document}
