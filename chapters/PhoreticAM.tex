Phoresis comes from Greek and means to be carried.
There are several types of phoresis, but common for all is that there is a gradient in some fields that carries particles around.
Some kinds of phoresis, and the corresponding gradients are
\begin{itemize}
    \item Diffusiophoresis, $\bm \nabla C$,
    \item Electrophoresis, $\bm \nabla \varphi = \bm E$,
    \item Thermophoresis, $\bm \nabla T$.
\end{itemize}
%
This gradient gives rise to a \emph{slip velocity} $\bm v_s$ along the surface of particles and allows for motion.

\todo[inline]{Draw diagrams}

\section{Diffusiophoresis}

We will now consider a particle of radius $R$ suspended in a liquid, surrounded by a chemical density $c$.
In this section, we will show that, given some criteria for the interaction of the particle with the chemical, this gives rise to a slip-velocity $\bm v_s$ of the fluid around the particle, which is given by
%
\begin{align}
    \bm v_s  = \mu \bm \nabla_\parallel c_{\mathrm{out}},
\end{align}
%
where $c_{\mathrm{out}}$ is the gradient surronding sphere, outside the slip region(?).
\todo[inline]{figure}

The slip region is a thin layer of width $\sigma \ll R$ around the sphere, where the chemical and fluid is interacting with the surface of the sphere.
\todo[inline]{figure}

We assume the particle interacts with the chemical through a potential of the form
%
\begin{align}
    W(\bm x) = 
    \begin{cases}
        0 , & z \geq \sigma \\
        W(z), & z < 0.
    \end{cases}
\end{align}
%
This potential must diverge as we approach the sphere, $W(z\rightarrow 0) = \infty$, as we model the particle as a hard sphere.

We assume the solute follow
%
\begin{align}
    \partial_t c &= \bm \cdot \bm J,\\
    \bm J &= - \bm \nabla c + \beta D C \bm F + c \bm v,
\end{align}
%
and the fluid is Stoksean,
%
\begin{align}
    - \eta \nabla^2 \bm v &= - \bm \nabla p + \bm f, \quad\quad \bm f = - c \bm \nabla W,\\
    \bm \nabla \cdot \bm v &= 0.
\end{align}
%
We will assume the chemical relaxes fast, so $\partial_t c = 0$.
Then we consider the flow in the $z$-direction, perpendicular to surface of the sphere, and the parallel flow seperatly.
This leaves the following equations,
%
\begin{subequations}
\begin{align}\label{eq c diffusion}
    D \nabla^2 c &= \beta D \bm \nabla \cdot (c \bm \nabla W) - \bm v \cdot \bm \nabla c,\\\label{eq vz}
    - \eta \nabla^2 v_z &= - \partial_z p - c \partial_z W, \\\label{eq v par}
    - \eta \nabla^2 \bm v_\parallel & = - \bm \nabla_\parallel p.
\end{align}    
\end{subequations}
%
We will further assume that the flow in the vertical direction is negligible, $v_z = 0$, and that the gradient is dominated by the $z$-direction, i.e., $\partial_z \gg |\bm \partial_\parallel c|$ and $\nabla^2 \approx \partial_z^2$.
This is correct to $\Oh((\sigma/R)^2)$.
With this, we have
%
\begin{align}
    \bm v &=
    \begin{pmatrix}
        v_x \\ v_y \\ 0
    \end{pmatrix},&
    \bm \nabla c
    &= 
    \begin{pmatrix}
        0 \\ 0 \\ \partial_z c
    \end{pmatrix}, &\implies &
    \bm \nabla \cdot\bm v = 0.
\end{align} 
%
With this, \autoref{eq c diffusion} becomes
%
\begin{align}
    \partial_z^2 c = \beta \partial_z \left(c \partial_z W\right),
\end{align}
%
Which we can solve exactly yielding the Boltzmann distribution,
%
\begin{align}
    c(\bm x) = c_\mathrm{out} e^{-\beta W(\bm x)}.
\end{align}
%
Next, \autoref{eq vz} becomes
%
\begin{align}
    \partial_z p =  c \partial_z W.
\end{align}
%
using the solution we found earlier, we have that $c \partial_z W = - \frac{1}{\beta}\partial_z c$, so we write
%
\begin{align}
    \partial_z\left(p - \frac{1}{\beta} c\right) = 0.
\end{align}
%
This means that the quantity $p - \frac{1}{\beta} c$ is constant within the slip layer, and we can apply the boundary condition to write
%
\begin{align}
    p(\bm x) 
    = p_\mathrm{out} 
    + k_B T c_\mathrm{out}(\bm x_\parallel) \left( e^{- \beta W(z)} - 1 \right).
\end{align}
%
Finally, \autoref{eq v par} becomes
%
\begin{align}
    \eta \partial_z \bm b = -\bm \nabla_\parallel p_\mathrm{out}
    - k_B T \left(\bm \nabla_\parallel c_\mathrm{out} \right)
    \left(e^{- \beta W}\right).
\end{align}
%
Outside the slip layer, the pressure is assumed constant, so $-\bm \nabla_\parallel p_\mathrm{out} = 0$.
We then apply $\int_{0}^\sigma z \, z$ to both parts, integrating by parts, to obtain
\todo[inline]{detail this}  
%
\begin{align}
    \bm v_z = \underbrace{\int \dd z \, z \left(1 - e^{-\beta w(z)}\right)}_{\mu}
    \bm \nabla_\parallel c_\mathrm{out},
\end{align}
%
as promised at the beginning of this section.

This calculation shows that, due to the interaction potential $W(z)$ between the particle and the solute concentration $c$, a gradient in $c$ gives rise to a flow along the surface of the particle.
This flow will exert a force on the particle.
In fact, we may use the reciprocal theorem, as discussed in \autoref{chap_hydro}, to calculate the resulting net velocity $\bm v$ and rotational rate $\omega$ of the particle.
\todo[inline]{details?}
They are given by integrals over the surface $s$ of the particles
%
\begin{align}
    \bm v & = 
    \frac{1}{4 \pi} \int \dd s \, \bm v_s
    = \frac{1}{4 \pi R^2} \in \dd s \,  \mu \bm \nabla_\parallel c_\mathrm{out}, 
    \Omega = - \frac{3}{8 \pi R^3} \int \dd s \, \hat {\bm e}_z \times \bm v_s.
\end{align}
%
As the simplest example, consider
\todo[inline]{write example}

\subsection*{Self-diffusiophoresis}

We will now consider what happens when a particle creates the gradient it is reacting to, at a rate $\alpha$.
This means the introduction of activity, as such a mechanism needs to consume energy.
Outside the slip layer, the chemical obeys the diffusion equation,
%
\begin{align}
    D_c \nabla^2 c = 0,
\end{align}
%
while at the boundary, the production of chemical by the particle results in a gradient, which imposes the boundary condition
%
\begin{align}
    - D_c \bm \nabla c \cdot \hat{\bm e}_r \big|_s = \alpha(\bm r_s).
\end{align}
%
We will consider the chemical diffuses a lot faster than the particle, $D_c \gg D$, and that \todo[noinline]{is this right?}
%
\begin{align}
    \mathrm{Pe} = \frac{R v}{D_c }\ll 1.
\end{align}
%
This results in the slip velocity
\todo[inline]{details}
%
\begin{align}
    \bm v_s = \mu\left( \one - \hat {\bm e}_r \otimes \hat {\bm e}_r \right) \bm \nabla c.
\end{align}
%

\subsection*{Mixture of $M$ spicies}

\todo[inline]{More on the jump from single particel to density?}

We now consider a mixture of $M$ different species with phoretic particles.
Particles of species $i$ each have the same mobility $\mu_i$ and activity $\alpha_i$, and the density of such particles is given by $\rho_i(\bm x, t)$.
These particles are thus active Brownian particles, but the self-propulsion velocity is now given not by the polarity $p_i$ and the inherent drive $v_0$, $\bm v = v_0 \bm p$, but rather the gradient and mobility $\bm v = - \mu \bm \nabla c$.
The equations of $\rho_i$ are therefore given by swapping the self propulsion in \autoref{eq: closed density},
%
\begin{align}
    \partial_t \rho_i 
    = - \bm \nabla \cdot \left[ (- \mu \bm \nabla c) \rho_i - D \bm \nabla \rho_i \right],
\end{align}
%
while the chemical follows
\todo[inline]{show this?}
%
\begin{align}
    \partial_t c = D_c \nabla^2 c + \sum_i \alpha_i \rho_i.
\end{align}
%
If we consider the perturbations $\delta \rho_i$ and $\delta c$ around homogenous solutions $\bar \rho_i$ and $c_0$, we see that
%
\begin{align}
    \bar c_0(t) = c_0 + t \sum_i \alpha_i \bar \rho_i
\end{align}
%
and, considering again the concentration fast so $0 = \partial_t\delta c$, we get
%
\begin{align}
    \nabla^2 \delta c = - \frac{1}{D_c} \sum_i \alpha_i \delta \rho_i.
\end{align}
%
For the density, we have
%
\begin{align}
    \partial_t \delta \rho_i 
    &= D \nabla^2 \delta \rho_i + \mu_i \bar \rho_i \nabla^2 \delta c\\
    & = D \nabla^2\delta \rho_i 
    - \mu_i \bar \rho_i \sum_j \frac{\alpha_j \delta \rho_j}{D_c}.
\end{align}
%
If we define
%
\begin{align}
    U_i &= \alpha_i \delta \rho_i & 
    \gamma_i = \frac{\alpha_i \mu_i \bar \rho_i}{D_c},
\end{align}
%
then, in Fourier space, this may be written as an eigenvalue equation,
%
\begin{align}
    \underbrace{\left[ \lambda + Dk^2 \right]}_{\tilde \lambda} \tilde U_i =
    - \gamma_i \sum_j \tilde U_j,
\end{align}
%
where $\tilde U_i(t) = e^{- \lambda t} \tilde U_i(t = 0)$.
In matrix option,
%
\begin{align}
    \begin{pmatrix}
        \gamma_1 & \cdots & \gamma_1 \\
        \vdots & \ddots & \vdots \\
        \gamma_M & \cdots & \gamma_M
    \end{pmatrix}
    \begin{pmatrix}
        \tilde U_1 \\ \vdots\\ \tilde U_M
    \end{pmatrix}
    = 
    \tilde \lambda 
    \begin{pmatrix}
        \tilde U_1 \\ \vdots \\\tilde U_M
    \end{pmatrix}.
\end{align}
%
All columns/rows of this matrix are linearly dependent on each other, so $M - 1$ of the eigenvalues vanish, $\tilde \lambda_- = 0 $.
The single non-vanishing eigenvalue is given by the trace, $\tilde \lambda_+ \equiv \Gamma = \sum_i \gamma_i$.
This eigenvalue gives the stability of the system.
The time evolution of a homogenous mixture is given by $\lambda$, and $\lambda_- = \tilde \lambda_- - D k^2 =  - Dk^2 \leq 0$, so these modes are exponentially supressed.
However, $\lambda_+ = -\Gamma - D k^2$ will be possitive for certain wavenumbers $k$ if $\Gamma< 0$.
\todo[noinline]{Draw diagram of $\lambda$}.
This criterion of in-stability translates to 
%
\begin{align}
    \sum_i \alpha_i \mu_i \bar \rho _i  < 0.
\end{align}
%


As a first example, consider $M=1$.
The criterion for instability is then $\alpha \mu < 0$.
This is called the Keller-Segel instability, and is the result of particles producing a chemical they are attracted to, and thus creating an effective, long-range attractive interaction.

In the case of $M = 2$, we can show that the lead order growth terms of the different species is given by
%
\begin{align}
    U_i \sim \frac{\gamma_i}{\Gamma} e^{-\Gamma t} + \dots
\end{align}
%
Thus, the relative stoichiometry in the domains are given by
%
\begin{align}
    \frac{\delta \rho_i}{\delta \rho_1} = \frac{\mu_i \bar \rho_i}{\mu_1 \bar \rho_1},
\end{align}
%
or
%
\begin{align}
    \delta \rho_2 = \frac{\mu_2 \bar \rho_2}{\mu_1 \bar \rho_1} \delta \rho_1.
\end{align}
%



\section{Polar phoretic active matter}

We now consider the case where the activity of the particle varies acros the surface.
The simplest case is the Janus-colloid, in which the two halves of the particle has two different mobilities $\mu_1, \mu_2$ and activities, $\alpha_1, \alpha_2$.
We will solve for the resulting concentration field, given a isolated particle.
The equations of the field are the
%
\begin{align}
    D\nabla^2 c & = 0\\
    -D \bm \nabla c \cdot \hat{ \bm e}_r|_s &= \alpha(\theta)\\
    \lim_{\bm x \rightarrow \infty} c(\bm x) &= 0.
\end{align}
%
We will assume a axisymmetric setup, so $\alpha$ only depends on $\theta$, and $\partial_\phi c = 0$.
In spherical coordinates, the laplace equation is then
%
\begin{align}
    \partial_r \left( r^2 \partial_r c \right) + \frac{1}{\sin \theta} \partial_\theta \left( \sin\theta \partial_\theta c \right) = 0.
\end{align}
%
By assuming separation of variables, $c(r, \theta) = f(r) \Phi(\theta)$, we can write this as
%
\begin{align}
    \frac{1}{f(r)} \partial_r f(r) r^2 \left[\partial_r f(r)\right] = 
    - \frac{1}{\Phi(\theta)\sin\theta} \partial_\theta \left( \sin\theta \partial_\theta \Phi(\theta) \right)  = \lambda.
\end{align}
%
We begin with the $\theta$ equation, defining $z = \cos\theta \in [-1, 1]$, and $Z(z) = \Phi(\arcsin z)$ which gives
%
\begin{align}
    \odv{}{z}   \left[ (1 - z^2) \odv{}{z} Z(z) \right] = - \lambda Z(z).
\end{align}
%
This is solved by Legandre polynomials, $Z(z) = P_m(z)$, with the eigenvalue $\lambda = m(m+1)$.
These have the orthogonality property
\begin{align}
    \int \dd z \, P_m(z)P_\ell(z) = \frac{2}{2m + 1} \delta_{m\ell}.
\end{align}
%
In terms of $\theta$, we thus have $\Phi(\theta) = P_m(\cos\theta)$.

The eigenvalue problem for the radial function is now
\begin{align}
    r^2\partial_r^2 f(r) + z r \partial_r  f(r)  -m(m+1) f(r) = 0.
\end{align}
%
This equation is homogneous, that is, invariant under $r\rightarrow c r$ for a constant $c$. 
This means that the solution has the form $f = r^{-\alpha}$, where we must have $\alpha>0$ to satisfy the vanishing boundary condition at infinity.
Inserting this ansatz, we find that $\alpha = m + 1$.
The full solution is a superposition of the eigenmode, so
%
\begin{align}
    c(r, \theta) = \sum_m c_m r^{-(m + 1)} P_m(\cos\theta).
\end{align}
%
We now apply the boundary condition, which give
\begin{align}
    \alpha(\theta) 
    = - \hat {\bm e}_r(\theta) \cdot \bm \nabla c(r)
    = D \sum_m c_m (m + 1) r^{-(m + 1)} P_m(\cos\theta).
\end{align}
%
We decompose the activity into lagandre polynomials as well, $\alpha(\theta) = \sum_m \alpha_m P_m(\cos\theta)$, which means that we may find $c_m$ by integrating and using the orthogonality property to find the coefficients $c_m$.

\todo[inline]{fill out rest}


\subsection*{Multi-particle system}

We now consider a system of many particles, each governed by a Langevin equation, as we did in \autoref{chap_scalar}.
However, we will now consider three dimensions, which means the equations are slightly different.
They take the form
\begin{align}
    \odv{}{t} \bm r_i &= v_i(c) + \sqrt{2D } \bm \xi,\\
    \odv{}{t} \hat{\bm e} & = \bm \omega(c) + \sqrt{2D} \hat{\bm e} \xi_r.
\end{align}
%
The unit vector $\hat {\bm n}$ now lives on a sphere, instead of the circle as in 2D.
The probability density is then
\begin{align}
    \calP(\bm x, \hat{\bm n}, t)
    = \E{ \delta(\bm r(t) - \bm x ) \delta(\hat{\bm n}(t) - \hat{\bm e}) },
\end{align}
%
and it is governed by the following Fokker-Planck equation,
%
\begin{align}
    \partial_t \calP = - \bm \nabla \cdot \bm J - \mathcal{R} \cdot \bm J_R.
\end{align}
%
Here, $\mathcal{R} = \hat{\bm n} \times \partial_{\hat{\bm n}}$ is the generators of rotation, i.e. translations on the sphere, in the same way that the nabla-operator $\bm \nabla$ is the generator of translations in space.
This operators behaves as a derivative, for example, one may perform partial integration.
The currents are
\begin{align}
    \bm J & = - D \bm \nabla \calP + \bm v(c) \calP,\\ 
    \bm J_r &= - D \mathcal{R} P + \bm \omega(c) \calP.
\end{align}
%
We are now, as in \autoref{chap_scalar}, going to perform a moment-expansion, to find the quations of $\rho$, $\bm p$ and $Q$.
\todo{Write down definitions in 3D}

The equation for the density is
\todo[inline]{Write mom-expansion}
