\label{section: active model B top down}

In~\autoref{chapter: phase sep} we have briefly reviewed the physics of phase separation at equilibirum.
In~\autoref{chap_scalar}, however, we have seen that active phase separation generally leads to a density current with terms that cannot be derived from a free energy functional.
Namely, we found
\begin{align} \label{eq_neq_J}
\bm J[\rho] & = - \rho M(\rho)\nabla\mu_{\rm neq}(\rho) + \zeta(\rho)(\nabla^2\rho)\nabla\rho, \\
\mu_{\rm neq}(\rho) & = f'(\rho) - \kappa(\rho) \nabla^2\rho + \lambda(\rho)|\nabla\rho|^2 . \nonumber
\end{align}
In particular, when the relation $2\lambda(\rho) + \kappa'(\rho) = 0$ is not satisfied, $\mu_{\rm neq}$ cannot be written as $\delta {\cal F}/ \delta \rho$.
Below, we show how when setting $\zeta(\rho) = 0$ one can formally evaluate the binodal densities associated to a phase-separated configuration. 

As was demonstrated in~\cite{Solon2018}, an effective free energy structure can be recovered through a mapping of $\rho$ and ${\cal F}$ to generalized thermodynamic variables.
Namely, let us consider the one to one mapping $\rho\to\psi$ and the functional ${\cal G}$ such that $\mu_{\rm neq} = \delta{\cal G}/\delta\psi$.
Writing, ${\cal G} = \intd{r} [g(\rho) + \tfrac{1}{2}B(\rho)|\nabla\psi|^2]$, we get
\begin{equation}
\frac{\delta {\cal G}}{\delta \psi} = \frac{g'(\rho)}{\psi'(\rho)} +\frac{1}{2}\frac{B'(\rho)}{\psi'(\rho)}|\nabla\psi|^2 - \nabla \cdot ( B(\rho)\nabla\psi) ,
\end{equation}
where primes denote derivatives wrt $\rho$.
Using that $\nabla\psi = \psi'\nabla\rho$, we obtain after straightforward algebra
\begin{align}
\mu_{\rm neq} & = \frac{g'(\rho)}{\psi'(\rho)} - B(\rho) \psi'(\rho) \nabla^2\rho  - \left[\psi''(\rho) B(\rho) + \frac{1}{2} B'(\rho)\psi'(\rho)\right]|\nabla\rho|^2 \\
& = f'(\rho) - \kappa(\rho) \nabla^2\rho + \lambda(\rho) |\nabla\rho|^2. \nonumber
\end{align}
Equating the r.h.s.\ of the above expressions term by term, we thus get
\begin{equation}
g(\rho) = \int^\rho \rmd\tilde{\rho} \, f'(\tilde{\rho}) \psi'(\tilde{\rho}) , \qquad
B(\rho) = \frac{\kappa(\rho)}{\psi'(\rho)} , \qquad
\kappa(\rho)\psi''(\rho) = -[2\lambda(\rho) + \kappa'(\rho)]\psi'(\rho).
\end{equation}
It can easily be check that that in the case where $2\lambda(\rho) + \kappa'(\rho) = 0$, the third equality above implies that $\psi = \rho$ up to a constant term which can be set to zero without loss of generality.
From the above, the dynamics of $\rho$ can now be written as minimizing a free-energy-like functional:
\begin{equation}
\partial_t \rho = \nabla\cdot \left[ \rho M (\rho) \nabla\frac{\delta {\cal G}}{\delta\psi}\right] ,
\end{equation}
but where, in contrast with the equilibrium case, the minimization of ${\cal G}$ is performed over the auxiliary variable $\psi$ and not $\rho$ itself. 

To identify the generalized pressure, we now note that the dynamics of $\rho$ can be generally expressed in terms of the stress tensor $\bm T$:
\begin{align}
\partial_t \rho & = \nabla\cdot \left [\rho M(\rho) \nabla \frac{\delta {\cal F}}{\delta\rho} \right] = -\nabla\cdot\left[ M(\rho) \nabla \cdot \bm T\right] , \nonumber \\
\label{eq_stress}
T_{ij} & = \delta_{ij} \left[ F - \rho\frac{\delta {\cal F}}{\delta\rho} \right] - \frac{\partial F}{\partial (\partial_j \rho)}\partial_i \rho ,
\end{align}
with $F(\rho,\nabla\rho)$ defined from ${\cal F} = \int\dd \bm r F(\rho,\nabla\rho)$.

\textit{{\bf Homework:} Demonstrate Eq.~\eqref{eq_stress}.}\\

It appears clearly from Eq.~\eqref{eq_stress} that the local diagonal part of the stress tensor corresponds to the pressure as defined from Eq.~\eqref{eq_binodal_P},
while in general the nondiagonal part defines an anisotropic pressure.
Therefore, neglecting nonlocal contributions as in the previous section we conclude from the above mapping that the generalized pressure for active model B reads
\begin{equation} \label{eq_genP}
\Pi = \psi \mu  - g(\psi) =  \psi \frac{\rmd g}{\rmd \psi} - g(\psi) .
\end{equation}
Together, equalities of pressure and chemical potential between both phases define a common tangent construction in terms of the variables $\psi$ and $g$, which
allows to determine the binodal densities and consequently the volume of each phase. 
The resulting phase diagram for active model B is therefore similar to the equilibirum one shown in Fig.~\ref{figeq}.

