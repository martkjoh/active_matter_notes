\label{section: active model B top down}

In the previous section we have briefly reviewed the physics of phase separation at equilibirum using the minimal formulation offered by model B.
A general question now is how to generalize model B to the case where the microscopic dynamics of the system is active, 
i.e.\ breaks time-reversal symmetry at microscopic scales.
As we want to describe scalar active systems, we can reasonably conjecture (and we will show it later) that the dynamics of their density field should obey an equation similar to~\eqref{eq_phi}. 
Keeping a vanishing stochastic current, 
a straightforward way to break the equilibrium structure of Eq.~\eqref{eq_phi} consists in adding to the chemical potential in Eq.~\eqref{eq_JD} a contribution that cannot be derived from a free energy.
Considering reasonably smooth functions, we immediately see that any nonequilibrium contribution to $\mu$ must come from the nonlocal (proportional to the gradients of $\phi$) terms.
We can thus write in general
\begin{equation} \label{eq_neq_mu}
\mu(\phi) = f'(\phi) - \kappa(\phi) \nabla^2\phi + \lambda(\phi)|\nabla\phi|^2 ,
\end{equation}
such that when the relation $2\lambda(\phi) + \kappa'(\phi) \ne 0$ is not satisfied, $\mu$ cannot be written as $\delta {\cal F}/ \delta \phi$.
While only the local term $f'(\phi)$ in~\eqref{eq_neq_mu} affects the evaluation of the spinodals, 
the absence of free energy structure prevents us to use the powerful formalism presented above to determine the binodals and thus the phase separated configurations.

Fortunately, as was demonstrated in~\cite{Solon2018} an effective free energy structure can be recovered through a mapping of $\phi$ and ${\cal F}$ to generalized thermodynamic variables.
Namely, let us consider the one to one mapping $\phi\to\psi$ and the functional ${\cal G}$ such that $\mu = \delta{\cal G}/\delta\psi$.
Writing, ${\cal G} = \intd{r} [g(\phi) + \tfrac{1}{2}B(\phi)|\nabla\psi|^2]$, we get
\begin{equation}
\frac{\delta {\cal G}}{\delta \psi} = \frac{g'(\phi)}{\psi'(\phi)} +\frac{1}{2}\frac{B'(\phi)}{\psi'(\phi)}|\nabla\psi|^2 - \nabla \cdot ( B(\phi)\nabla\psi) ,
\end{equation}
where primes denote derivatives wrt $\phi$.
Using that $\nabla\psi = \psi'\nabla\phi$, we obtain after straightforward algebra
\begin{align}
\mu & = \frac{g'(\phi)}{\psi'(\phi)} - B(\phi) \psi'(\phi) \nabla^2\phi  - \left[\psi''(\phi) B(\phi) + \frac{1}{2} B'(\phi)\psi'(\phi)\right]|\nabla\phi|^2 \\
& = f'(\phi) - \kappa(\phi) \nabla^2\phi + \lambda(\phi) |\nabla\phi|^2. \nonumber
\end{align}
Equating the rhs of the above expressions term by term, we thus get
\begin{equation}
g(\phi) = \int^\phi \rmd\tilde{\phi} \, f'(\tilde{\phi}) \psi'(\tilde{\phi}) , \qquad
B(\phi) = \frac{\kappa(\phi)}{\psi'(\phi)} , \qquad
\kappa(\phi)\psi''(\phi) = -[2\lambda(\phi) + \kappa'(\phi)]\psi'(\phi).
\end{equation}
You can check that that in the case $2\lambda(\phi) + \kappa'(\phi) = 0$ the third equality above implies without much surprise that $\psi = \phi$ up to a constant term which can be set to zero without loss of generality.
From the above, the dynamics of $\phi$ can now be written as minimizing a free-energy-like functional:
\begin{equation}
\partial_t \phi = \nabla\cdot \left[ M (\phi) \nabla\frac{\delta {\cal G}}{\delta\psi}\right] ,
\end{equation}
but where, in contrast with the equilibrium case, the minimization of ${\cal G}$ is performed over the auxiliary variable $\psi$ and not $\phi$ itself. 

\noindent {\it Exercise: Derive the equation governing the dynamics of $\psi$, what is the essential difference between it and the equilibrium model B?}\\

To identify the generalized pressure, we now note that the dynamics of $\phi$ can be generally expressed in terms of the stress tensor $\bm T$:
\begin{align}
\partial_t \phi & = \nabla\cdot \left [M(\phi) \nabla \frac{\delta {\cal F}}{\delta\phi} \right] = -\nabla\cdot\left[ \frac{M(\phi)}{\phi} \nabla \cdot \bm T\right] , \nonumber \\
\label{eq_stress}
T_{ij} & = \delta_{ij} \left[ F - \phi\frac{\delta {\cal F}}{\delta\phi} \right] - \frac{\partial F}{\partial (\partial_j \phi)}\partial_i \phi ,
\end{align}
with $F(\phi,\nabla\phi)$ defined from ${\cal F} = \intd{r} F$.

\noindent{\it Exercise: Demonstrate Eq.~\eqref{eq_stress}.}\\

It appears clearly from Eq.~\eqref{eq_stress} that the local diagonal part of the stress tensor corresponds to the pressure as defined from Eq.~\eqref{eq_binodal_P},
while in general the nondiagonal part defines an anisotropic pressure.
Therefore, neglecting nonlocal contributions as in the previous section we conclude from the above mapping that the generalized pressure for the active model B reads
\begin{equation} \label{eq_genP}
\Pi = \psi \mu  - g(\psi) =  \psi \frac{\rmd g}{\rmd \psi} - g(\psi) .
\end{equation}
Together, equalities of pressure and chemical potential between both phases therefore define a common tangent construction in terms of the variables $\psi$ and $g$ which
allows to determine the binodal densities and consequently the volume of each phase. 
The resulting phase diagram for active model B is therefore similar to the equilibirum one shown in Fig.~\ref{figeq}.

Similarly to equilibrium, taking into account finite size effects one can show that in this case the generalized pressure~\eqref{eq_genP} follows the Laplace law, 
such that the coarsening in active model B defined from Eq.~\eqref{eq_neq_mu} is also driven by Ostwald ripening.

Overall, the phenomenology of active model B (AMB) is similar to that of phase separation at equilibrium as it also admits a generalized free-energy-like structure. 
However, the generalized thermodynamic variables defined in this section do not carry any obvious mechanical interpretation and should only be seen as 
a convenient calculation framework to study the active phase separation physics.
These similarities to equilibrium are thus mostly due to the fact that time-reversal symmetry is broken in AMB in a `minimal way',
while of course other nonequilibrium contributions can be added to the current~\eqref{eq_JD} which might have much more dramatic effect.

An example worth mentioning here is the active model B+~\cite{Tjhung2018PRX} for which the current~\eqref{eq_JD} includes a term
\begin{equation}\label{eq_AMB_plus}
\bm J^{\rm AMB+}_{\rm D} = \zeta(\phi) (\nabla^2\phi) \nabla\phi ,
\end{equation}
that cannot be written as deriving from a generalized chemical potential.
This new term changes qualitatively the structure of the problem as it 
allows for non-curl-free currents ($\nabla \times \bm J \ne \bm 0$) in the steady state.
Such feature leads to qualitative changes in the phenomenology of active phase separation,
in particular the study of curved interfaces shows that $\bm J^{\rm AMB+}_{\rm D}$ is responsible for the emergence of negative interfacial tension, 
leading to reversed Ostwald ripening selecting a preferred (finite) droplet radius.

