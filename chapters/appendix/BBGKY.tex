\label{appendxi: BBKGY}

We first recall the definition of the microscopic model:
\begin{subequations}
\begin{align} \label{eq_Langevin_ABP_APP}
    \odv{\bm r_i}{t} & = v_0 \hat {\bm e}(\theta_i) - \frac{1}{\zeta} \nabla_{\bm r_i} U(\{\bm r_j\}) + \sqrt{ 2 D } \bm \xi_i, \\
    \odv{\theta_i}{t} & = \sqrt{ 2 D_r } \chi_i.
\end{align}
\end{subequations}
See Eq.~\eqref{eq_Langevin_int_ABPs} for a definition of all the terms.

Below, we denote $N$ the total particle number and $\calP_N(\{\bm r_i,\theta_i\},t)$ the $N$-body particle distribution.
From standard stochastic calculus, $\calP_N$ obeys the Fokker-Planck equation
\begin{equation} \label{eq_FPN}
    \partial_t \calP_N = \sum_{i=1}^{N} \nabla_{\bm r_i}\cdot \left[ \left( \frac{1}{\zeta} \nabla_{\bm r_i}U - v_0 \hat{\bm e}(\theta_i) + D\nabla_{\bm r_i} \right)\calP_N\right] + D_r \sum_{i=1}^{N} \partial_{\theta_i\theta_i}^2 \calP_N ,
\end{equation}
where, to lighten notations, we make the dependencies of the distribution in the degrees of freedom and time implicit.

Equation~\eqref{eq_FPN} is exact. However, it is of limited practical use as it describes a system with $3N$ microscopic degrees of freedom with a partial differential equation (PDE) for a distribution with $3N$ independent variables. 
Our goal now is to derive a simpler description of the system, essentially by integrating out `fast' processes which do not affect the dynamics over large scales.
To achieve this, we now consider the single-body distribution
\begin{equation} \label{eq_FP_P1}
    \calP(\bm r,\theta,t) = N \int \prod_{j=2}^N [\rmd^2 \bm r_j \rmd \theta_j] \calP_N(\{\bm r,\theta,\bm r_2,\theta_2,\ldots,\bm r_N,\theta_N\},t),
\end{equation}
obtained by integrating $\calP_N$ over $3(N-1)$ degrees of freedom and where the $N$ factor on the rhs accounts for the fact that the particles are indistinguishable.  
Note that here for simplicity we have dropped the `1' subscript. 
Integrating Eq.~\eqref{eq_FPN} over $3(N-1)$ degrees of freedom, we find that $\calP$ obeys
\begin{equation} \label{eq_FP1}
    \partial_t \calP = \nabla \cdot \left[ - \bm J_{\rm eff} - v_0 \hat{\bm e}(\theta) \calP + D\nabla \calP \right] + D_r \partial_{\theta\theta}^2 \calP ,
\end{equation}
where the effective flux coming from the interaction term reads
\begin{align*}
    \bm J_{\rm eff} & = -\frac{1}{\zeta} N \int \prod_{j=2}^N [\rmd^2 \bm r_j \rmd \theta_j] \, \calP_N(\{\bm r,\theta,\ldots,\bm r_N,\theta_N\},t) \sum_{j \ne 1} \nabla u(|\bm r - \bm r_j|) \\
    & = -\frac{1}{\zeta} \int \rmd^2 \bm r' \, \tilde{\calP}_2(\bm r,\theta,\bm r',t) \nabla u(|\bm r - \bm r'|) \\
    & = -\frac{1}{\zeta} \int \rmd^2 \bm r' \, \tilde{\calP}_2(\bm r,\theta,\bm r',t) u'(|\bm r - \bm r'|) \frac{\bm r - \bm r'}{|\bm r - \bm r'|} ,
\end{align*}
where 
\begin{equation}
    \tilde{\calP}_2(\bm r,\theta,\bm r',t) = N(N-1) \int \rmd\theta' \prod_{j=3}^N [\rmd^2 \bm r_j \rmd \theta_j] \calP_N(\{\bm r,\theta,\bm r',\theta',\ldots,\bm r_N,\theta_N\},t) ,
\end{equation}
denotes the two body probability density to find a particle at position $\bm r$ with an orientation $\theta$ while another particle is at position $\bm r'$ with an arbitrary orientation. 
    
Due to the presence of pairwise interactions, the equation for $\calP$ is coupled to the two-body particle distribution. Similarly, if we had derived the equation for $\tilde{\calP}_2$ we would have found that it depends on the three-body distribution, etc. 
This structure commonly appears when one coarse-grains interacting systems, and is known as the \emph{BBGKY hierarchy}, where the acronym stands for Bogoliubov–Born–Green–Kirkwood–Yvon. To proceed further, we truncate the BBGKY hierarchy in order to get a closed equation for $\calP$. This can be done by means of various approximation schemes, the most common one being the molecular chaos hypothesis (or `Stosszahlansatz' for German speakers) which can be used to derive e.g.\ the Boltzmann equation for dilute gases. The molecular chaos assumption consists in factorizing the two-body particle distribution into the product of two single-body distributions.    
This amounts to assume that the positions and/or velocities of two particles on average decorrelate between collisions. This approximation works decently for low particle density, but of course becomes increasingly worse as the density becomes high.
Here, we therefore write instead
\begin{equation} \label{eq_P2}
    \tilde{\calP}_2(\bm r,\theta,\bm r',t) = \calP(\bm r,\theta,t) \, \rho(\bm r',t) \, g(|\bm r - \bm r'|,\varphi|\bm r,\theta,t),
\end{equation}
where $\rho$ denotes the particle density field, while $g(|\bm r - \bm r'|,\varphi|\bm r,\theta,t)$ 
is the conditional probability to find another 
particle at position $\bm r'$ given that there is another particle at position $\bm r$ moving along $\theta$, the angle $\varphi$ being that formed by the vectors $\bm r' - \bm r$ and $\hat{\bm e}(\theta)$.

Setting $g=1$ in Eq.~\eqref{eq_P2} corresponds to the molecular chaos assumption. 
Getting an explicit expression for $g$, however, would require to consider the dynamics of the two-body distribution. In general, such approach is quite tedious and we won't pursue it here.
For quantitative descriptions, a simpler approach is to measure $g$ directly from simulations of the microscopic dynamics~\eqref{eq_Langevin_ABP_APP}. 
Doing this in a dilute suspension, one finds that $g$ is generally maximal 
at $\varphi = 0$, i.e.\ that there is more chance for a tagged particle to find another one at its front (where here `front' and `back' are defined wrt the particle polarity $\hat{\bm e}(\theta)$) than behind it~\cite{Bialk_2013}.

Using~\eqref{eq_P2}, we now rewrite the effective current as
\begin{align}
    \bm J_{\rm eff} & = -\frac{1}{\zeta} \calP(\bm r,\theta,t) \int \rmd^2 \bm r' \, \rho(\bm r',t) g(|\bm r - \bm r'|,\varphi|\bm r,\theta,t) u'(|\bm r - \bm r'|) \frac{\bm r - \bm r'}{|\bm r - \bm r'|} , \\
    & = \frac{1}{\zeta} \calP(\bm r,\theta,t) \int_0^\infty s \rmd s \int_0^{2\pi} \rmd\varphi \, \rho(\bm r + s\hat{\bm e}(\theta + \varphi),t) g(s,\varphi|\bm r,\theta,t) u'(s) \hat{\bm e}(\theta + \varphi) ,
    \end{align}
where $\bm s = \bm r' - \bm r = s \hat{\bm e}(\theta + \varphi)$.
Assuming that $u$ is sufficiently short-ranged such that we can neglect variations of $\rho$ over the length scale of the interaction, while supposing that $g$ is homogeneous and stationary: $g(s,\varphi|\bm r,\theta,t) \simeq g(s,\varphi)$,
we moreover get that $\bm J_{\rm eff} = \tfrac{1}{\zeta} \calP(\bm r,\theta,t) \bm F_{\rm eff}(\bm r,\theta,t)$, with
\begin{equation}
    \bm F_{\rm eff}(\bm r,\theta,t) = -\rho(\bm r,t) \bm \omega(\theta) = \rho(\bm r,t) \int_0^\infty s \rmd s \, u'(s) \int_0^{2\pi} \rmd \varphi \,  g(s,\varphi) \bm \hat{\bm e}(\varphi+\theta).
\end{equation}
To simplify $\bm F_{\rm eff}$ further, we note that
\begin{equation}
    \hat{\bm e}_\perp(\theta)\cdot\bm \omega(\theta) = -\int_0^\infty s \rmd s \, u'(s) \int_0^{2\pi} \rmd \varphi \,  g(s,\varphi) \sin(\varphi) = 0.
\end{equation}
where $\hat{\bm e}_\perp(\theta) = \hat{\bm e}(\theta+\pi/2)$ is orthogonal to the self-propulsion direction of the tagged particle.
The cancelling of the intergal over $\varphi$ comes from the left-right symmetry of the active particles, which imposes that $g(s,\varphi)$ is an even function of $\varphi$.
We thys get
$\bm \omega(\theta) = \omega_\| \hat{\bm e}(\theta)$,
where the parameter $\omega_\|$ is given by
\begin{equation}
    \omega_\| = -\int_0^\infty s \rmd s \, u'(s) \int_0^{2\pi} \rmd \varphi \,  g(s,\varphi) \cos(\varphi).
\end{equation}

Putting back the explicit expression of $\bm J_{\rm eff}$ into Eq.~\eqref{eq_FP_P1}, we finally end up with a drift-diffusion-type equation:
\begin{equation} \label{eq_FP_P}
    \partial_t \calP = -\nabla \cdot \left[ v_{\rm eff}(\rho) \hat{\bm e}(\theta) \calP - D\nabla \calP \right] + D_r \partial_{\theta\theta}^2 \calP ,
\end{equation}
where $v_{\rm eff}(\rho) = v_0 - \omega_\|\rho/\zeta$ is the effective self-propulsion velocity at the mean field level.
\begin{itemize}
    \item The main effect of the presence of activity is to induce an anisotropic behavior of the correlation function $g$ w.r.t. the angle $\varphi$. As a consequence, activity renormalizes the self-propulsion velocity. In a passive system, one would indeed get $\bm \omega = 0$ by symmetry.
    \item For repulsive interactions between the particles $u'(r) < 0$. Moreover, we know that $g(r,\varphi)$ is maximal in the sector $-\tfrac{\pi}{2} \le \varphi \le \tfrac{\pi}{2}$ where $\cos(\varphi) \ge 0$, so one generally expects $\omega_\| > 0$. Therefore, our derivation reveals that the main effect of coupling self-propulsion with interactions is to slow down particles in dense regions ($v_{\rm eff} < v_0$ for $\rho > 0$). As one can already anticipate, this phenomenon amounts to having effective attractive interactions between the particles despite their absence at the microscopic level.
    Note that the linear decay of $v_{\rm eff}(\rho)$ was observed in numerical simulations of repelling ABPs~\cite{CatesMIPS}.
\end{itemize}
